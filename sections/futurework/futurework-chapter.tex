\section{Future work}\label{sec:futurework}
In the following section, we will define potential future work based on the information acquired.
In \autoref{sec:discussion}, we highlighted multiple possible directions for future work.

\subsection{Side-information to improve density}
The \textit{IS-UserBased} method shows potential for getting good results, but due to its item splitting approach, it creates a greater level of sparsity than already exists in the datasets.
To combat this sparsity, we decided to look into adding side-information to the items to re-establish this connection.
One way to do this is by adding side-information about the movie, such as director, actors, or genre as we have seen in the \textit{KGAT} method.
We theorize that adding this side-information could help improve the recommendations generated, since it allows re-connecting these fictional items through multiple relations.
Additionally, utilizing this along a multi-hop approach as seen in \textit{KGAT}, we may be able to infer that a user has a preference for a given genre or actor in a given context, rather than just a preference for a given movie.\\
For future work on this method, adding side-information to items or users may help alleviate this sparsity and reconnect the fictional items it creates.
Adding this side-information to other methods such as \textit{NGCF} or \textit{LightGCN} may also be interesting to look into.

\subsection{Adding context to existing methods}
It could be interesting to look at ways to add context to \textit{LightGCN} or \textit{NGCF}.
One way of doing this could be by employing methods to represent context in a format usable by the initial embeddings for these methods, and investigating results on the contextual datasets.

\subsection{Data discretization}
We defined certain categories for discretization.
A potential way to extend the results would be to employ different categories and examine changes in results.

\subsection{Collect better context datasets}
One of the major problems when working with CARS is the lack of datasets that have both useful contextual variables and a reasonable amount of interactions.
Currently, the most appropriate datasets for CARS are CoMoDa with 2,296 ratings and 12 contextual variables, or MovieLens 100k which has 100,000 ratings but only a timestamp as contextual information.
In order to further investigate CARS, it is necessary to gather more information with useful contextual information.
