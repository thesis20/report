\chapter{Introduction}\label{ch:introduction}
Recommender systems(RS) are widely used to combat information overload for users when accessing new information by recommending specific items that the user is likely to find compatible \cite{YouTubeNeural,IndustryPerspective}.
RS are conventionally defined through three categories - \textit{content-based}, \textit{collaborative filtering} and \textit{hybrid}.
\textit{Content-based} RS provide recommendations based on content information such as genre, \textit{collaborative filtering} recommend items based on similar users, and \textit{hybrid} combines the approaches\cite{ContextSurvey2020}.
\textit{Context-Aware Recommender Systems} (CARS) exist as a subset of RS, aiming to incorporate contextual information into the process of recommendation.
For a movie RS, the objective is to estimate a rating function $R: User \times Item \rightarrow Rating$ for the $(user, \ item)$ pairs that are not yet rated\cite{RecommenderHandbook2015}.
CARS extends this function to $R: User \times Item \times Context \rightarrow Rating$, taking the known contextual information associated with the application into account.
Contextual information can be classified into three categories: \textit{fully observable}, \textit{partially observable} and \textit{unobservable}.
Fully observable contextual information is available explicitly, partially observable defines that only some of the information is available, and unobservable defines that no contextual information is explicitly available.
The structure of contextual information can remain stable over time, defined as static, or change in some way over time, defined as dynamic.

\section{Examples}
You can also have examples in your document such as in example~\ref{ex:simple_example}.
\begin{example}{An Example of an Example}
  \label{ex:simple_example}
  Here is an example with some math
  \begin{equation}
    0 = \exp(i\pi)+1\ .
  \end{equation}
  You can adjust the colour and the line width in the {\tt macros.tex} file.
\end{example}

\section{How Does Sections, Subsections, and Subsections Look?}
Well, like this
\subsection{This is a Subsection}
and this
\subsubsection{This is a Subsubsection}
and this.

\paragraph{A Paragraph}
You can also use paragraph titles which look like this.

\subparagraph{A Subparagraph} Moreover, you can also use subparagraph titles which look like this\todo{Is it possible to add a subsubparagraph?}. They have a small indentation as opposed to the paragraph titles.

\todo[inline,color=green]{I think that a summary of this exciting chapter should be added.}
