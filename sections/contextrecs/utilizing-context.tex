\subsection{Utilizing and obtaining context}\label{subsec:utilizingandobtainingcontext}
The contextual information used for CARS can be obtained in several ways: \textit{Explicitly}, \textit{implicitly} and \textit{inferring}\cite{RecommenderHandbook2015}.
Obtaining the data is explicitly done by approaching relevant people and asking direct questions, or, for example, eliciting the information through a questionnaire.
Implicit collection is done from the data or environment, such as a change in the location of the user, or a time derived from a timestamp.
Finally, inferring the contextual information is done through statistical or data mining methods where the information is modeled as a latent variable, for example by analyzing a customer review or inferring who is watching TV by the programs being watched.
The relevance of each contextual factor can differ.
Identifying a candidate set of factors for an application requires leveraging domain expertise from a designer, or computing relevance computationally.
This context is dealt with in one of three ways\cite{Adomavicius2011}: \textit{Contextual pre-filtering}, \textit{contextual post-filtering}, and \textit{contextual modelling}.
In the following section, we will investigate each of these and briefly explain how they deal with contextual information.

\subsubsection*{Pre-filtering}
For all three approaches, we will look at the rating prediction task and start with data in the form $U \times I \times C \rightarrow R$ where $C$ is the additional contextual dimension.\\
Applying the filtering approaches to top-$N$ generation should be trivial.
With contextual pre-filtering, the data selected for the recommendation process is filtered based on the current context $c$.
This leads to recommendations being based on a subset of the original data, where we only consider the items observed in the current context.\\
An advantage of this approach is that it allows the use of any of the traditional recommendation techniques\cite{Adomavicius2011}.
The filtering step can be done either as an \textit{exact pre-filter}, where the context has to fit the exact context of other interactions, such as being made on a \textit{Sunday} in the company of the user's \textit{Partner}.
Alternatively, generalized pre-filtering can be used\cite{Adomavicius2011}, in which contextual variables can be seen as hierarchical items, for example:
\begin{itemize}
	\item \textit{Company}: Partner $\rightarrow$ Family $\rightarrow$ Friends $\rightarrow$ AnyCompany
	\item \textit{Time}: Sunday $\rightarrow$ Weekend $\rightarrow$ AnyTime
\end{itemize}
Utilizing this hierarchy structure, we can for example make the following generalizations for the exact pre-filter example (Sunday, Partner):
\begin{itemize}
	\item (Partner, Sunday)
	\item (Partner, Weekend)
	\item (Family, Weekend)
	\item (Friends, AnyTime)
\end{itemize}
Choosing the correct generalized pre-filter is important for accurate recommendations, and can either be done manually or by empirically evaluating the performance of the system based on each generalized pre-filter.

\subsubsection*{Post-filtering}
In contextual post-filtering, the contextual information is, at first, ignored while generating recommendations.
Unlike pre-filtering, the recommendations will be made based on the full dataset, and then after the recommendations have been generated, they will be filtered based on the current context $c$.\\
It is possible to do two kinds of post-filtering: Either you remove the predicted ratings that do not fit the context if the method produces predictions for all items for all possible contexts, such that the user is only recommended items that correlate with the context $c$, or adjust the ranking of the items based on $c$.

\subsubsection*{Contextual modelling}
Contextual modelling is slightly different than the two other approaches, in that it does not use the context for filtering, but rather the contextual information is used directly in the modelling technique.
In traditional recommendation systems ratings are of the form $Rating = R(User, \: Item)$, but adding the contextual information to the actual model means that we modify this to $Rating = R(User, \: Item, \: Context)$.
This means that traditional recommendation systems will not work on this set, since it has an extra dimension that they do not account for, however, multiple models have been suggested to fit this form\cite{Adomavicius2011}.
\\\\
According to a comparison study conducted by Campos et. al\cite{10.1007/978-3-642-39878-0_13}, which aimed to see which of these three approaches gave the better results, there is not a single approach that is clearly superior to the others.
Rather, they observed that the performance depends to a large extent on the particular combination of the contextualization approach and the underlying recommendation used.\\
Additionally, they conclude that using all available context information does not necessarily result in the best recommendations since this will introduce higher dimensionality, leading to even sparser data.

\subsubsection*{Discretizing the context}
Some context variables can be very specific, such as a timestamp, which will usually include the year, month, day, hour, minute, and even seconds for when the rating was made.
In order for this information to be more useful, it needs to be discretized.
For time units such as hours and smaller units, this can be done by discretizing it into categories such as \textit{morning (06.00-11.00}), \textit{evening (11.00-14.00)} and so on.
This allows the recommender to better group ratings together, since the movies a user watches at 13.00 may not be very different from 14.00, but it may differ from morning time to night time.
The timestamp can also be split into categories such as whether it is a weekend or not or the current season at the time of the interaction.

