\subsection{Synthetic data}
Datasets where the ratings are context-dependent are needed in order to enable testing of context-aware recommender systems.
One approach is to make a semi-synthetic dataset by adding a new contextual feature that the ratings will depend on\cite{baltrunasContextItemSplit}.
For this approach, the idea is to add a new contextual feature \textit{c} that represents a contextual condition that could affect the rating.
The value of \textit{c} is either 0 or 1. 
If \textit{c} is 1, the corresponding rating between 1 and 5 is increased by 1, if it is not already 5.
If \textit{c} is 0, the rating is instead decreased by 1 if is not already 1.
An $\alpha$ value is then chosen between 0 and 1 which determines the fractions of ratings that are affected by the contextual feature \textit{c}.
By using this method, one can modify a dataset into being artificially context dependent for testing purposes.
This approach can be seen in \autoref{tab:synthetic-data1} and \autoref{tab:synthetic-data2}.
\begin{table*}[hbt!]
    \centering
    \begin{tabular}{|c|c|c|}
    \hline
    UserId & MovieId & Rating \\ [0.5ex] 
    \hline\hline
    1 & 101 & 3 \\
    \hline
    2 & 102 & 4 \\
    \hline
    \end{tabular}
    \caption{Ratings table without the contextual feature.}
    \label{tab:synthetic-data1}
\end{table*}
\begin{table*}[hbt!]
    \centering
    \begin{tabular}{|c|c|c|c|} 
    \hline
    UserId & MovieId & Rating & Contextual\\Feature \\ [0.5ex] 
    \hline\hline
    1 & 101 & 2 & 0 \\
    \hline
    2 & 102 & 5 & 1 \\
    \hline
    \end{tabular}
    \caption{Ratings table with the contextual feature.}
    \label{tab:synthetic-data2}
\end{table*}
