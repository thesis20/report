\subsubsection{DeepWalk}
\textit{DeepWalk} is an algorithm that uses a deep learning technique to learn social representations of a graph's nodes\cite{DeepWalk}.
For the experiments we use our own implementation.
It makes use of neural language models where the language consists of "sentences" generated by doing random walks between the nodes of the graph.
The random walk is based on an assumption that adjacent nodes are similar and should have similar embeddings.
By doing this, latent features that capture the neighborhood similarity of the nodes can be learned.
This results in a low dimensional vector representation of each node in the graph.
The random walks are done a number of times for each node $v_i$.
The nodes in the walk are chosen at random to form the random walks $W_{vi}$ which are used as "sentences" in the neural language model and the nodes are the "words" of the sentences.
Thus, a random walk is a sequence of nodes.
The language model is then used to estimate the likelihood that nodes appear in the same random walk.
This is done using the language model \textit{SkipGram}, and the result of this is a vector representation of each node in $d$-dimensional space.
\\\\
The next step is to then use the vector representations to generate recommendations.
Here we opted to go for a k-nearest neighbor (kNN) approach where we find the $k$ nearest user nodes to a user node $u$ using the vector representations.
To then generate a rating prediction for a specific item $i$, we look at whether the $k$ neighbors have rated item $i$ and use the average of their ratings of $i$ as the predicted rating for user $u$ on item $i$.
