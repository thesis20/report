\chapter{Recommender systems (RS)}\label{ch:introduction}
are widely used to combat information overload for users when accessing new information by recommending specific items that the user is likely to find compatible\cite{YouTubeNeural,IndustryPerspective}.
RS are conventionally defined through three categories: \textit{content-based}, \textit{collaborative filtering} and \textit{hybrid}.
\textit{Content-based} RS provide recommendations based on content information such as genre, \textit{collaborative filtering} recommend items based on similar users, and \textit{hybrid} combines the two prior approaches\cite{ContextSurvey2020}.
\textit{Context-Aware Recommender Systems} (CARS) exist as a subset of RS, aiming to incorporate contextual information into the process of recommendation.
Contextual information is intuitively useful for making recommendations.
In a movie recommendation situation this context could be company, time, or mood.
For example, a user may not want to watch the same movies in the company of their significant other, as they would with their parents.
Likewise, time could be an important factor, since movies such as \textit{A Christmas Story} may be more relevant during the winter months than during summer.\\\
Generally, for a RS, the objective is to either: estimate a rating function $R: User \times Item \rightarrow Rating$ for the $(user, \ item)$ pairs that are not yet rated, or to compute a list $L$ of the top recommended items for an active user $U_a$ containing items that are likely to be of interest $R: User \times Item \rightarrow L(Ua)$\cite{RecommenderHandbook2015}.
CARS extends these functions to respectively $R: User \times Item \times Context \rightarrow Rating$ and $R: User \times Item \times Context \rightarrow L(Ua)$ by taking the known contextual information associated with the application into account.
