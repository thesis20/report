%%%%%%%%%%%%%%%%%%%%%%%%%%%%%%%%%%%%%%%%%
% Journal Article
% LaTeX Template
% Version 1.4 (15/5/16)
%
% This template has been downloaded from:
% http://www.LaTeXTemplates.com
%
% Original author:
% Frits Wenneker (http://www.howtotex.com) with extensive modifications by
% Vel (vel@LaTeXTemplates.com)
%
% License:
% CC BY-NC-SA 3.0 (http://creativecommons.org/licenses/by-nc-sa/3.0/)
%
%%%%%%%%%%%%%%%%%%%%%%%%%%%%%%%%%%%%%%%%%

%----------------------------------------------------------------------------------------
%	PACKAGES AND OTHER DOCUMENT CONFIGURATIONS
%----------------------------------------------------------------------------------------

\documentclass[twoside,twocolumn]{article}

\usepackage{blindtext} % Package to generate dummy text throughout this template 

\usepackage[sc]{mathpazo} % Use the Palatino font
\usepackage[T1]{fontenc} % Use 8-bit encoding that has 256 glyphs
\linespread{1.05} % Line spacing - Palatino needs more space between lines
\usepackage{microtype} % Slightly tweak font spacing for aesthetics

\usepackage[english]{babel} % Language hyphenation and typographical rules

\usepackage[hmarginratio=1:1,top=32mm,columnsep=20pt]{geometry} % Document margins
\usepackage[hang, small,labelfont=bf,up,textfont=it,up]{caption} % Custom captions under/above floats in tables or figures
\usepackage{booktabs} % Horizontal rules in tables

\usepackage{lettrine} % The lettrine is the first enlarged letter at the beginning of the text

\usepackage{enumitem} % Customized lists
\setlist[itemize]{noitemsep} % Make itemize lists more compact

\usepackage{abstract} % Allows abstract customization
\renewcommand{\abstractnamefont}{\normalfont\bfseries} % Set the "Abstract" text to bold
\renewcommand{\abstracttextfont}{\normalfont\small\itshape} % Set the abstract itself to small italic text

\usepackage{titlesec} % Allows customization of titles
\renewcommand\thesection{\Roman{section}} % Roman numerals for the sections
\renewcommand\thesubsection{\roman{subsection}} % roman numerals for subsections
\titleformat{\section}[block]{\large\scshape\centering}{\thesection.}{1em}{} % Change the look of the section titles
\titleformat{\subsection}[block]{\large}{\thesubsection.}{1em}{} % Change the look of the section titles

\usepackage{fancyhdr} % Headers and footers
\pagestyle{fancy} % All pages have headers and footers
\fancyhead{} % Blank out the default header
\fancyfoot{} % Blank out the default footer
\fancyhead[C]{mi902e20 $\bullet$ spMI-1 2020} % Custom header text
\fancyfoot[RO,LE]{\thepage} % Custom footer text

\usepackage{titling} % Customizing the title section

\usepackage{hyperref} % For hyperlinks in the PDF

% Bibliography
% http://en.wikibooks.org/wiki/LaTeX/Bibliography_Management
%%%%%%%%%%%%%%%%%%%%%%%%%%%%%%%%%%%%%%%%%%%%%%%%
% Add the \citep{key} command which display a
% reference as [author, year]
\usepackage[backend=bibtex,
    bibencoding=utf8
]{biblatex}
\addbibresource{bib/mybib}
\usepackage{csquotes}
\usepackage[nottoc]{tocbibind}
\usepackage{amsmath}
\usepackage{tikz} % Used to create bipartite graph
\usepackage{pgfplots}
\pgfplotsset{width=7cm,compat=1.8}
\usetikzlibrary{positioning,chains,fit,shapes,calc}

\definecolor{myblue}{RGB}{80,80,160}
\definecolor{mygreen}{RGB}{80,160,80}

\usepackage{amsmath}


%----------------------------------------------------------------------------------------
%	TITLE SECTION
%----------------------------------------------------------------------------------------

\setlength{\droptitle}{-4\baselineskip} % Move the title up

\pretitle{\begin{center}\Huge\bfseries} % Article title formatting
\posttitle{\end{center}} % Article title closing formatting
\title{Context in Recommender Systems} % Article title
\author{%
\textsc{Andreas Stenshøj, Daniel Moesgaard Andersen, Rasmus Bundgaard Eduardsen} \\[1ex] % Your name
\normalsize Aalborg University \\ % Your institution
\normalsize \href{mailto:astens16@student.aau.dk}{astens16@student.aau.dk} % Your email address
\normalsize \href{mailto:dand16@student.aau.dk}{dand16@student.aau.dk} % Your email address
\normalsize \href{mailto:reduar16@student.aau.dk}{reduar16@student.aau.dk} % Your email address
%\and % Uncomment if 2 authors are required, duplicate these 4 lines if more
%\textsc{Jane Smith}\thanks{Corresponding author} \\[1ex] % Second author's name
%\normalsize Aalborg University \\ % Second author's institution
%\normalsize \href{mailto:jane@smith.com}{jane@smith.com} % Second author's email address
}
\date{\today} % Leave empty to omit a date
\renewcommand{\maketitlehookd}{%
\begin{abstract}
\noindent How do graph-based methods compare to non-graph based methods for the recommender system domain, and how does context influence the accuracy of predicted items? 
This paper investigates state-of-the-art within context-aware and graph-based recommender system, datasets available for context-aware methods as well as useful evaluation metrics.
The outcome is a comparison of the identified methods, showing performance on a broad range of evaluation metrics on relatively small datasets.
The results show that lol.
\end{abstract}
}

%----------------------------------------------------------------------------------------

\begin{document}

% Print the title
\maketitle

%----------------------------------------------------------------------------------------
%	ARTICLE CONTENTS
%----------------------------------------------------------------------------------------
\chapter{WIP}
Here goes the generic intro about how there is too much data and we need help with information overload.
\section{Context in recommender systems}
Context in my recommender system? It is more likely and effective than you think.

\subsection{Utilizing and obtaining context}\label{subsec:utilizingandobtainingcontext}
The contextual information used for CARS can be obtained in several ways: \textit{Explicitly}, \textit{implicitly} and \textit{inferring}\cite{RecommenderHandbook2015}.
Obtaining the data is explicitly done by approaching relevant people and asking direct questions, or, for example, eliciting the information through a questionnaire.
Implicit collection is done from the data or environment, such as a change in the location of the user, or a time derived from a timestamp.
Finally, inferring the contextual information is done through statistical or data mining methods where the information is modeled as a latent variable, for example by analyzing a customer review or inferring who is watching TV by the programs being watched.
The relevance of each contextual factor can differ.
Identifying a candidate set of factors for an application requires leveraging domain expertise from a designer, or computing relevance computationally.
This context is dealt with in one of three ways\cite{Adomavicius2011}: \textit{Contextual pre-filtering}, \textit{contextual post-filtering}, and \textit{contextual modelling}.
When dealing with context it might also be necessary to discretize certain values.
In the following section, we will investigate each of the methods for handling context and briefly explain how they deal with contextual information as well as define when discretizing context is useful.

\subsubsection*{Pre-filtering}
For all three approaches, we will look at the rating prediction task and start with data in the form $U \times I \times C \rightarrow R$ where $C$ is the additional contextual dimension.\\
Applying the filtering approaches to top-$N$ generation should be trivial.
With contextual pre-filtering, the data selected for the recommendation process is filtered based on the current context $c$.
This leads to recommendations being based on a subset of the original data, where we only consider the items observed in the current context.\\
An advantage of this approach is that it allows the use of any of the traditional recommendation techniques\cite{Adomavicius2011}.
The filtering step can be done either as an \textit{exact pre-filter}, where the context has to fit the exact context of other interactions, such as being made on a \textit{Sunday} in the company of the user's \textit{Partner}.
Alternatively, generalized pre-filtering can be used\cite{Adomavicius2011}, in which contextual variables can be seen as hierarchical items, for example:
\begin{itemize}
	\item \textbf{Company}: Partner $\rightarrow$ Family $\rightarrow$ Friends $\rightarrow$ AnyCompany
	\item \textbf{Time}: Sunday $\rightarrow$ Weekend $\rightarrow$ AnyTime
\end{itemize}
Utilizing this hierarchy structure, we can for example make the following generalizations for the exact pre-filter example (Sunday, Partner):
\begin{itemize}
	\item (Partner, Sunday)
	\item (Partner, Weekend)
	\item (Family, Weekend)
	\item (Friends, AnyTime)
\end{itemize}
Choosing the correct generalized pre-filter is important for accurate recommendations, and can either be done manually or by empirically evaluating the performance of the system based on each generalized pre-filter.

\subsubsection*{Post-filtering}
In contextual post-filtering, the contextual information is, at first, ignored while generating recommendations.
Unlike pre-filtering, the recommendations will be made based on the full dataset, and then after the recommendations have been generated, they will be filtered based on the current context $c$.\\
It is possible to do two kinds of post-filtering: Either you remove the predicted ratings that do not fit the context if the method produces predictions for all items for all possible contexts, such that the user is only recommended items that correlate with the context $c$, or adjust the ranking of the items based on $c$.

\subsubsection*{Contextual modelling}
Contextual modelling is slightly different than the two other approaches, in that it does not use the context for filtering, but rather the contextual information is used directly in the modelling technique.
In traditional recommendation systems ratings are of the form $Rating = R(User, \: Item)$, but adding the contextual information to the actual model means that we modify this to $Rating = R(User, \: Item, \: Context)$.
This means that traditional recommendation systems will not work on this set, since it has an extra dimension that they do not account for, however, multiple models have been suggested to fit this form\cite{Adomavicius2011}.
\\\\
According to a comparison study conducted by Campos et. al\cite{10.1007/978-3-642-39878-0_13}, which aimed to see which of these three approaches gave the better results, there is not a single approach that is clearly superior to the others.
Rather, they observed that the performance depends to a large extent on the particular combination of the contextualization approach and the underlying recommendation used.\\
Additionally, they conclude that using all available context information does not necessarily result in the best recommendations since this will introduce higher dimensionality, leading to even sparser data.

\subsubsection*{Discretizing the context}
Some context variables can be very specific, such as a timestamp, which will usually include the year, month, day, hour, minute, and even seconds for when the rating was made.
In order for this information to be more useful, it needs to be discretized.
For time units such as hours and smaller units, this can be done by discretizing it into categories such as \textit{morning (06.00-11.00}), \textit{evening (11.00-14.00)} and so on.
This allows the recommender to better group ratings together, since the movies a user watches at 13.00 may not be very different from 14.00, but it may differ from morning time to night time.
The timestamp can also be split into categories such as whether it is a weekend or not or the current season at the time of the interaction.


\section{Graph recommender systems}\label{sec:graph-rec-sys}
One of the problems often encountered in recommender systems is the problem of sparsity in larger recommender systems.
Two different kinds of approaches are often used to combat this. 
The first is to use dimensionality reduction where the representation of users and items are converted into a more compact representation, which should capture the most important features that describe the user or item.
An example of this is the matrix factorization technique, where a matrix $U \times I$ is decomposed into the product of two matrices $U \times K $ and $ K \times I$ where $K$ is the number of features to capture.
These representations are able to capture meaningful information about the relations between users and items, even if users have rated different items, or items were rated by different users. 
The other approach used for the sparsity problem is the use of graph representations for the data.
When using graph representations for the data, graph-based methods are able to utilize the transitivity in the relations between the data points. 
This means that similarity can be found between users that are not directly connected, which can help alleviate the sparsity problem.
Graph representations are also able to preserve the local relations as they have the edges between them that show the local neighborhood and how closely connected the nodes are, compared to the dimensionality reduction approach where some of this local information is lost as a result of the feature extraction\cite{RecommenderHandbook2015}.
\\\\
A typical way to represent graphs in recommender systems is using bipartite graphs.
In a bipartite graph, there are two distinct sets of nodes.
One set of nodes is the users and the other set is the items.
The edges in the bipartite graph connect the user and item nodes if the user has rated the item. 
The edge can have the rating as its weight if a rating is given.
An example of how a bipartite graph could look can be seen on \autoref{fig:bipartite-graph}.
\begin{figure}[h]
\begin{tikzpicture}[thick,
    every node/.style={draw,circle},
    fsnode/.style={fill=myblue},
    ssnode/.style={fill=mygreen},
    every fit/.style={ellipse,draw,inner sep=-2pt,text width=2cm},
    shorten >= 3pt,shorten <= 3pt
  ]
  
  % the vertices of U
  \begin{scope}[start chain=going below,node distance=7mm]
  \foreach \i in {u1,u2,u3}
    \node[fsnode,on chain] (f\i) [label=left: \i] {};
  \end{scope}
  
  % the vertices of I
  \begin{scope}[xshift=4cm,start chain=going below,node distance=7mm]
  \foreach \i in {i1,i2,i3,i4}
    \node[ssnode,on chain] (s\i) [label=right: \i] {};
  \end{scope}
  
  % the set U
  \node [myblue,fit=(fu1) (fu3),label=above:$U$] {};
  % the set I
  \node [mygreen,fit=(si1) (si4),label=above:$I$] {};
  
  % the edges
  \draw (fu1) -- (si1);
  \draw (fu1) -- (si4);
  \draw (fu2) -- (si1);
  \draw (fu2) -- (si3);
  \draw (fu3) -- (si3);
  \draw (fu3) -- (si2);
\end{tikzpicture}
\caption{Example of a bipartite graph}
\label{fig:bipartite-graph}
\end{figure}

As mentioned, with the use of graph representations, users and items not directly connected can influence each other.
There are two properties that are often explored for graph-based approaches when doing similarity measures, namely propagation and attenuation.
Propagation is done through propagating information along the edges of the graph, also using the weight of the edge to determine how much information is allowed to pass through.
Attenuation considers that nodes that are further away from each other in the graph should influence each other less.
This transitive association in graph-based approaches can be used in two ways to recommend items.
The relevance between a user \textit{u} and an item \textit{i} can be evaluated using the proximity of \textit{u} to the item \textit{i}, meaning that the items that are "closest" in the graph to \textit{u} will be recommended.
A similarity between a pair of users or items can also be found which can be expressed as a weight between them\cite{RecommenderHandbook2015}.

\subsection{Side-information in graphs}
With the structure of a graph, it is possible to extend it with side-information.
This can be done either with information about the interaction between user and an item by adding side-information to the edge between the two nodes, such that the edge $(u, i)$ that previously indicated an interaction between a user and an item is extended with side-information such as $(u, r, i)$, where $r$ is a rating that user $u$ has assigned to item $i$\cite{aggarwal2016recommender}. 
Additionally, side-information can be added to user or item nodes, which allows us to do constraints on the recommendations, similar to how contextual filtering works as described in \autoref{subsec:utilizingandobtainingcontext}.
For users, this side-information can be attributes such as their age, location or gender.
For a movie, it can be attributes such as actors, genre or budget.
Adding this side-information allows us to connect items through other nodes than just user nodes and vice-versa.


\subsection{Neural Graph Collaborative Filtering}
\textit{Neural Graph Collaborative Filtering} (NGCF) is a state-of-the-art recommendation framework for collaborative filtering, exploiting a user-item bipartite graph structure by propagating embeddings on it.
Learnable collaborative filtering models contain two key components: embedding, which transforms users and items to representations, and interaction modeling, which reconstructs interactions based on these embeddings.
NGCF aims to enhance recommendation by integrating user-item interactions into the embedding function, which has previously been built with descriptive features only, such as ID and attributes.
It does this through three components: an embedding layer that initializes user and item embeddings, embedding propagation layers that refine the embeddings and a prediction layer aggregating the refined embeddings.
An initial embedding for a user $u$ is described as an embedding vector $e_u \in \R^d$, where $d$ is the embedding size.
Embeddings are propagated through two operations, construction and aggregation.
The embedding from item $i$ to user $u$ is defined as
\begin{equation}
    m_{u \leftarrow i} = \frac{1}{\sqrt{|N_u||N_i|}} (W_1e_i + W_2(e_i \odot e_u))
\end{equation}
where $m_{u \leftarrow i}$ is the information to be propagated, $N_u$ and $N_i$ are the first-hop neighbors of user $u$ and item $i$, $W_1, W_2 \in \R^{(d' \times d)}$ are trainable weight matrices, $d'$ is the transformation size and $\odot$ is the element-wise product.
The element-wise product ensures that the interaction between the embedding of item $i$ and user $u$ contribute to the result.
Aggregation for high-order propagation is defined through the equation:

\begin{equation}
    e_{u}^{(l)} = \textrm{LeakyReLU}(m_{u \leftarrow u} + \sum_{i \in N_u} m_{u \leftarrow i})
\end{equation}
where $e_u^{(l)}$ is the representation of $u$ obtained at the propagation layer and LeakyReLU is the activation function.


\begin{math}
    \left\{
      \begin{array}{l}
        m_{u \leftarrow i}^{(l)} = \frac{1}{\sqrt{|N_u||N_i|}} (W_1^{(l)} e_i^{(l-1)} + W_2^{(l)} (e_i^{(l-1)} \odot e_u^{(l-1)}))\\
        m_{u \leftarrow u}^{(l)} = W_1^{(l)}e_u^{(l-1)}
      \end{array}
    \right.
  \end{math}

\begin{equation}
    Loss = \sum_{(u, i, j) \in O} - \textrm{ln} \: \sigma (\hat{y}_{ui} - \hat{y}_{uj}) + \lambda ||\Theta||_2^2
\end{equation}

\section{Experiments}\label{sec:experiments}
This section will describe our experiments and findings regarding utilizing context and graphs in recommender systems.
The section is structured as follows: First we will describe the state-of-the-art and baseline methods that have been investigated in \autoref{subsec:sotaoverview}.
We will then look at various datasets containing context, and briefly describe each of them in \autoref{subsec:desc-of-datasets}.
Following that we will look at various metrics used to evaluate recommender systems in \autoref{sec:evaluationmetrics}, and how these state-of-the-art methods compare to each other and baseline methods using these metrics in \autoref{subsec:resultsofexperiment}.

\subsection{State-of-the-art methods investigated}\label{subsec:sotaoverview}
This section provides an in depth invistigation of the different methods that are examined in this paper.
\subsubsection{IS-UserBased-Graph}\label{method:IS-UserBased-Graph}
The \textit{IS-UserBased-Graph} method was first proposed by Tu Minh Phuong et. al in \cite{GraphBasedCollaborativePaper}.
As no implementation was available for this method , all results in later sections are based on our implementation.\\
The method utilizes a contextual modeling method called item splitting, in which each item is split into multiple fictional items to create a fictional item for each item and context variation.
\\
For example, if we consider the movie \textit{Die Hard}, and the contextual variable "watching with" which can have a value of either \textit{partner, family} or \textit{friends}, we will generate the following fictional items: \textit{(Die Hard, Partner)}, \textit{(Die Hard, Family)}, and \textit{(Die Hard, Friends)}.
Based on these fictional items, a bipartite graph is generated with the two sets $U$ for users and $I$ for fictional items, with each edge containing the rating that a user has rated a given fictional item.\\
To generate recommendations, the method utilizes a user similarity matrix to find the $N$ most similar users to the current user we are trying to generate recommendations for.\\
They present two ways to do this: Either by using matrix factorization or by using a spreading activation algorithm on the bipartite graph.
The spreading activation works by starting out in the active users' node, from which you search through the graph, calculating similarity for other users based on common item nodes and distance from the root node.
Both of these methods results in a list of the top-$N$ similar users, from which it is possible to estimate the missing rating $\hat{r}_{ui}$ of the current user $u$ for a given item $i$, as seen on \Cref{eqn:israj}.
\begin{equation}
    \label{eqn:israj}
    \hat{r}_{ui} = \frac{\sum_{r_{ui} \in R_u} r_{ui}}{|R_u|}, R_u = \{ r_{ui} | r_{ui} \neq \emptyset\}
\end{equation}
The method takes 2 parameters that influence the performance.
First is the \textit{L} value that indicates how many steps away from the current user you can move to calculate similarity by utilizing transitive associations, meaning that you take multiple steps away from the current node in a graph.
Second is the \textit{k} value which is the delimiter used for the \textit{kNN} algorithm in the next step.
After calculating the missing ratings, the last step is to map the fictitious items back to real items with their contextual information.
For example, if the fictional item \textit{(Die Hard, Family)} was the top recommendation, this would need to be mapped back to the movie Die Hard with the observed contextual variable \textit{watch with family}.\\
Finally, we apply post-filtering and remove all movies from the top-$N$ recommendations that do not fit the user's current context.\\
With this we now have a list of the top-$N$ recommended movies for the user $u$ in their given contextual situation.

\subsection{Context-Aware Matrix Factorization}
One of the context-aware recommender system methods we have looked into is context-aware matrix factorization (CAMF)\cite{baltrunasCAMF}.
As no implementation was available for the method, all results are based on our implementation.
The method is an extension of the classical Matrix Factorization (MF) approach which takes the contextual side information into account when making rating predictions.
There are different models of CAMF that deal with different levels of granularity in terms of the context.
For CAMF-C, it is assumed that each of the contextual conditions has a global influence on the ratings independently of the items.
In the CAMF-C model, a single parameter is introduced for each value of each contextual factor.
The second model is CAMF-CI which introduces a parameter for each contextual condition and item pair, meaning that it has a finer granularity.
By modeling it like this, it is possible to capture when the contextual factors have a different effect on the rating depending on which item it is.
The expression for predicting ratings with the CAMF models is seen in \autoref{eqn:camf-rating-pred}.
\begin{equation}
    \label{eqn:camf-rating-pred}
    \hat{r}_{uic_1...c_k} = \vec{v_u} * \vec{q_i} + \bar{i} + b_u + \sum\limits_{j = 1}^k B_{ijc_j}
\end{equation}
We have that $\hat{r}_{uic_1...c_k}$ is the predicted rating for user $u$, item $i$ under the contextual values $c_1...c_k$.
The variables $\vec{v_u} $ and $ \vec{q_i}$ are the feature vectors for user $u$ and item $i$ which are multiplied, just as in matrix factorization.
$\bar{i}$ is the global average rating for item $i$, meaning the average rating across all users in the training set, and $\bar{b_u}$ is a user bias.
$B_{ijc_j}$ are the parameters modeling the interaction of the contextual conditions and the items.
For the CAMF-CI model, $B_{ijc_j}$ will result in a lot of parameters, and for CAMF-C it will result in less, as a result of how the two different approaches perform the contextual modeling.
All of the parameters in the model, except for the average item rating, are learned through stochastic gradient descent.
This is done through the use of a loss function which is seen in \autoref{eqn:camf-loss-func}.
\begin{equation}
    \label{eqn:camf-loss-func}
    \begin{split}
        \min_{v*, q*, b*, B*}\sum \limits_{r \in  R}\left [ \left (  \hat{r}_{uic_1...c_k} - \vec{v_u} * \vec{q_i} - \bar{i} - b_u - \sum\limits_{j = 1}^k B_{ijc_j}\right )^2 \right. \\
        \left. + \lambda \left({b_u}^2 +{\left \| \vec{v_u} \right \|}^2  + {\left \|\vec{q_i}  \right \|}^2 + \sum\limits_{j = 1}^k \sum\limits_{c_j = 1}^{z_j} B_{ijc_j}^{2}\right ) \right ]
    \end{split}
\end{equation}
Here we have that $r = (u,i,c_1...c_k)$ and that R is the context-dependent ratings from the training set.
The loss function includes regularization controlled by the $\lambda$ parameter to avoid overfitting the training data.
The parameters are then updated based on the gradient of the loss function for each of the ratings in the training set.

\subsubsection{Neural Graph Collaborative Filtering}\label{subsec:ngcf-desc}
\textit{Neural Graph Collaborative Filtering} (NGCF) is a method for collaborative filtering, employing a user-item bipartite graph structure.
For the experiments we use the official implementation of the paper.
Learnable collaborative filtering models contain two key components: embedding, which transforms users and items to representations, and interaction modeling, which reconstructs interactions based on these embeddings.
\textit{NGCF} aims to enhance recommendation by integrating user-item interactions into the embedding function, which has previously been built with descriptive features only, such as ID and attributes.
It does this through three components: an embedding layer that initializes user and item embeddings, embedding propagation layers that refine the embeddings, and a prediction layer aggregating the refined embeddings.
An initial embedding for a user $u$ is described as an embedding vector $e_u \in \R^d$, where $d$ is the embedding size.
Embeddings are propagated through two operations, construction and aggregation.
The embedding from item $i$ to user $u$ is defined in \autoref{eqn:ngcf-embed}:
\begin{equation}\label{eqn:ngcf-embed}
    m_{u \leftarrow i} = \frac{1}{\sqrt{|N_u||N_i|}} (W_1e_i + W_2(e_i \odot e_u))
\end{equation}
where $m_{u \leftarrow i}$ is the information to be propagated, $N_u$ and $N_i$ are the first-hop neighbors of user $u$ and item $i$, which are the neighbors that can be reached in one step from the node, $W_1, W_2 \in \R^{(d' \times d)}$ are trainable weight matrices, $d'$ is the transformation size and $\odot$ is the element-wise product.
The element-wise product ensures that the interaction between the embedding of item $i$ and user $u$ contributes to the result.
Aggregation for multiple layers is defined in \autoref{eqn:ngcf-agg}:
\begin{equation}\label{eqn:ngcf-agg}
    e_{u}^{(l)} = \textrm{LeakyReLU}(m_{u \leftarrow u} + \sum_{i \in N_u} m_{u \leftarrow i})
\end{equation}
where $e_u^{(l)}$ is the representation of $u$ obtained at the layer and LeakyReLU is the activation function.
The information being propagated, $m_{u \leftarrow i}$ and $m_{u \leftarrow u}$ is defined in \autoref{eqn:ngcf-userrep}:
\begin{equation}\label{eqn:ngcf-userrep}
  \begin{cases}
    m_{u \leftarrow i}^{(l)} = \frac{1}{\sqrt{|N_u||N_i|}} (W_1^{(l)} e_i^{(l-1)} + W_2^{(l)} (e_i^{(l-1)} \odot e_u^{(l-1)}))\\
    m_{u \leftarrow u}^{(l)} = W_1^{(l)}e_u^{(l-1)}
  \end{cases}
\end{equation}
where $W_1^{(l)}$ and $W_2^{(l)}$ once again are trainable matrices, and $e_i^{(l-1)}$ is the item representation generated at the previous layer.
After $L$ layers of propagation there are multiple representations for user $u$, ${e_u^{(1)},...,e_u^{(L)}}$. These representations are concatenated for the final representation, meaning each corresponding entry in the different embeddings are combined for the final embedding, and the same operation is performed on the item representations.
Finally, the inner product is calculated between the user and given item representation, as shown in \autoref{eqn:ngcf-predict}:
\begin{equation}\label{eqn:ngcf-predict}
  \hat{y}(u, i) = e_U^{*^\textrm{T}} e_i^*
\end{equation}
where $*$ is used to denote the final concatenated representation of the user and the item.
Learning model parameters with \textit{NGCF} is done through optimizing the pairwise Bayesian Personalized Ranking (BPR) loss.
BPR assumes that observed interactions should be assigned higher prediction values than unobserved ones since these are more reflective of preferences.
The objective function is defined in \autoref{eqn:ngcf-loss}.
\begin{equation}\label{eqn:ngcf-loss}
    Loss = \sum_{(u, i, j) \in O} - \textrm{ln} \: \sigma (\hat{y}_{ui} - \hat{y}_{uj}) + \lambda ||\Theta||_2^2
\end{equation}
where $O = {(u, i, j) | (u, i) \in R^+, (u, j) \in R^-}$, where $i$ and $j$ are items from respectively observed interactions $R^+$ and unobserved interactions $R^-$, $\sigma$ is the sigmoid function, $\Theta$ denotes all trainable model parameters and $\lambda$ is a regularization parameter.
\textit{NGCF} employs dropout to avoid overfitting in neural networks.
\textit{NGCF} randomly drops information being propagated through \autoref{eqn:ngcf-userrep} by a probability, as well as dropping nodes in training by randomly blocking and discarding all its outgoing information.

\subsubsection{LightGCN}
In \cite{LightGCN}, Xiangnan He et. al investigated the \textit{NCGF} method by exploring the effects that the nonlinear activation function and feature transformation had on the model's results.
They found that by removing these parts of the \textit{NGCF} model they were able to achieve better results, meaning that these parts negatively affected the training of the model.
Inspired by this, Xiangnan He et. al came up with a light but effective model of a graph convolutional network (GCN) with only the most essential parts used for collaborative filtering called \textit{LightGCN}.
For the experiments, we use the official TensorFlow based implementation of the paper available on \href{https://github.com/kuandeng/LightGCN}{GitHub}.
Like \textit{NGCF}, this method produces implicit ratings in the form of a top-$N$ list.
\\\\
The idea of a GCN is to learn the representation of nodes by smoothing features over the graph \cite{LightGCN}.
This is done by performing graph convolution where features of neighboring nodes are aggregated to give a new representation of the target node.
They propose an aggregation function called Light Graph Convolution (LGC) seen in \cref{eqn:lgc}.
\begin{align}\label{eqn:lgc}
    e_{u}^{(k+1)}=\sum_{i\in N_u}\frac{1}{\sqrt{\left | N_u \right |}\sqrt{\left | N_i \right |}}e_{i}^{(k)},\nonumber\\
    e_{i}^{(k+1)}=\sum_{u\in N_i}\frac{1}{\sqrt{\left | N_i \right |}\sqrt{\left | N_u \right |}}e_{u}^{(k)}
\end{align}
$e_{u}^{(k+1)}$ and $e_{i}^{(k+1)}$ are the embeddings at layer $k+1$ for a user $u$ and item $i$ respectively.
$N_u$ are the items that user $u$ has interacted with and $N_i$ are the users that have interacted with item $i$.
Note that this aggregation does not contain any nonlinear activation function nor any feature transformation, unlike NGCF's aggregation function described in \cref{eqn:ngcf-agg}.
$\frac{1}{\sqrt{\left | N_u \right |}\sqrt{\left | N_i \right |}}$ is a symmetric normalization term which makes sure that the embeddings do not scale with increasing graph convolution operations.
In LGC the target node is not included in the aggregation and only the neighboring nodes are used which means that it does not handle self-connections.
This is instead handled in a later operation where the layers are combined.
\\
The only trainable model parameters in \textit{LightGCN} are the embeddings at the 0-th layer which are $e_{u}^{(0)}$ for the users and $e_{i}^{(0)}$ for the items.
Embeddings at higher layers are computed using the LGC aggregation function. 
When the embeddings at the final layer have been computed, layers are combined through a summation to get a final presentation of the users and items.
\begin{align}\label{eqn:lgcn-layer-comb}
    e_{u} = \sum_{k=0}^K \alpha_k e_{u}^{(k)} \nonumber\\
    e_{i} = \sum_{k=0}^K \alpha_k e_{i}^{(k)}
\end{align}
In \cref{eqn:lgcn-layer-comb} we have $\alpha_k \geq 0$ which is a weight that indicates the importance of the embedding on layer $k$.
They found that setting this parameter to $1/(K+1)$ uniformly led to a good performance, but the parameter can be learned as a model parameter.
There are a few reasons why the layer combination is a good thing to do.
First of all, the embeddings become over-smoothed with an increasing number of layers, meaning that the embeddings slowly become indistinguishable when too much graph convolution is done.
This may be avoided by combining the embeddings at the different layers in the final step of the model.
Secondly, the embeddings at the different layers capture different features that would not be expressed if only the last layer's embeddings were used for the final embeddings.
For example, aggregation at the first layer captures the direct interactions between users and items, the second layer captures users that have an overlap in their item interactions, and so on.
Finally, when the embeddings at different layers are combined with a weighted sum, the effect of graph convolution with self-connections is captured.
\\
After the layer combination, the final embeddings can be used to make predictions with the model as seen in \cref{eqn:lgcn-pred}.
\begin{align}\label{eqn:lgcn-pred}
    \hat{y}_{ui} = e_U^{*^\textrm{T}} e_i^*
\end{align}

\subsubsection{KGAT}
Knowledge Graph Attention Network for Recommendation (KGAT) was proposed by Xiang Wang et. al in \cite{KGAT}.
The main contribution for \textit{KGAT} is the ability to model side information, such as item attributes and context while taking high-order connectivity into account by utilizing a collaborative knowledge graph (CKG).
This collaborative knowledge graph is presented as an extension to the bipartite graph with the two distinct sets $U$ and $I$ indicating respectively users and items.
In addition to the bipartite graph, we have a knowledge graph containing side information for items, where an interaction between users $U$ and items $I$ is described as a tuple $(u, r, i)$ indicating that a user $u \in U$ has a relation $r$ to an item $i \in I$.
These two are combined as a CKG where we represent a user behavior triplet as $(u, Interact, i)$ to indicate an interaction between $u$ and $i$, by unifying the two graphs as $G = \{(u, r, i) | u, i \in E, r \in R\}$, where $E$ is the set of entities describing items and side-information and $R$ is the set of possible relations between entities.\\\\
These relations can, for example, be \textit{actor, director} or \textit {genre}, such that a tuple \textit{(Bruce Willis, Actor, Die Hard}) can be constructed, which indicates that Bruce Willis was an actor in the entity Die Hard.
Finally, \textit{KGAT} utilizes high-order connectivity to increase recommendation quality.
The \textit{L-order connectivity} between nodes is defined as a multi-hop relational path between entities: $e_0 \overset{r_1}{\rightarrow} e_1 \overset{r_2}{\rightarrow} \dots \overset{r_L}{\rightarrow} e_L$ where $e_l \in E$, $r_l \in R$ and $l \in L$.
In a real world scenario, this could look like: $u_1 \overset{r_1}{\rightarrow} i_1 \overset{-r_1}{\rightarrow} u_2 \overset{r_1}{\rightarrow} i_2$ suggesting that $u_1$ would be interested in $i_2$ since their similar user $u_2$ has expressed interest in $i_2$.
The relations $r$ are arbitrary directed relations, and a negative relation $-r$ indicates going against the direction.
Additionally, \textit{KGAT} allows us to model relations based on entities across different relation types, for example if $u_1$ likes movies with Bruce Willis as an actor, they may also like movies produced by him.
\\
The methodology for doing this consists of three main components: An embedding layer, an attentive embedding propagation layer, and a prediction layer.\\
For an overview, the embedding layer handles mapping entities and their relations into a lower dimension space.
The attention embedding propagation layer is responsible for generating weights between entities.
The prediction layer is responsible for generating an ordered list of recommendations for the users by aggregating the information from the previous layers.

\subsubsection{DeepWalk}
\textit{DeepWalk} is an algorithm that uses a deep learning technique to learn social representations of a graph's nodes\cite{DeepWalk}.
It makes use of neural language models where the language consists of "sentences" generated by doing random walks between the nodes of the graph.
By doing this, latent features that capture the neighborhood similarity of the nodes can be learned.
This results in a low dimensional vector representation of each node in the graph.
The random walks are done a number of times for each node $v_i$.
The nodes in the walk are chosen at random to form the random walks $W_{vi}$ which are used as "sentences" in the neural language model and the nodes are the "words" of the sentences.
The language model is then used to estimate the likelihood that nodes appear in the same random walk.
This is done using the language model SkipGram, and the result of this is a vector representation of each node in $d$-dimensional space.
\\\\
The next step is to then use the vector representations to generate recommendations.
Here we opted to go for a k-nearest neighbor (kNN) approach where we find the 10 nearest user nodes to a user node $u$ using the vector representations.
To then generate a rating prediction for a specific item $i$, we look at whether the 10 neighbors have rated item $i$ and use the average of their ratings of $i$ as the predicted rating for user $u$ on item $i$.

\subsection{Description of datasets}\label{subsec:desc-of-datasets}
Context-aware recommender systems consider various types of contextual information such as time, location, and social information when generating recommendations.
They have generally been observed to greatly improve the effectiveness of recommendation processes \cite{aggarwal2016recommender}.
To establish the usefulness of adding context to recommender systems we will conduct an investigation into recent papers relating to the topic examining the experimental results of different proposed methods.
We will investigate the kinds of context data used in existing papers, how this context data was used, and how it was evaluated.
\autoref{tab:paperdatasets} shows an overview of different papers relating to the topic of context-aware recommendations and the datasets used for evaluation.
In the following subsections we will discuss the specific datasets and how they were used.

\subsubsection{LDOS-CoMoDa}
The LDOS-CoMoDa dataset is a context rich movie recommender dataset \cite{comoda}.
At the time of access, the dataset contains 121 users, 1232 unique movies, and 2296 ratings.
Most context variables are expressed for each rating.
The dataset contains the following context variables and their conditions:
\begin{itemize}
    \item Time: Morning, Afternoon, Evening, Night
    \item Daytype: Working day, Weekend, Holiday
    \item Season: Spring, Summer, Autumn, Winter
    \item Location:  Home, Public place, Friend's house
    \item Weather: Sunny / clear, Rainy, Stormy, Snowy, Cloudy
    \item Social: Alone, My partner, Friends, Colleagues, Parents, Public, My family
    \item EndEmo: Sad, Happy, Scared, Surprised, Angry, Disgusted, Neutral
    \item DominantEmo: Sad, Happy, Scared, Surprised, Angry, Disgusted, Neutral
    \item Mood: Positive, Neutral, Negative
    \item Physical: Healthy, Ill
    \item Decision: User decided which movie to watch, User was given a movie
    \item Interaction: First interaction with a movie, N-th interaction with a movie
\end{itemize}
\textit{EndEmo} and \textit{DominantEmo} relate to the emotional state of the user during the consumption stage.
\textit{DominantEmo} defines the emotional state that was dominant during the consumption of the movie, whereas \textit{EndEmo} defines the emotional state of the user at the end of the movie \cite{COMODA2013}.
\cite{COMODA2013} indicates that, on other datasets where the only context that could be derived were based on timestamps, many users would leave these ratings in a relatively short period of time, making them not representative of the contextual situation of the user at the time of consumption.
The paper thus proposes the LDOS-CoMoDa dataset containing potential contextual information from the consumption stage, gathered through ratings and an accompanying questionnaire.
\\\\
\cite{COMODA2013} employed a relevant-context-detection procedure to determine which of these contextual variables were in fact relevant.
This was done through statistical hypothesis testing with a power analysis, where independence was tested between each contextual variable and the ratings.
The null hypothesis of the test stated that the two variables were independent, whereas the alternative hypothesis stated that they were dependent.
If the null hypothesis was rejected, a conclusion was drawn that the contextual variable and the rating were dependent and thus the contextual information was relevant.
They employed a significance level of $\alpha = 0.05$ for the test.
\\\\
The testing determined that six of the variables proved to be relevant, making them contextual - \textit{EndEmo, DominantEmo, Mood, Physical, Decision} and \textit{Interaction}.
\textit{Location} and \textit{Daytype} could not be declared irrelevant, and \textit{Time, Season, Weather} and \textit{Social} were rejected as irrelevant contextual information.
Generally, the paper finds that the variables detected as relevant perform better than the irrelevant ones, apart from the \textit{Mood} variable, which performed worse than a variable deemed irrelevant.
This means that, with the exception of the variable \textit{Mood}, the paper finds that the contextual variables detected as relevant tend to perform better than the uncontextualized models, while the contextual variables detected as irrelevant tend to perform worse than the uncontextualized models, if there are enough ratings per each context variable value during training.
The anomaly regarding \textit{Mood} can be explained by an insolated case of high sparsity in the negative condition for the dataset.

\subsubsection{MovieLens}
The MovieLens datasets are datasets provided by GroupLens research from the MovieLens web site.
These datasets were collected over various periods of time, and are available in different sizes \cite{movielens}.
The papers that were investigated made use of both the 1M dataset and the 100K stable benchmark datasets.
The 1M dataset contains 1,000,209 ratings of 3706 movies made by 6040 users with a density of 4.47\%, representing the percentage of cells in the full user-item matrix that contain rating values \cite{MovieLens2015}.
The 100K Dataset contains 100,000 ratings of 1682 movies made by 943 users with a density of 6.30\% \cite{MovieLens2015}.
Each user in both datasets has rated at least 20 movies.
The datasets do not contain specific contextual information, but it is possibly to derive a time context dimension from the timestamps provided along the ratings.

\subsubsection{Frappé}
Frappé is a mobile app recommender providing context-aware mobile app recommendations by means of a tensor factorization approach based on implicit feedback data\cite{baltrunas2015frappe}.
Frappé was deployed on Android, leading to a context-aware app usage data set.
Frappé collected implicit data on the following relevant context dimensions: \textit{time of day, weekday, whether or not it is weekend, at home or at work, weather, country, cost} and \textit{city}. 
The dataset consists of 96,203 entries by 957 users for 4082 apps.
Implicit data in this context means that the data is gathered without the user explicitly giving a rating, unlike the other datasets where a user has explicitly given a rating from 1-5 before an interaction is registered.

\subsubsection{InCarMusic}
InCarMusic is a mobile Android application offering music recommendations for the passengers of cars.
In order to provide these recommendations, \cite{InCarMusic2011} collected the user's assessment of the effect of context on their music preferences, as well as had them enter ratings for tracks assuming certain contextual conditions held.
\cite{InCarMusic2011} identified the following contextual variables as potentially relevant: driving style, road type, landscape, sleepiness, traffic conditions, mood, weather, natural phenomena.
The data collection was carried out in two phases: one with an aim of determining the contextual factors that are more influential in changing the propensity of the user to listen to music of different genres, and another interested in individual tracks and their ratings, examining the case without considering any contextual conditions, and the case under the assumption that a certain contextual condition holds.
Ultimately, this resulted in a dataset consisting of 4012 ratings, given by 42 different users on 139 songs.
An issue with this dataset is that each entry only has data on one contextual dimension, and the rest are unknown.

\subsubsection{DePaul}
The \textit{DePaulMovie} dataset is another alternative.
This dataset has 5,029 ratings from 1-5, given by 97 users on 79 movies within three different context dimensions: \textit{time, location} and \textit{companion}\cite{DePaulData}.
The contextual dimensions distinguish between weekday or weekend, whether or not the movie was watched at home or at the cinema, and finally if it was watched alone, with family or with partner.

\subsubsection{Yahoo! Movies}
The \textit{Yahoo!} dataset contains a dataset of $7,642$ users, $11,915$ movies and $211,231$ ratings.
All users have rated at least one item, and all items are rated by at least one user.
In terms of contextual information, the dataset provides \textit{year of birth} for the user, \textit{gender} and \textit{age group} which is discretized as either child, adult or elder.
The movies are also linked to $32$ types of content information, such as \textit{synopsis}, \textit{running time}, \textit{release date}, \textit{list of genres} and \textit{list of crew members}.
This information can be used for methods that utilize knowledge, such as knowledge-graph-based methods.

\subsection{Evaluation protocols}\label{sec:evaluationmetrics}
Ricci et al. \cite{RecommenderHandbook2015} defines a set of properties for recommender systems for evaluation: \textit{user preference, prediction accuracy, coverage, confidence, trust, novelty, serendipity, diversity, utility, risk, robustness, privacy, adaptivity} and \textit{scalability}.
Each property is suited to certain types of tests, and different metrics are used for evaluating these properties.
Some, such as user preference and trust are more suitable for testing through user studies.
Properties relating to algorithmic effectiveness such as prediction accuracy, coverage and confidence are more suitable for offline experiments, while properties that relate to active use of the recommender system, such as serendipity, are suitable for online studies where real users interact with the system.
\\\\
\autoref{tab:metricsused} shows the metrics used for evaluation for the relevant methods, while \autoref{tab:methodsused} examines each paper for the evaluation protocol.
It is evident that most of these papers focus on offline evaluation, investigating metrics related to algorithmic precision and efficiency, such as root-mean-square error(RMSE), mean absolute error(MAE) and F1-measure.
These metrics are useful for two important problems associated with recommender systems: rating prediction and top-N recommendation\cite{RecommenderHandbook2015}.
\\\\
The rating prediction problem concerns itself with predicting the rating that a user $u$ will give an unrated item $i$ denoted as $r_{ui}$, which is often defined as learning a function $f : U \times I \rightarrow S$, that predicts the rating $f(u, i)$ of a user $u$ for a new item $i$, where $U$ the set of users, $I$ is the set of items, and $S$ is the set of possible values for a rating.
Typically, ratings in a set of ratings $R$ are divided into a training set $R_{train}$ used to learn $f$, and a test set $R_{test}$ used for evaluation prediction accuracy, which is usually done through MAE and RMSE.
These metrics compare the predicted ratings to the observed values in the test set, meaning they depend on the magnitude of the errors made.
\autoref{tab:rmsevsmae} illustrates two potential test sets along with fictive predicted errors for the items they contain.
The main difference between the two metrics is that RMSE penalizes larger errors more harshly compared to MAE.
RMSE would prefer test set 1 in \autoref{tab:rmsevsmae}, even though it has an aggregated error of 6 compared to test set 2 with an error of 4, due to the larger error of 4 on item 3.
The RMSE for the first test set would be 3.46 versus 4 on the second set.
MAE would prefer test set 2 with a score of 1 versus 1.5 on the first.
\begin{table}[]
    \begin{tabular}{|l|l|l|l|l|}
    \hline
               & Item 1 & Item 2 & Item 3 & Item 4 \\ \hline
    Test set 1 & 2      & 0      & 2      & 2      \\ \hline
    Test set 2 & 0      & 0      & 4      & 0      \\ \hline
    \end{tabular}
    \caption{Example of two test sets and the predicted errors on the items}
    \label{tab:rmsevsmae}
\end{table}
MAE is calculated according to \autoref{eqn:MAE}.
\begin{equation}
    \label{eqn:MAE}
    MAE(f) = \frac{1}{|R_{test}|} \sum\limits_{r_{ui} \epsilon R_{test}} |f(u,i)-r_{ui}|
\end{equation}
RMSE is calculated according to \autoref{eqn:RMSE}.
\begin{equation}
    \label{eqn:RMSE}
    RMSE(f) = \sqrt{\frac{1}{R_{test}} \sum\limits_{r_{ui} \epsilon R_{test}} (f(u, i) - r_{ui})^2}
\end{equation}
The top-N recommendation problem is the task of recommending a list $L(u_a)$ to an active user $u_a$ containing $N$ items to likely be of interest.
Evaluating the quality of such method can be done by splitting $I$ into a training set $I_{train}$ used to learn $L$ and a test set $I_{test}$.
Let $T(u) \subset I_u \cap I_{test}$, where $I_u$ is the subset of items that have been rated by user $u$, then precision can be calculated according to \autoref{eqn:precision}, and recall according to \autoref{eqn:recall}.
Precision defines the proportion of relevant instances among the predicted relevant items, while recall is the proportion of true positives that were correctly predicted.
A trade off between these measures is expected.
Allowing longer recommendation lists improves recall and is likely to reduce precision \cite{RecommenderHandbook2015}.
\begin{equation}
    \label{eqn:precision}
    Precision(L) = \frac{1}{|U|} \sum\limits_{u \epsilon U} |L(u) \cap T(u)| / |L(u)|
\end{equation}
\begin{equation}
    \label{eqn:recall}
    Recall(L) = \frac{1}{|U|} \sum\limits_{u \epsilon U} |L(u) \cap T(u)| / |T(u)|
\end{equation}
The F1-measure also employed by some of the papers summarizes precision and recall into a single number, the harmonic mean of precision and recall, defined by \autoref{eqn:f1}.
This is useful for the purposes of comparison rather than employing either recall or precision by itself, taking the trade off into account.
\begin{equation}
    \label{eqn:f1}
    F1-measure = \frac{2*precision*recall}{precision+recall}
\end{equation}
Other metrics are derived from these basic information retrieval metrics, with \textit{Mean Average Precision} (MAP) representing a popular metric \cite{ChoosingMetricsEvaluation}.
To calculate MAP, the first step is to calculate the \textit{Average Precision} (AP).
This measure takes each relevant item and calculates precision in relation to the position of the item in the recommendation set.
This is defined in \autoref{eq:averageprecision}, where $P(r)$ is the precision at the $r^{th}$ item and $rel(r)$ is an indicator that the item was relevant, taking the value of either 1 if relevant, or 0 if not.
The denominator used for the calculation is normalized to be the smaller of either the length of the list, the $N$ value, or the number of relevant items in case the user has less than $N$ observed ratings.
A point to note about AP is that it rewards recommendation lists with relevant items appearing at the front of the list, leading to a higher AP value in comparison to lists where relevant items appear nearer the N value. 
\begin{equation}
    \label{eq:averageprecision}
    AP = \frac{\sum\limits_{r=1}^N (P(r) \times rel(r))}{minimum(number \: of \: relevant \: documents, N)}
\end{equation}
Once the AP is calculated for a single user $u$, MAP@N for a recommender system is defined through \autoref{eq:map}, where $U$ is the set of all users: 
\begin{equation}
    \label{eq:map}
    MAP = \frac{\sum\limits_{u=1}^U AP_u}{U}
\end{equation}
Finally, some of the papers use the Discounted Cumulative Gain(DCG) metric or the normalized version NDCG.
Assuming each user $u$ has a gain $g_{ui}$ from being recommended item $i$, then the average DCG for a list of $J$ items is defined in \autoref{eqn:dcg}, where $i_j$ is the item at position $j$ in the list, and the logarithm base is a free parameter, where $2$ is most commonly used to ensure all positions are discounted.
This metric also rewards lists that frontload relevant items.
\begin{equation}
    \label{eqn:dcg}
    DCG = \frac{1}{N} \sum\limits_{u=1}^N \sum\limits_{j = 1}^J \frac{g_{ui_j}}{log_b (j+1)}
\end{equation}
NDCG is defined in \autoref{eqn:ndcg}, where $DCG*$ is the ideal DCG, which is defined by sorting the DCG vector such that the most relevant items appear in the start of the list\cite{dcgpaper}, resulting in the biggest possible DCG value.
\begin{equation}
    \label{eqn:ndcg}
    NDCG = \frac{DCG}{DCG*}
\end{equation}
\\\\
The papers examined throughout this research employ cross validation across a number of folds, in a range of 5-10 folds.
Most of them conform to splitting the data into an 80\%/20\% split of training and test data, as seen on \autoref{tab:methodsused}.
Another interesting aspect to look into for the various papers that have been researched is which metrics they use for comparisons.
The results of this can be seen on \autoref{tab:metricsused}.

\begin{table*}[]
    \centering
    \begin{tabular}{|l|l|l|l|l|l|l|l|l|l|}
    \hline
             & \textbf{MSE} & \textbf{RMSE} & \textbf{MAE} & \textbf{Precision} & \textbf{Recall} & \textbf{NDCG} & \textbf{F1} & \textbf{Hit rate} & \textbf{MAP} \\ \hline
KGAT \cite{KGAT}        &              &               &              &                    &                 & x             &             &                   &              \\ \hline
IS-UserBased \cite{GraphBasedCollaborativePaper} &              &               &              &                    &                 & x             &             &                   & x            \\ \hline
NeuMF \cite{neuMF}       &              &               &              &                    &                 & x             &             & x                 &              \\ \hline
LightGCN \cite{LightGCN}    &              &               &              &                    & x               & x             &             &                   &              \\ \hline
NGCF \cite{NGCF}         &              &               &              &                    & x               & x             &             &                   &              \\ \hline
DeepWalk \cite{DeepWalk}          &              &               &              & Implicit                & Implicit             &               & x           &                   &              \\ \hline
NMF \cite{NMF}          &              & x             &              &                    &                 &               &             &                   &              \\ \hline
CAMF(I) \cite{baltrunasCAMF}         &              &               & x            &                    &                 &               &             &                   &              \\ \hline
SVD \cite{standardMF}          & x            & x             &              &                    &                 &               &             &                   &              \\ \hline
SVD++ \cite{svd++}       &              & x             &              &                    &                 &               &             &                   &              \\ \hline
    \end{tabular}
    \caption{Metrics used}
    \label{tab:metricsused}
\end{table*}

\begin{table*}[]
    \begin{tabular}{|l|l|}
    \hline
    \textbf{}     & \textbf{Method used}                          \\ \hline
    \textbf{KGAT} & Training, validation, test (70\%, 10\%, 20\%) \\ \hline
    IS-UserBased  & 10-fold cross validation                      \\ \hline
    NeuMF         & Leave-one-out                                 \\ \hline
    LightGCN      & From original authors (including splits)      \\ \hline
    NGCF          & Training, validation, test (70\%, 10\%, 20\%) \\ \hline
    DeepWalk      & 10-fold cross validation                      \\ \hline
    NMF           & 80/20 split and 5-fold cross validation       \\ \hline
    CAMF          & 25 splits of training, test (90\%, 10\%)      \\ \hline
    SVD           & Netflix prize testing                         \\ \hline
    SVD++         & Test, validation (both 1.4 million ratings)   \\ \hline
    \end{tabular}
    \caption{Methods used}
    \label{tab:methodsused}
\end{table*}

\subsection{Summary}\label{sec:datasetsummary}
A summary of the details of the datasets can be seen in \autoref{tab:datasetstats}.
The most interesting datasets for context-aware recommendation are LDOS-CoMoDa due to the amount of contextual values, the Yahoo! dataset due to the size and inclusion of context, the MovieLens datasets due to their size and ability to extract a temporal context, as well as Frappé due to its size.
Most of the examined papers perform offline evaluation, mainly employing recall and NDCG to evaluate the algorithms, and cross-validate on the sets when testing.

\begin{table*}[]
    \centering
    \begin{tabular}{|c|c|c|c|c|} 
    \hline
               & \#Ratings & \#Items & \#Users & \#Context variables  \\ 
    \hline
    LDOS-CoMoDa    & 2,296      & 1,232    & 121     & 12                   \\ 
    \hline
    MovieLens 1M   & 1,000,209 & 3,706    & 6,040    & 1                    \\ 
    \hline
    MovieLens 100K & 100,000   & 1,682    & 943     & 1                    \\ 
    \hline
    Frappé         & 96,203     & 957     & 4,082    & 8                    \\ 
    \hline
    InCarMusic     & 4,012      & 139     & 42      & 8                    \\ 
    \hline
    DePaulMovie    & 5,029      & 79      & 97      & 3                    \\
    \hline
    Yahoo!    & 211,231      & 11,915      & 7,642      & 2                    \\
    \hline
    \end{tabular}
    \caption{A final summary of the datasets.}
    \label{tab:datasetstats}
\end{table*}

\subsection{Experiments}\label{subsec:experimentprotocol}
Experiments are carried out in order to gain an understanding of baselines that employ different methods and investigate the impact of contextual data.
This section describes these experiments and the protocol used.

\subsubsection{Datasets}
Four datasets are considered for experimentation based on the investigation in \autoref{subsec:desc-of-datasets} and \autoref{sec:evaluationmetrics}: MovieLens 100k, LDOS-CoMoDa, Frappé, as well as the Yahoo! movie dataset of ratings and descriptive content information\cite{Yahoo!-movie}.
These datasets all provide contextual information, side-information, or allow for context to be inferred through the timestamps provided.
Users that have not provided at least 5 ratings are pruned for the LDOS-CoMoDa dataset.
The Frappé dataset does not provide explicit ratings, and methods that require this do not report results on this dataset.
The remaining datasets provide ratings in the range of 1 to 5.
The contextual variables and side information used is defined in \autoref{tbl:sideinfoprotocol} and \autoref{tbl:contextprotocol}.
For the time interval defined based on the timestamp there are five possible values as defined in \autoref{tbl:contextprotocol}.
For the experiments, we define these as $06.00$-$09.00$ being the morning, $09.00$-$12.00$ being noon, $12.00$-$17.00$ being afternoon, $17.00$-$22.00$ being evening, and the remaining interval $22.00$-$05.00$ being night.
KGAT uses all the side-information in \autoref{tbl:sideinfoprotocol} except for age group and gender, while for the Yahoo! dataset the age group and gender side-information is used as a substitute for contextual variables. 
\\
\begin{table*}[]\centering
    \caption{The side-information used for the experiments along with their amount of possible values.}\label{tbl:sideinfoprotocol}
    \scriptsize
    \begin{tabular}{cccc}\toprule
         & \textbf{Yahoo!} & \textbf{MovieLens} & \textbf{Frappé}\\\cmidrule{1-4}
         \multirow{8}{*}[-10pt]{Side-information} & Age group (3 values) & Age group (3 values) & Categories (32 values)\\\cmidrule{2-4}
         & Gender (2 values) & Gender (2 values) & Downloads (16 values) \\\cmidrule{2-4}
         & MPAA-rating (21 values)  &  Genre (19 values) & Developer (2809 values) \\\cmidrule{2-4}
         & Distributor (372 values)  & Release date (8 values) & Language (29 values) \\\cmidrule{2-4}
         & Genre (315 values)  &  &  \\\cmidrule{2-4}
         & DirectorID (5596 values)  &  &  \\\cmidrule{2-4}
         & ActorID (10344 values)  &  &  \\\cmidrule{2-4}
    \bottomrule
    \end{tabular}
\end{table*}
\begin{table*}[]\centering
    \caption{The contextual variables used for the experiments along with their amount of possible values.}\label{tbl:contextprotocol}
    \scriptsize
    \begin{tabular}{ccc}\toprule
         & \textbf{MovieLens} & \textbf{LDOS-CoMoDa}\\\cmidrule{1-3}
         \multirow{2}{*}[-3pt]{Contextual variables} & Day of week (7 values) & DominantEmo (5 values)\\\cmidrule{2-3}
         & Time intervals (5 values) & Physical (2 values) \\\cmidrule{2-3}
    \bottomrule
    \end{tabular}
\end{table*}

\noindent
\textbf{Cross-validation}\\
For each dataset we employ \textit{5-fold} cross-validation.
We split the data into 5 folds, and employ 4 of these as training and 1 as test.
This procedure is repeated 5 times.
The same splits are used for evaluation for each method to ensure proper comparability between the methods.

\subsection{Experimental settings}
In the following we define the evaluation metrics used, the baselines that are compared as well as the hyperparameter settings used.
\\\\
\textbf{Evaluation metrics}\\
As defined in \autoref{sec:evaluationmetrics}, the metrics used are Precision@N, Recall@N, MAP@N, RMSE and MAE.
All five metrics are reported for baselines that provide rating predictions, whereas RMSE and MAE are not reported for methods that only compute a list of top-$N$ recommendations.
For Precision@N, Recall@N and MAP@N, the $k$-value is defined as $10$.
For evaluation purposes we define a relevant item for a given user as an item in which a rating with a value of $3$ or more is observed.
\\\\
\textbf{Baselines and state-of-the-art}\\
The experiments are carried out on the following baselines:
\begin{itemize}
    \item \textbf{Random}: Using random predictions, which is done by either assigning a random rating for rating prediction, or by generating a random list for top-$N$ list recommendations.
    \item \textbf{SVD}: A standard matrix factorization approach. This is a latent factor approach, where items and users are transformed to the same latent factor space, explaining items and users on factors automatically inferred.
    \item \textbf{SVD++}: An extension of the matrix factorization latent factor approach, combining it with neighborhood models that analyze similarities between products and users.
    \item \textbf{Non-negative MF}: A variation of a matrix factorization approach for collaborative filtering, in which the matrices representing the latent factor space are subject to the non-negative constraint, i.e. $\geq$ 0.
\end{itemize}

And the following state-of-the-art methods:
\begin{itemize}
    \item \textbf{KGAT}: Considers side-information when providing recommendation by linking items with their attributes as higher-order relations in a knowledge graph. Embeddings from a node's neighbor are propagated, and an attention mechanism is used to discriminate importance of neighbors.
    \item \textbf{IS-UserBased}: A context-based recommendation approach making use of transitive associations in a bipartite graph to model and find nearest neighbor similarity to produce recommendations on a dataset that has been split on items.
    \item \textbf{LightGCN}: A graph-convolutional network approach that learns user and item embeddings by propagating them on a user-item interaction graph and uses the weighted sum of the embeddings learned at all layers as the final embedding.
    \item \textbf{NGCF}: A neural-network based approach that integrates user-item interactions into embeddings through a bipartite graph-structure.
    \item \textbf{DeepWalk + KNN}: This approach learns latent representations of nodes in a network, using information from truncated random walks. The embeddings learned through this approach are used for a $k$-nearest neighbor recommendation approach.
    \item \textbf{CAMF-C}: A context-aware extension of the SVD approach, introducing a single parameter for all contextual conditions, modeling how the rating deviates from the effect of the context.
    \item \textbf{CAMF-CI}: A variation of CAMF-C, introducing one parameter per contextual condition and item pair.
\end{itemize}
\textbf{Method setup and parameter settings}
\\
This part describes the setup used for the different methods and how the parameters were set.
For all methods that include random initialization, a seed is used to allow the methods to be rerun under the same conditions.
\begin{itemize}
    \item \textbf{KGAT}: A batch size of $1/10$ of the users in dataset is used. The learning rate used is 0.0001 for all datasets. An embedding size of 64 is used.
    \item \textbf{IS-UserBased}: For LDOS-CoMoDa the setup is $L=10$, $n=12$. For Yahoo! it is $L=8$, $n=10$. For MovieLens $L=10$, $n=20$.
    \item \textbf{LightGCN}: For all datasets, the setup is a learning rate of 0.001, 3 layers, an embedding size of 64 and a top-$N$ list of size 10 is generated. The batch size used is $1/10$ of the users in dataset.
    \item \textbf{NGCF}: For all datasets the setup is a learning rate of 0.0005, 3 layers, and an embedding size of 64 is used. The batch size used is $1/10$ of the users in dataset.
    \item \textbf{DeepWalk + KNN}: For all datasets, the embedding size used for \textit{DeepWalk} is 20. The number of neighbors used in \textit{kNN} is 10.
    \item \textbf{CAMF-C}: For all datasets, a learning rate of 0.005, a regularization coefficient of 0.2 and a batch size $1/10$ of users are used. The number of features learned in the matrix factorization is set to 10.
    \item \textbf{CAMF-CI}: For all datasets, a learning rate of 0.01, a regularization coefficient of 0.2 and a batch size $1/10$ of users are used. The number of features learned in the matrix factorization is set to 10.
\end{itemize}

\subsection{Results of experiments}\label{subsec:resultsofexperiment}
In this section we look at the results from the experiments.
\subsubsection{LDOS-CoMoDa}
For the experiments conducted on the LDOS-CoMoDa dataset, we should see the methods that handle context do well, as this dataset has many context variables that the rating depends on.
As mentioned, we are looking at two of these context variables, which are the \textit{dominantEmo} and the \textit{physical} attributes as these are some of the most influential, as described in \autoref{subsec:desc-of-datasets}.
The methods that handle context are \texttt{Itemsplitting} and the \texttt{CAMF} methods.
We see that \texttt{Itemsplitting} has the largest Recall@10, but also the smallest Precision@10, resulting in a small F1@10 value, and in general, it does not perform well on LDOS-CoMoDa, which is surprising as it is able to utilize the context.
However, \texttt{Itemsplitting} is able to achieve one of the highest NDCG scores.
Both \texttt{CAMF-C} and \texttt{CAMF-CI} perform well on LDOS-CoMoDA in terms of most metrics, except for MAP@N.
They are able to achieve the lowest prediction errors amongst all methods in terms of RMSE and MAE, which could indicate that they are able to make good use of the context.
Close to \texttt{CAMF} methods we have baselines such as \texttt{SVD} and \texttt{SVD++} which both are competitive in terms of precision@10, Recall@10 and F1@10. 
These also achieve the highest MAP@N values but do not have as low prediction errors as the \texttt{CAMF} methods.
\texttt{NMF} generally perform a little worse in all metrics compared to \texttt{SVD}, except for NDCG.
The \texttt{DeepWalk + kNN} method has high prediction errors and low Precision@10 and MAP@10 values but has the second-highest Recall@10 and one of the highest NDCG value.
Finally, we have the methods that do not handle rating predictions but only do top-$N$ recommendations, so these do not have RMSE and MAE values, since they work with interactions rather than ratings.
We see that \texttt{LightGCN} is not able to perform well on any of the metrics.
We suspect that this is because the LDOS-CoMoDa dataset is as small as it is with only 121 users, 1,232 unique movies, and 2,296 ratings, and it is therefore not able to utilize its strengths.
And all of these numbers are considerably lower on the actual data used, due to pruning.
However, \texttt{NGCF} is able to achieve both the highest Precision@10 as well as the highest NDCG but does have a lower Recall@10 resulting in a low F1@10 score.
\\\\
To summarize, we expected the context-aware methods to perform well on the LDOS-CoMoDA dataset, as the ratings are context-dependent. 
The \texttt{CAMF} methods do perform well as expected, but \texttt{Itemsplitting} is underperforming in most metrics.
Some baselines such as \texttt{SVD} and \texttt{SVD++} are closely competitive with \texttt{CAMF} as the best performing methods.
The results of the experiment could also indicate that \texttt{NGCF} could be interesting to look at in terms of adding context, as it is able to perform well on a context dataset without utilizing the context available. 
\texttt{NGCF} is also a graph-based approach which, as mentioned in \autoref{sec:graph-rec-sys}, is able to alleviate the sparsity problem introduced by adding context.

\subsection{Graphs test}

\begin{figure}[h]
\begin{tikzpicture}
\begin{axis}[
    ybar,
    enlargelimits=0.15,
    enlarge y limits={upper,value=0.3},
    legend style={at={(0.5,1)},
      anchor=north,legend columns=-1},
    ylabel={RMSE for CoMoDa},
    symbolic x coords={ItemSplitting,CAMF,CAMF-CI,DW+kNN,NMF,Random,SVD++,SVD},
    x tick label style={font=\tiny,rotate=75,anchor=east},
    xtick=data,
    nodes near coords,
    every node near coord/.append style={rotate=90, anchor=west},
    nodes near coords align={vertical},
    ymin=0
    ]
\addplot coordinates {(ItemSplitting,1.118) (CAMF,0.669) (CAMF-CI,0.824) (DW+kNN,1.211) (NMF,1.173) (Random,1.763) (SVD++,1.02) (SVD,1.019)};
\end{axis}
\end{tikzpicture}
\end{figure}

\begin{figure}[h]
\begin{tikzpicture}
\begin{axis} [
    title    = RMSE,
    xbar=5pt,
    /pgf/bar width=5pt,
    y axis line style = { opacity = 0 },
    axis x line       = none,
    tickwidth         = 0pt,
    enlarge y limits  = 0.2,
    enlarge x limits  = 0.02,
    ytick=data,
    y=1.3cm,
    nodes near coords,
    legend style={at={(0.5,1)}, anchor=north,legend columns=-1},
    symbolic y coords = {ItemSplitting,CAMF,CAMF-CI,DW+kNN,NMF,Random,SVD++,SVD},
  ]
\addplot coordinates{(1.118,ItemSplitting) (0.669,CAMF) (0.824,CAMF-CI) (1.211,DW+kNN) (1.173,NMF) (1.763,Random) (1.02,SVD++) (1.019,SVD)};
\addplot coordinates{(1.234,ItemSplitting) (1.003,CAMF) (1.03,CAMF-CI) (1.195,DW+kNN) (1.089,NMF) (1.996,Random) (0.999,SVD++) (1.013,SVD)};
\addplot coordinates{(0,ItemSplitting) (0.931,CAMF) (0.978,CAMF-CI) (0,DW+kNN) (0.962,NMF) (1.702,Random) (0.917,SVD++) (0.935,SVD)};
\legend{CoMoDa, Yahoo, MovieLens}
\end{axis}
\end{tikzpicture}
\end{figure}

\begin{figure}[h]
\begin{tikzpicture}
\begin{axis} [
    title    = NDCG,
    xbar=5pt,
    /pgf/bar width=5pt,
    y axis line style = { opacity = 0 },
    axis x line       = none,
    tickwidth         = 0pt,
    enlarge y limits  = 0.2,
    enlarge x limits  = 0.02,
    ytick=data,
    y=1.3cm,
    nodes near coords,
    legend style={at={(0.5,1)}, anchor=north,legend columns=-1},
    symbolic y coords = {ItemSplitting,CAMF,CAMF-CI,DW+kNN,NMF,Random,SVD++,SVD},
  ]
\addplot coordinates{(0.565,ItemSplitting) (0.477,CAMF) (0.469,CAMF-CI) (0.583,DW+kNN) (0.539,NMF) (0.484,Random) (0.478,SVD++) (0.502,SVD)};
\addplot coordinates{(0.502,ItemSplitting) (0.492,CAMF) (0.476,CAMF-CI) (0.509,DW+kNN) (0.473,NMF) (0.451,Random) (0.507,SVD++) (0.502,SVD)};
\addplot coordinates{(0,ItemSplitting) (0.375,CAMF) (0.403,CAMF-CI) (0,DW+kNN) (0.441,NMF) (0.476,Random) (0.540,SVD++) (0.508,SVD)};
\legend{CoMoDa, Yahoo, MovieLens}
\end{axis}
\end{tikzpicture}
\end{figure}

\begin{figure}[h]
\begin{tikzpicture}
\begin{axis} [
    title    = F1@10,
    xbar=5pt,
    /pgf/bar width=5pt,
    y axis line style = { opacity = 0 },
    axis x line       = none,
    tickwidth         = 0pt,
    enlarge y limits  = 0.2,
    enlarge x limits  = 0.02,
    ytick=data,
    y=1.3cm,
    nodes near coords,
    legend style={at={(0.5,1)}, anchor=north,legend columns=-1},
    symbolic y coords = {ItemSplitting,CAMF,CAMF-CI,DW+kNN,NMF,Random,SVD++,SVD},
  ]
\addplot coordinates{(0.274,ItemSplitting) (0.615,CAMF) (0.609,CAMF-CI) (0.512,DW+kNN) (0.550,NMF) (0.364,Random) (0.614,SVD++) (0.617,SVD)};
\addplot coordinates{(0.350,ItemSplitting) (0.549,CAMF) (0.543,CAMF-CI) (0.437,DW+kNN) (0.523,NMF) (0.308,Random) (0.540,SVD++) (0.543,SVD)};
\addplot coordinates{(0,ItemSplitting) (0.683,CAMF) (0.668,CAMF-CI) (0,DW+kNN) (0.650,NMF) (0.469,Random) (0.673,SVD++) (0.673,SVD)};
\legend{CoMoDa, Yahoo, MovieLens}
\end{axis}
\end{tikzpicture}
\end{figure}


\section{Discussion}
Did the results work? This will all be discussed in this chapter.

\section{Conclusion}\label{sec:conclusion}
Throughout this paper, we have looked into various metrics used for evaluating recommender systems. What we found was that there is no consistent set of metrics used for evaluation, but the most commonly used are NDCG and Recall.
We found that utilizing bipartite graphs is a popular and intuitive way to handle information in the system, but in the investigated metrics, they are often outperformed by traditional methods such as SVD.
The cause of this is speculative but may be because methods like SVD make use of explicit rating information, where the investigated graph-based methods use implicit ratings.
\\\\
From the knowledge acquired throughout this paper, we propose that instead of only looking into context as information about the interaction, it could be beneficial to expand the model to include side-information about the items or users in addition to the context.
This could allow the system to identify otherwise unseen trends, such as a user liking a specific movie genre in a certain context, rather than just looking at which specific movies they like.
Additionally, adding information about the users may help connect like-minded users in scenarios where the current user has not yet rated anything in their given context.

\section{Future work}
Future work, proposals for the masters thesis next semester.

%----------------------------------------------------------------------------------------
%	REFERENCE LIST
%----------------------------------------------------------------------------------------

\printbibliography[heading=bibintoc]
\label{bib:mybiblio}

%----------------------------------------------------------------------------------------

%----------------------------------------------------------------------------------------
%	APPENDIX
%----------------------------------------------------------------------------------------
\newpage
\onecolumn
\appendix
\input{appendix/tables.tex}

\end{document}
