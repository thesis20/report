%%%%%%%%%%%%%%%%%%%%%%%%%%%%%%%%%%%%%%%%%
% Journal Article
% LaTeX Template
% Version 1.4 (15/5/16)
%
% This template has been downloaded from:
% http://www.LaTeXTemplates.com
%
% Original author:
% Frits Wenneker (http://www.howtotex.com) with extensive modifications by
% Vel (vel@LaTeXTemplates.com)
%
% License:
% CC BY-NC-SA 3.0 (http://creativecommons.org/licenses/by-nc-sa/3.0/)
%
%%%%%%%%%%%%%%%%%%%%%%%%%%%%%%%%%%%%%%%%%

%----------------------------------------------------------------------------------------
%	PACKAGES AND OTHER DOCUMENT CONFIGURATIONS
%----------------------------------------------------------------------------------------

\documentclass[twoside,twocolumn]{article}

\usepackage{blindtext} % Package to generate dummy text throughout this template 

\usepackage[sc]{mathpazo} % Use the Palatino font
\usepackage[T1]{fontenc} % Use 8-bit encoding that has 256 glyphs
\linespread{1.05} % Line spacing - Palatino needs more space between lines
\usepackage{microtype} % Slightly tweak font spacing for aesthetics

\usepackage[english]{babel} % Language hyphenation and typographical rules

\usepackage[left=24mm, right=24mm,top=32mm,columnsep=20pt]{geometry} % Document margins
\usepackage[hang, small,labelfont=bf,up,textfont=it,up]{caption} % Custom captions under/above floats in tables or figures
\usepackage{booktabs} % Horizontal rules in tables

\usepackage{lettrine} % The lettrine is the first enlarged letter at the beginning of the text

\usepackage{enumitem} % Customized lists
\setlist[itemize]{noitemsep} % Make itemize lists more compact

\usepackage{abstract} % Allows abstract customization
\renewcommand{\abstractnamefont}{\normalfont\bfseries} % Set the "Abstract" text to bold
\renewcommand{\abstracttextfont}{\normalfont\small\itshape} % Set the abstract itself to small italic text

\usepackage{titlesec} % Allows customization of titles
%\renewcommand\thesection{\Roman{section}} % Roman numerals for the sections
%\renewcommand\thesubsection{\roman{subsection}} % roman numerals for subsections
\titleformat{\section}[block]{\large\scshape\centering}{\thesection.}{1em}{} % Change the look of the section titles
%\titleformat{\subsection}[block]{\large}{\thesubsection.}{1em}{} % Change the look of the section titles

\usepackage{fancyhdr} % Headers and footers
\pagestyle{fancy} % All pages have headers and footers
\fancyhead{} % Blank out the default header
\fancyfoot{} % Blank out the default footer
\fancyhead[C]{mi902e20 $\bullet$ spMI-1 2020} % Custom header text
\fancyfoot[RO,LE]{\thepage} % Custom footer text

\usepackage{titling} % Customizing the title section

\usepackage{hyperref} % For hyperlinks in the PDF

% Bibliography
% http://en.wikibooks.org/wiki/LaTeX/Bibliography_Management
%%%%%%%%%%%%%%%%%%%%%%%%%%%%%%%%%%%%%%%%%%%%%%%%
% Add the \citep{key} command which display a
% reference as [author, year]
\usepackage[backend=bibtex,
    bibencoding=utf8
]{biblatex}
\addbibresource{bib/mybib}
\usepackage{csquotes}
\usepackage[nottoc]{tocbibind}
\usepackage{amsmath}
\usepackage{tikz} % Used to create bipartite graph
\usepackage{pgfplots}
\pgfplotsset{width=5.5cm,compat=1.8}
\pgfplotsset{scaled y ticks=false}
\usetikzlibrary{positioning,chains,fit,shapes,calc}
\usepackage{amsmath}


\definecolor{myblue}{RGB}{80,80,160}
\definecolor{mygreen}{RGB}{80,160,80}

\newcommand{\R}{\mathbb{R}}
\usepackage{amsmath}
\AfterEndEnvironment{figure}{\noindent\ignorespaces}
\usepackage{multirow}

\usepackage{pgfplots}
\usepackage{pgfplotstable}

\usepackage{booktabs} % for borders and merged ranges
\usepackage{soul}% for underlines
\usepackage{changepage,threeparttable} % for wide tables
\usepackage{cleveref}

%----------------------------------------------------------------------------------------
%	TITLE SECTION
%----------------------------------------------------------------------------------------

\setlength{\droptitle}{-4\baselineskip} % Move the title up

\pretitle{\begin{center}\Huge\bfseries} % Article title formatting
\posttitle{\end{center}} % Article title closing formatting
\title{Context and Side-Information in Recommender Systems} % Article title
\author{%
\textsc{Andreas Stenshøj, Daniel Moesgaard Andersen, Rasmus Bundgaard Eduardsen} \\[1ex] % Your name
\normalsize Aalborg University \\ % Your institution
\normalsize \href{mailto:astens16@student.aau.dk}{astens16@student.aau.dk} % Your email address
\normalsize \href{mailto:dand16@student.aau.dk}{dand16@student.aau.dk} % Your email address
\normalsize \href{mailto:reduar16@student.aau.dk}{reduar16@student.aau.dk} % Your email address
%\and % Uncomment if 2 authors are required, duplicate these 4 lines if more
%\textsc{Jane Smith}\thanks{Corresponding author} \\[1ex] % Second author's name
%\normalsize Aalborg University \\ % Second author's institution
%\normalsize \href{mailto:jane@smith.com}{jane@smith.com} % Second author's email address
}
\date{\today} % Leave empty to omit a date
\renewcommand{\maketitlehookd}{%
\begin{abstract}
\noindent How do graph-based methods compare to non-graph based methods for the recommender system domain, and how does context and side-information influence the accuracy of predicted items? 
This paper investigates state-of-the-art within context-aware and graph-based recommender systems, datasets available for context-aware methods as well as useful evaluation metrics.
The outcome is a comparison of the identified methods, showing performance on a broad range of evaluation metrics on relatively small datasets.
The results show that the addition of context or side-information to recommendation systems does not necessarily improve performance, and that methods utilizing implicit ratings tend to perform worse on the investigated datasets.
\end{abstract}
}

%----------------------------------------------------------------------------------------

\begin{document}

% Print the title
\maketitle

%----------------------------------------------------------------------------------------
%	ARTICLE CONTENTS
%----------------------------------------------------------------------------------------
\chapter{Recommender systems (RS)}\label{ch:introduction}
are widely used to combat information overload for users when accessing new information by recommending specific items that the user is likely to find compatible\cite{YouTubeNeural,IndustryPerspective}.
RS are conventionally defined through three categories: \textit{content-based}, \textit{collaborative filtering} and \textit{hybrid}.
\textit{Content-based} RS provide recommendations based on content information such as genre, \textit{collaborative filtering} recommend items based on similar users, and \textit{hybrid} combines the two prior approaches\cite{ContextSurvey2020}.
\textit{Context-Aware Recommender Systems} (CARS) exist as a subset of RS, aiming to incorporate contextual information into the process of recommendation.
Contextual information is intuitively useful for making recommendations.
In a movie recommendation situation this context could be company, time, or mood.
For example, a user may not want to watch the same movies in the company of their significant other, as they would with their parents.
Likewise, time could be an important factor, since movies such as \textit{A Christmas Story} may be more relevant during the winter months than during summer.\\\
Generally, for a RS, the objective is to either: estimate a rating function $R: User \times Item \rightarrow Rating$ for the $(user, \ item)$ pairs that are not yet rated, or to compute a list $L$ of the top recommended items for an active user $U_a$ containing items that are likely to be of interest $R: User \times Item \rightarrow L(Ua)$\cite{RecommenderHandbook2015}.
CARS typically extend these functions to respectively $R: User \times Item \times Context \rightarrow Rating$ and $R: User \times Item \times Context \rightarrow L(Ua)$ by taking the known contextual information associated with the application into account.
\\\\
The remainder of the paper is organized as follows. 
\autoref{sec:contextinrecommendersystems} is a brief overview of how context is used in recommender systems.
\autoref{sec:graph-rec-sys} introduces how graphs are used for recommender systems.
\autoref{sec:experiments} describes how we investigated various methods for recommendations on different datasets and how they compare in a series of metrics such as RMSE, precision and NDCG.
\autoref{sec:discussion} is a discussion about the results acquired in the previous section, and how context is used in the various recommender system methods.
We conclude the paper in Sections~\ref{sec:conclusion} and \ref{sec:futurework} by summarizing our research contributions and pointing out future directions.

\section{Context in recommender systems}
Context in my recommender system? It is more likely and effective than you think.
\section{Graph recommender systems}
One of the problems often encountered in recommender systems is the problem of sparsity in larger recommender systems.
Two different kinds of approaches are often used to combat this. 
The first is to use dimensionality reduction where the representation of users and items are converted into a more compact representation, which should capture the most important features.
These representations are able to capture meaningful information about the relations between users and items, even if users have rated different items, or items were rated by different users. 
The other approach used for the sparsity problem is the use of graph representations for the data.
When using graph representations for the data, graph-based methods are able to utilize the transitivity in the relations between the data points. 
This means that similarity can be found between users that are not directly connected, which can help alleviate the sparsity problem.
Compared to the dimensionality reduction approach, graph representations are also able to preserve some of the "local" relations in the data\cite{RecommenderHandbook2015}.
\\\\
A typical graph representation used for the data used in recommender systems is the bipartite graph.
In a bipartite graph, there are two sets of nodes.
One set of nodes is the users and the other set is the items.
The edges in the bipartite graph connect the user and item nodes if the user has rated the item. 
The edge can have the rating as its weight if a rating is given.
An example of how a bipartite graph could look can be seen on \autoref{fig:bipartite-graph}.
\begin{figure}[h]
\begin{tikzpicture}[thick,
    every node/.style={draw,circle},
    fsnode/.style={fill=myblue},
    ssnode/.style={fill=mygreen},
    every fit/.style={ellipse,draw,inner sep=-2pt,text width=2cm},
    shorten >= 3pt,shorten <= 3pt
  ]
  
  % the vertices of U
  \begin{scope}[start chain=going below,node distance=7mm]
  \foreach \i in {u1,u2,u3}
    \node[fsnode,on chain] (f\i) [label=left: \i] {};
  \end{scope}
  
  % the vertices of I
  \begin{scope}[xshift=4cm,start chain=going below,node distance=7mm]
  \foreach \i in {i1,i2,i3,i4}
    \node[ssnode,on chain] (s\i) [label=right: \i] {};
  \end{scope}
  
  % the set U
  \node [myblue,fit=(fu1) (fu3),label=above:$U$] {};
  % the set I
  \node [mygreen,fit=(si1) (si4),label=above:$I$] {};
  
  % the edges
  \draw (fu1) -- (si1);
  \draw (fu1) -- (si4);
  \draw (fu2) -- (si1);
  \draw (fu2) -- (si3);
  \draw (fu3) -- (si3);
  \draw (fu3) -- (si2);
\end{tikzpicture}
\caption{Example of a bipartite graph}
\label{fig:bipartite-graph}
\end{figure}

As mentioned, with the use of graph representations, users and items not directly connected can influence each other.
There are two properties that are often explored for graph-based approaches when doing similarity measures, namely propagation and attenuation.
Propagation is done through propagating information along the edges of the graph, also using the weight of the edge to determine how much information is allowed to pass through.
Attenuation considers that nodes that are further away from each other in the graph should influence each other less.
This transitive association in graph-based approaches can be used in two ways to recommend items.
The relevance between a user \textit{u} and an item \textit{i} can be evaluated using the proximity of \textit{u} to the item \textit{i}, meaning that the items that are "closest" in the graph to \textit{u} will be recommended.
A similarity between a pair of users or items can also be found which can be expressed as a weight between them\cite{RecommenderHandbook2015}.

\section{Experiments}
This section will describe our experiments and findings regarding utilizing context in recommender systems. Will include setup, research questions and various graphs showing results.
\subsection{Description of datasets}\label{subsec:desc-of-datasets}
Context-aware recommender systems consider various types of contextual information such as time, location, and social information when generating recommendations.
They have generally been observed to greatly improve the effectiveness of recommendation processes\cite{aggarwal2016recommender}.
To establish the usefulness of adding context to recommender systems we will conduct an investigation into recent papers relating to the topic examining the experimental results of different proposed methods.
We will investigate the kinds of context data used in existing papers, how this context data was used, and how it was evaluated.
\autoref{tab:paperdatasets} shows an overview of different papers relating to the topic of context-aware recommendations and the datasets used for evaluation.
In the following subsections we will discuss the specific datasets and how they were used.

\subsubsection{LDOS-CoMoDa}
The LDOS-CoMoDa dataset is a context rich movie recommender dataset\cite{comoda}.
At the time of access, the dataset contains 121 users, 1,232 unique movies, and 2,296 ratings.
Most context variables are expressed for each rating.
The dataset contains the following context variables and their conditions:
\begin{itemize}
    \item Time: Morning, Afternoon, Evening, Night
    \item Daytype: Working day, Weekend, Holiday
    \item Season: Spring, Summer, Autumn, Winter
    \item Location:  Home, Public place, Friend's house
    \item Weather: Sunny / clear, Rainy, Stormy, Snowy, Cloudy
    \item Social: Alone, My partner, Friends, Colleagues, Parents, Public, My family
    \item EndEmo: Sad, Happy, Scared, Surprised, Angry, Disgusted, Neutral
    \item DominantEmo: Sad, Happy, Scared, Surprised, Angry, Disgusted, Neutral
    \item Mood: Positive, Neutral, Negative
    \item Physical: Healthy, Ill
    \item Decision: User decided which movie to watch, User was given a movie
    \item Interaction: First interaction with a movie, N-th interaction with a movie
\end{itemize}
\textit{EndEmo} and \textit{DominantEmo} relate to the emotional state of the user during the consumption stage.
\textit{DominantEmo} defines the emotional state that was dominant during the consumption of the movie, whereas \textit{EndEmo} defines the emotional state of the user at the end of the movie\cite{COMODA2013}.
\cite{COMODA2013} indicates that, on other datasets where the only context that could be derived were based on timestamps, many users would leave these ratings in a relatively short period of time, making them not representative of the contextual situation of the user at the time of consumption.
The paper thus proposes the LDOS-CoMoDa dataset containing potential contextual information from the consumption stage, gathered through ratings and an accompanying questionnaire.
\\\\
\cite{COMODA2013} employed a relevant-context-detection procedure to determine which of these contextual variables were in fact relevant.
This was done through statistical hypothesis testing with a power analysis, where independence was tested between each contextual variable and the ratings.
The null hypothesis of the test stated that the two variables were independent, whereas the alternative hypothesis stated that they were dependent.
If the null hypothesis was rejected, a conclusion was drawn that the contextual variable and the rating were dependent and thus the contextual information was relevant.
They employed a significance level of $\alpha = 0.05$ for the test.
\\\\
The testing determined that six of the variables proved to be relevant, making them contextual - \textit{EndEmo, DominantEmo, Mood, Physical, Decision} and \textit{Interaction}.
\textit{Location} and \textit{Daytype} could not be declared irrelevant, and \textit{Time, Season, Weather} and \textit{Social} were rejected as irrelevant contextual information.
Generally, the paper finds that the variables detected as relevant perform better than the irrelevant ones, apart from the \textit{Mood} variable, which performed worse than a variable deemed irrelevant.
This means that, with the exception of the variable \textit{Mood}, the paper finds that the contextual variables detected as relevant tend to perform better than the uncontextualized models, while the contextual variables detected as irrelevant tend to perform worse than the uncontextualized models, if there are enough ratings per each context variable value during training.
The anomaly regarding \textit{Mood} can be explained by an insolated case of high sparsity in the negative condition for the dataset.

\subsubsection{MovieLens}
The MovieLens datasets are datasets provided by GroupLens research from the MovieLens web site.
These datasets were collected over various periods of time, and are available in different sizes\cite{movielens}.
The papers that were investigated made use of both the 1M dataset and the 100K stable benchmark datasets.
The 1M dataset contains 1,000,209 ratings of 3,706 movies made by 6,040 users with a density of 4.47\%, representing the percentage of cells in the full user-item matrix that contain rating values \cite{MovieLens2015}.
The 100K Dataset contains 100,000 ratings of 1,682 movies made by 943 users with a density of 6.30\% \cite{MovieLens2015}.
Each user in both datasets has rated at least 20 movies.
The datasets do not contain specific contextual information, but it is possibly to derive a time context dimension from the timestamps provided along the ratings.

\subsubsection{Frappé}
Frappé is a mobile app recommender providing context-aware mobile app recommendations by means of a tensor factorization approach based on implicit feedback data\cite{baltrunas2015frappe}.
Frappé was deployed on Android, leading to a context-aware app usage data set.
Frappé collected implicit data on the following relevant context dimensions: \textit{time of day, weekday, whether or not it is weekend, at home or at work, weather, country, cost} and \textit{city}. 
The dataset consists of 96,203 entries by 957 users for 4,082 apps.
Implicit data in this context means that the data is gathered without the user explicitly giving a rating, unlike the other datasets where a user has explicitly given a rating from 1-5 before an interaction is registered.

\subsubsection{InCarMusic}
InCarMusic is a mobile Android application offering music recommendations for the passengers of cars.
In order to provide these recommendations, \cite{InCarMusic2011} collected the user's assessment of the effect of context on their music preferences, as well as had them enter ratings for tracks assuming certain contextual conditions held.
\cite{InCarMusic2011} identified the following contextual variables as potentially relevant: driving style, road type, landscape, sleepiness, traffic conditions, mood, weather, natural phenomena.
The data collection was carried out in two phases: one with an aim of determining the contextual factors that are more influential in changing the propensity of the user to listen to music of different genres, and another interested in individual tracks and their ratings, examining the case without considering any contextual conditions, and the case under the assumption that a certain contextual condition holds.
Ultimately, this resulted in a dataset consisting of 4,012 ratings, given by 42 different users on 139 songs.
An issue with this dataset is that each entry only has data on one contextual dimension, and the rest are unknown.

\subsubsection{DePaul}
The \textit{DePaulMovie} dataset is another alternative.
This dataset has 5,029 ratings from 1-5, given by 97 users on 79 movies within three different context dimensions: \textit{time, location} and \textit{companion}\cite{DePaulData}.
The contextual dimensions distinguish between weekday or weekend, whether or not the movie was watched at home or at the cinema, and finally if it was watched alone, with family or with partner.

\subsubsection{Yahoo! Movies}
The \textit{Yahoo!} dataset contains a dataset of $7,642$ users, $11,915$ movies and $211,231$ ratings.
All users have rated at least one item, and all items are rated by at least one user.
In terms of contextual information, the dataset provides \textit{year of birth} for the user, \textit{gender} and \textit{age group} which is discretized as either child, adult or elder.
The movies are also linked to $32$ types of content information, such as \textit{synopsis}, \textit{running time}, \textit{release date}, \textit{list of genres} and \textit{list of crew members}.
This information can be used for methods that utilize knowledge, such as knowledge-graph-based methods.

\subsection{Evaluation protocols}\label{sec:evaluationmetrics}
Ricci et al.\cite{RecommenderHandbook2015} defines a set of properties for recommender systems for evaluation: \textit{user preference, prediction accuracy, coverage, confidence, trust, novelty, serendipity, diversity, utility, risk, robustness, privacy, adaptivity} and \textit{scalability}.
Each property is suited to certain types of tests, and different metrics are used for evaluating these properties.
Some, such as user preference and trust are more suitable for testing through user studies.
Properties relating to algorithmic effectiveness such as prediction accuracy, coverage and confidence are more suitable for offline experiments, while properties that relate to active use of the recommender system, such as serendipity, are suitable for online studies where real users interact with the system.
\\\\
\autoref{tab:metricsused} shows the metrics used for evaluation for the relevant methods, while \autoref{tab:methodsused} examines each paper for the evaluation protocol.
It is evident that most of these papers focus on offline evaluation, investigating metrics related to algorithmic precision and efficiency, such as root-mean-square error (RMSE), mean absolute error (MAE) and F1-measure.
These metrics are useful for two important problems associated with recommender systems: rating prediction and top-N recommendation\cite{RecommenderHandbook2015}.
\\\\
The rating prediction problem concerns itself with predicting the rating that a user $u$ will give an unrated item $i$ denoted as $r_{ui}$, which is often defined as learning a function $f : U \times I \rightarrow S$, that predicts the rating $f(u, i)$ of a user $u$ for a new item $i$, where $U$ the set of users, $I$ is the set of items, and $S$ is the set of possible values for a rating.
Typically, ratings in a set of ratings $R$ are divided into a training set $R_{train}$ used to learn $f$, and a test set $R_{test}$ used for evaluation prediction accuracy, which is usually done through MAE and RMSE.
These metrics compare the predicted ratings to the observed values in the test set, meaning they depend on the magnitude of the errors made.
\autoref{tab:rmsevsmae} illustrates two potential test sets along with fictive predicted errors for the items they contain.
The main difference between the two metrics is that RMSE penalizes larger errors more harshly compared to MAE.
RMSE would prefer test set 1 in \autoref{tab:rmsevsmae}, even though it has an aggregated error of 6 compared to test set 2 with an error of 4, due to the larger error of 4 on item 3.
The RMSE for the first test set would be 3.46 versus 4 on the second set.
MAE would prefer test set 2 with a score of 1 versus 1.5 on the first.
\begin{table}[]
    \begin{tabular}{|l|l|l|l|l|}
    \hline
               & Item 1 & Item 2 & Item 3 & Item 4 \\ \hline
    Test set 1 & 2      & 0      & 2      & 2      \\ \hline
    Test set 2 & 0      & 0      & 4      & 0      \\ \hline
    \end{tabular}
    \caption{Example of two test sets and the predicted errors on the items}
    \label{tab:rmsevsmae}
\end{table}
MAE is calculated according to \autoref{eqn:MAE}.
\begin{equation}
    \label{eqn:MAE}
    MAE(f) = \frac{1}{|R_{test}|} \sum\limits_{r_{ui} \epsilon R_{test}} |f(u,i)-r_{ui}|
\end{equation}
RMSE is calculated according to \autoref{eqn:RMSE}.
\begin{equation}
    \label{eqn:RMSE}
    RMSE(f) = \sqrt{\frac{1}{R_{test}} \sum\limits_{r_{ui} \epsilon R_{test}} (f(u, i) - r_{ui})^2}
\end{equation}
The top-N recommendation problem is the task of recommending a list $L(u_a)$ to an active user $u_a$ containing $N$ items to likely be of interest.
Evaluating the quality of such method can be done by splitting $I$ into a training set $I_{train}$ used to learn $L$ and a test set $I_{test}$.
Let $T(u) \subset I_u \cap I_{test}$, where $I_u$ is the subset of items that have been rated by user $u$, then precision can be calculated according to \autoref{eqn:precision}, and recall according to \autoref{eqn:recall}.
Precision defines the proportion of relevant instances among the predicted relevant items, while recall is the proportion of true positives that were correctly predicted.
A trade off between these measures is expected.
Allowing longer recommendation lists improves recall and is likely to reduce precision\cite{RecommenderHandbook2015}.
\begin{equation}
    \label{eqn:precision}
    Precision(L) = \frac{1}{|U|} \sum\limits_{u \epsilon U}\frac{|L(u) \cap T(u)|}{|L(u)|}
\end{equation}
\begin{equation}
    \label{eqn:recall}
    Recall(L) = \frac{1}{|U|} \sum\limits_{u \epsilon U} \frac{|L(u) \cap T(u)|}{|T(u)|}
\end{equation}
The F1-measure also employed by some of the papers summarizes precision and recall into a single number, the harmonic mean of precision and recall, defined by \autoref{eqn:f1}.
This is useful for the purposes of comparison rather than employing either recall or precision by itself, taking the trade off into account.
\begin{equation}
    \label{eqn:f1}
    F1 = \frac{2*Precision(L)*Recall(L)}{Precision(L)+Recall(L)}
\end{equation}
Other metrics are derived from these basic information retrieval metrics, with \textit{Mean Average Precision} (MAP) representing a popular metric\cite{ChoosingMetricsEvaluation}.
To calculate MAP, the first step is to calculate the \textit{Average Precision} (AP).
This measure takes each relevant item and calculates precision in relation to the position of the item in the recommendation set.
This is defined in \autoref{eq:averageprecision}, where $Precision(n)$ is the precision at the $n^{th}$ item and $Relevant(n)$ is an indicator that the item was relevant, taking the value of either 1 if relevant, or 0 if not.
The denominator used for the calculation is normalized to be the smaller of either the length of the list, the $N$ value, or the number of relevant items in case the user has less than $N$ observed ratings.
A point to note about AP is that it rewards recommendation lists with relevant items appearing at the front of the list, leading to a higher AP value in comparison to lists where relevant items appear nearer the N value. 
\begin{equation}
    \label{eq:averageprecision}
    AP = \frac{\sum\limits_{n=1}^N (Precision(n) \times Relevant(n))}{min(|Relevant(N)|,\:N)}
\end{equation}
Once the AP is calculated for a single user $u_a$, MAP@N for a recommender system is defined through \autoref{eq:map}, where $U$ is the set of all users: 
\begin{equation}
    \label{eq:map}
    MAP = \frac{\sum\limits_{u=1}^|U| AP_u}{|U|}
\end{equation}
Finally, some of the papers use the Discounted Cumulative Gain(DCG) metric or the normalized version NDCG.
Assuming each user $u$ has a gain $g_{ui}$ from being recommended item $i$, then the average DCG for a list of $J$ items is defined in \autoref{eqn:dcg}, where $i_j$ is the item at position $j$ in the list, and the logarithm base is a free parameter, where $2$ is most commonly used to ensure all positions are discounted.
This metric also rewards lists that frontload relevant items.
\begin{equation}
    \label{eqn:dcg}
    DCG = \frac{1}{|U|} \sum\limits_{u=1}^|U| \sum\limits_{i = 1}^|L| \frac{g_{ui_l}}{log_b (l+1)}
\end{equation}
NDCG is defined in \autoref{eqn:ndcg}, where $DCG*$ is the ideal DCG, which is defined by sorting the DCG vector such that the most relevant items appear in the start of the list\cite{dcgpaper}, resulting in the biggest possible DCG value.
\begin{equation}
    \label{eqn:ndcg}
    NDCG = \frac{DCG}{DCG*}
\end{equation}
\\\\
The papers examined throughout this research employ cross validation across a number of folds, in a range of 5-10 folds.
Most of them conform to splitting the data into an 80\%/20\% split of training and test data, as seen on \autoref{tab:methodsused}.
Another interesting aspect to look into for the various papers that have been researched is which metrics they use for comparisons.
The results of this can be seen on \autoref{tab:metricsused}.

\begin{table*}[]
    \centering
    \begin{tabular}{|l|l|l|l|l|l|l|l|l|l|}
    \hline
             & \textbf{MSE} & \textbf{RMSE} & \textbf{MAE} & \textbf{Precision} & \textbf{Recall} & \textbf{NDCG} & \textbf{F1} & \textbf{Hit rate} & \textbf{MAP} \\ \hline
KGAT\cite{KGAT}        &              &               &              &                    &                 & x             &             &                   &              \\ \hline
IS-UserBased\cite{GraphBasedCollaborativePaper} &              &               &              &                    &                 & x             &             &                   & x            \\ \hline
NeuMF\cite{neuMF}       &              &               &              &                    &                 & x             &             & x                 &              \\ \hline
LightGCN\cite{LightGCN}    &              &               &              &                    & x               & x             &             &                   &              \\ \hline
NGCF\cite{NGCF}         &              &               &              &                    & x               & x             &             &                   &              \\ \hline
DeepWalk\cite{DeepWalk}          &              &               &              & Implicit                & Implicit             &               & x           &                   &              \\ \hline
NMF\cite{NMF}          &              & x             &              &                    &                 &               &             &                   &              \\ \hline
CAMF(I)\cite{baltrunasCAMF}         &              &               & x            &                    &                 &               &             &                   &              \\ \hline
SVD\cite{standardMF}          & x            & x             &              &                    &                 &               &             &                   &              \\ \hline
SVD++\cite{svd++}       &              & x             &              &                    &                 &               &             &                   &              \\ \hline
    \end{tabular}
    \caption{Metrics used}
    \label{tab:metricsused}
\end{table*}

\begin{table*}[]
    \begin{tabular}{|l|l|}
    \hline
    \textbf{}     & \textbf{Method used}                          \\ \hline
    \textbf{KGAT} & Training, validation, test (70\%, 10\%, 20\%) \\ \hline
    IS-UserBased  & 10-fold cross validation                      \\ \hline
    NeuMF         & Leave-one-out                                 \\ \hline
    LightGCN      & From original authors (including splits)      \\ \hline
    NGCF          & Training, validation, test (70\%, 10\%, 20\%) \\ \hline
    DeepWalk      & 10-fold cross validation                      \\ \hline
    NMF           & 80/20 split and 5-fold cross validation       \\ \hline
    CAMF          & 25 splits of training, test (90\%, 10\%)      \\ \hline
    SVD           & Netflix prize testing                         \\ \hline
    SVD++         & Test, validation (both 1.4 million ratings)   \\ \hline
    \end{tabular}
    \caption{Methods used}
    \label{tab:methodsused}
\end{table*}

\subsection{Summary}\label{sec:datasetsummary}
A summary of the details of the datasets can be seen in \autoref{tab:datasetstats}.
The most interesting datasets for context-aware recommendation are LDOS-CoMoDa due to the amount of contextual values, the Yahoo! dataset due to the size and inclusion of context, the MovieLens datasets due to their size and ability to extract a temporal context, as well as Frappé due to its size.
Most of the examined papers perform offline evaluation, mainly employing recall and NDCG to evaluate the algorithms, and cross-validate on the sets when testing.

\begin{table*}[]
    \centering
    \begin{tabular}{|c|c|c|c|c|} 
    \hline
               & \#Ratings & \#Items & \#Users & \#Context variables  \\ 
    \hline
    LDOS-CoMoDa    & 2,296      & 1,232    & 121     & 12                   \\ 
    \hline
    MovieLens 1M   & 1,000,209 & 3,706    & 6,040    & 1                    \\ 
    \hline
    MovieLens 100K & 100,000   & 1,682    & 943     & 1                    \\ 
    \hline
    Frappé         & 96,203     & 957     & 4,082    & 8                    \\ 
    \hline
    InCarMusic     & 4,012      & 139     & 42      & 8                    \\ 
    \hline
    DePaulMovie    & 5,029      & 79      & 97      & 3                    \\
    \hline
    Yahoo!    & 211,231      & 11,915      & 7,642      & 2                    \\
    \hline
    \end{tabular}
    \caption{A final summary of the datasets.}
    \label{tab:datasetstats}
\end{table*}

\subsection{Synthetic data}
Datasets where the ratings are context-dependent are needed in order to enable testing of context-aware recommender systems.
One approach is to make a semi-synthetic dataset by adding a new contextual feature that the ratings will depend on \cite{baltrunasContextItemSplit}.
For this approach, the idea is to add a new contextual feature \texttt{c} that represents a contextual condition that could affect the rating.
The value of \texttt{c} is either 0 or 1. 
If \texttt{c} is 1, the corresponding rating between 1 and 5 is increased by 1, if it is not already 5.
If \texttt{c} is 0, the rating is instead decreased by 1 if is not already 1.
An $\alpha$ value is then chosen between 0 and 1 which determines the fractions of ratings that are affected by the contextual feature \texttt{c}.
By using this method, one can modify a dataset into being artificially context dependent for testing purposes.
This approach can be seen in \autoref{tab:synthetic-data1} and \autoref{tab:synthetic-data2}.
\begin{table*}[hbt!]
    \centering
    \begin{tabular}{|c|c|c|}
    \hline
    UserId & MovieId & Rating \\ [0.5ex] 
    \hline\hline
    1 & 101 & 3 \\
    \hline
    2 & 102 & 4 \\
    \hline
    \end{tabular}
    \caption{Ratings table without the contextual feature}
    \label{tab:synthetic-data1}
\end{table*}
\begin{table*}[hbt!]
    \centering
    \begin{tabular}{|c|c|c|c|} 
    \hline
    UserId & MovieId & Rating & Contextual\\Feature \\ [0.5ex] 
    \hline\hline
    1 & 101 & 2 & 0 \\
    \hline
    2 & 102 & 5 & 1 \\
    \hline
    \end{tabular}
    \caption{Ratings table with the contextual feature}
    \label{tab:synthetic-data2}
\end{table*}

\subsection{Experiments}\label{subsec:experimentprotocol}
Experiments are carried out in order to gain an understanding of baselines that employ different methods and investigate the impact of contextual data.
This section describes these experiments and the protocol used.

\subsubsection{Datasets}
Four datasets are considered for experimentation based on the investigation in \autoref{subsec:desc-of-datasets} and \autoref{sec:evaluationmetrics}: \textit{MovieLens 100k}, \textit{LDOS-CoMoDa}, \textit{Frappé}, as well as the \textit{Yahoo!} movie dataset of ratings and descriptive content information\cite{yahoo-movie}.
These datasets all provide contextual information, or allow for context to be inferred through the timestamps provided.
Users that have not provided at least 5 ratings are pruned for the \textit{LDOS-CoMoDa} dataset.
The \textit{Frappé} dataset does not provide explicit ratings, and methods that require this do not report results on this dataset.
The remaining datasets provide ratings in the range of $1-5$.
The contextual variables and side information used is defined in \autoref{tbl:sideinfoprotocol} and \autoref{tbl:contextprotocol}.
For the time interval defined based on the timestamp there are five possible values as defined in \autoref{tbl:contextprotocol}.
For the experiments, we define these as $06.00$-$09.00$ being the morning, $09.00$-$12.00$ being noon, $12.00$-$17.00$ being afternoon, $17.00$-$22.00$ being evening, and the remaining interval $22.00$-$05.00$ being night.
\\
\begin{table*}[!htb]
	\centering
    \begin{tabular}{|l|l|l|l|}
    \hline
                                                                         & \textbf{Yahoo}                                                                          & \textbf{MovieLens}                                                                  & \textbf{Frappe}                                                                                                                                        \\ \hline
    \textbf{\begin{tabular}[c]{@{}l@{}}Side-\\ information\end{tabular}} & \begin{tabular}[c]{@{}l@{}}Age group\\ (3 values)\\ \\ Gender\\ (2 values)\end{tabular} & \begin{tabular}[c]{@{}l@{}}Age group\\ (3 values)\\ \\ Gender\\ (2 values)\end{tabular} & \begin{tabular}[c]{@{}l@{}}Category\\ (32 values)\\ \\ Downloads\\ (16 values)\\ \\ Developer\\ (2809 values)\\ \\ Language\\ (29 values)\end{tabular} \\ \hline
    \end{tabular}
    \caption{The side information categories used for the experiments along with their dimensions.}
    \label{tbl:sideinfoprotocol}
\end{table*}

\begin{table*}[!htb]
	\centering
    \begin{tabular}{|l|l|l|}
    \hline
                                                                            & \textbf{MovieLens}                                                                            & \textbf{LDOS-CoMoDa}                                                                        \\ \hline
    \textbf{\begin{tabular}[c]{@{}l@{}}Contextual\\ variables\end{tabular}} & \begin{tabular}[c]{@{}l@{}}Day Of Week\\ (7 values)\\ \\ Time intervals\\ (5 values)\end{tabular} & \begin{tabular}[c]{@{}l@{}}DominantEmo\\ (5 values)\\ \\ Physical\\ (2 values)\end{tabular} \\ \hline
    \end{tabular}
    \caption{The contextual variables used for the experiments along with their dimensions.}
    \label{tbl:contextprotocol}
\end{table*}
\noindent
\textbf{Cross-validation}\\
For each dataset we employ \textit{5-fold} cross-validation.
We split the data into 5 folds, and employ 4 of these as training and 1 as test.
This procedure is repeated 5 times.
The same splits are used for evaluation for each method to ensure proper comparability between the methods.

\subsection{Experimental settings}
In the following we define the evaluation metrics used, the baselines that are compared as well as the hyperparameter settings used.
\\\\
\textbf{Evaluation metrics}\\
As defined in \autoref{sec:evaluationmetrics}, the metrics used are \textit{precision@N}, \textit{recall@N}, \textit{map@N}, \textit{RMSE} and \textit{MAE}.
All five metrics are reported for baselines that provide rating predictions, whereas \textit{RMSE} and \textit{MAE} are not reported for methods that only compute a list of \textit{top-N} recommendations.
For \textit{precision@N}, \textit{recall@N} and \textit{map@N}, the k-value is defined as $10$.
For evaluation purposes we define a relevant item for a given user as an item in which a rating with a value of $3$ or more is observed.
For methods that evaluate on batches, we employ a batch size of approximately one tenth of the total number of users.
For different learning rates and embedding sizes for methods that require these, we use the ones provided by the authors as examples for the methods. 
If the methods require random sampling, we employed the number 42 as the seed.
\\\\
\textbf{Baselines and state-of-the-art}\\
The experiments are carried out on the following baselines:
\begin{itemize}
    \item \textbf{Random}: Using random predictions, which is done by either assigning a random rating for rating prediction, or by generating a random list for top K list recommendations.
    \item \textbf{Non-negative MF}: Another variation of a matrix factorization approach for collaborative filtering, in which the matrices representing the latent factor space are subject to the non-negative constraint, i.e. $\geq$ 0.
    \item \textbf{SVD}: A standard matrix factorization approach. This is a latent factor approach, where items and users are transformed to the same latent factor space, explaining items and users on factors automatically inferred.
    \item \textbf{SVD++}: An extension of the matrix factorization latent factor approach, combining it with neighborhood models that analyze similarities between products and users.
\end{itemize}

And the following state-of-the-art methods:
\begin{itemize}
    \item \textbf{KGAT}: Considers side information when providing recommendation by linking items with their attributes as higher-order relations in a knowledge graph. Embeddings from a node's neighbor are recursively propagated, and an attention mechanism is used to discriminate importance of neighbors.
    \item \textbf{ItemSplitting}: A context-based recommendation approach making use of transitive associations in a bipartite graph to model and find nearest neighbor similarity to produce recommendations on a dataset that has been split on items.
    \item \textbf{LightGCN}: A graph-convolutional network approach that learns user and item embeddings by linearly propagating them on a user-item interaction graph and uses the weighted sum of the embeddings learned at all layers as the final embedding.
    \item \textbf{NGCF}: A neural-network based approach that integrates user-item interactions into embeddings through a bipartite graph-structure, leveraging higher-order connectivities to incorporate the collaborative signal.
    \item \textbf{DeepWalk + KNN}: This approach learns latent representations of nodes in a network, using information from truncated random walks. The embeddings learned through this approach are used for a k-nearest neighbor recommendation approach.
    \item \textbf{CAMF-C}: A context-aware extension of the SVD approach, introducing a single parameter for all contextual conditions, modeling how the rating deviates from the effect of the context.
    \item \textbf{CAMF-CI}: A variation of CAMF-C, introducing one parameter per contextual condition and item pair.
\end{itemize}

\subsection{Results of experiments}\label{subsec:resultsofexperiment}
In the experiments, we ran a number of methods on four different datasets that contain either contextual information or side-information.
The methods included two context-aware methods, some state-of-the-art graph-based methods, and some simple baselines.
These methods were evaluated on a number of different metrics that showed their performance on the different datasets.
In this section we look at the results from the experiments.
\\\\
\Cref{fig:comoda-rmse,fig:yahoo-rmse,fig:movielens-rmse} visualize RMSE and MAE for LDOS-CoMoDa, Yahoo! and MovieLens respectively, whereas \Cref{fig:comoda-mae,fig:yahoo-mae,fig:movielens-mae} visualize MAE for the respective datasets.\\
For the LDOS-CoMoDa dataset, the metrics Precision@10, Recall@10, NDCG@10, MAP@10 and F1@10 are visualized in respectively \Cref{fig:comoda-precision,fig:comoda-recall,fig:comoda-ndcg,fig:comoda-map,fig:comoda-f1}.
Likewise, for the MovieLens 100K datasets the metrics are visualized in \Cref{fig:movielens-precision,fig:movielens-recall,fig:movielens-ndcg,fig:movielens-map,fig:movielens-f1}.
The same metrics for Yahoo! Movies are visualized in \Cref{fig:yahoo-precision,fig:yahoo-recall,fig:yahoo-ndcg,fig:yahoo-map,fig:yahoo-f1}.
Finally, the Frappé dataset visualizes results on Precision@10, Recall@10, NDCG@10, MAP@10 and F1@10 on respectively \Cref{fig:frappe-precision,fig:frappe-recall,fig:frappe-ndcg,fig:frappe-map,fig:frappe-f1}.

\subsubsection{LDOS-CoMoDa}
For the experiments conducted on the LDOS-CoMoDa dataset, we should see the methods that handle context do well, as this dataset has many context variables that the rating depends on.
Two of these context variables are used for the experiments, which are the dominantEmo and Physical as these are some of the most influential, as described in \Cref{subsec:desc-of-datasets}.
The methods that handle context are \textit{IS-UserBased} and the \textit{CAMF} methods.
We see that \textit{IS-UserBased} has the largest Recall@10, but also the third smallest Precision@10, resulting in a small F1@10 value, and in general, it does not perform well on LDOS-CoMoDa, which is surprising as it is able to utilize the context.
\\
Both \textit{CAMF-C} and \textit{CAMF-CI} perform well on LDOS-CoMoDA in terms of most metrics, except for MAP@10 and NDCG@10.
They are able to achieve the lowest prediction errors amongst all methods in terms of RMSE and MAE, which could indicate that they are able to make good use of the context.
Close to the results of the \textit{CAMF} methods we have baselines such as \textit{SVD} and \textit{SVD++} which both are competitive in terms of Precision@10, Recall@10 and F1@10. 
These also achieve the highest MAP@10 values but do not have as low prediction errors as the \textit{CAMF} methods.
\textit{NMF} generally performs a little worse in all metrics compared to \textit{SVD}, except for NDCG@10.
\\
The \textit{DeepWalk + kNN} method has high prediction errors and low Precision@10 but has the second-highest Recall@10 and one of the highest NDCG@10 values.
Finally, we have the methods that do not handle rating predictions but only do top-$N$ recommendations, so these do not have RMSE and MAE values, since they work with interactions rather than ratings.
We see that \textit{LightGCN} is not able to perform well on any of the metrics.
We suspect that this is because the LDOS-CoMoDa dataset is as small as it is with only 121 users, 1,232 unique movies, and 2,296 ratings, and it is therefore not able to utilize its strengths.
All of these numbers are also lower on the actual data used, due to pruning.
However, \textit{NGCF} is able to achieve both the highest Precision@10 as well as the highest NDCG@10, but it does have a lower Recall@10 resulting in a low F1@10 score.
\\\\
To summarize, we expected the context-aware methods to perform well on the LDOS-CoMoDA dataset, as the ratings are context-dependent. 
The \textit{CAMF} methods do perform well as expected, and \textit{IS-UserBased} performs well in Recall@10, NDCG@10, RMSE and MAE but poorly in the remaining metrics.
The results of the experiment could also indicate that \textit{NGCF} could be interesting to look at in terms of adding context, as it is able to perform well on a context dataset without utilizing the context available. 
\textit{NGCF} is also a graph-based approach which, as mentioned in \Cref{sec:graph-rec-sys}, is able to alleviate the sparsity problem introduced by adding context.
\pgfplotstableread{
x y y-min y-max
{SVD} 1.020 0.068 0.116
{SVD++} 1.020 0.064 0.115
{Random} 1.763 0.053 0.081
{CAMF-CI}  0.824 0.046 0.061
{CAMF-C}  0.669 0.028 0.027
{DeepWalk + kNN} 1.216 0.040 0.054
{IS-UserBased} 0.944 0.181 0.184
{NMF} 1.173 0.071 0.092
}{\differanser}
\begin{tikzpicture}[scale=1] 
\begin{axis} [
title    = {MovieLens: RMSE},
width  = 0.5*\textwidth,
height = 8cm,
symbolic x coords={{SVD},{SVD++},{Random},{CAMF-CI},{CAMF-C},{DeepWalk + kNN},{IS-UserBased},{NMF}},
minor ytick={0.9,1,1.1,1.2,1.3,1.4,1.5,1.6,1.7,1.8},
yminorgrids,
xtick=data,
ticklabel style = {font=\tiny},
x tick label style={rotate=60,anchor=east},
legend style={at={(0.05,0.95)},anchor=north west,cells={anchor=west},column
sep=1ex}
]
\addplot+[blue, very thick, forget plot,only marks] 
plot[very thick, error bars/.cd, y dir=plus, y explicit]
table[x=x,y=y,y error expr=\thisrow{y-max}] {\differanser};
\addplot+[red, very thick, only marks,xticklabels=\empty] 
plot[very thick, error bars/.cd, y dir=minus, y explicit]
table[x=x,y=y,y error expr=\thisrow{y-min}] {\differanser};
\end{axis} 
\end{tikzpicture}

\pgfplotstableread{
x y y-min y-max
{SVD} 0.826 0.046 0.094
{SVD++} 0.828 0.047 0.099
{Random} 1.454 0.062 0.078
{CAMF-CI}  0.625 0.026 0.030
{CAMF-C}  0.475 0.023 0.012
{DeepWalk + kNN} 0.887 0.030 0.069
{IS-UserBased} 0.629 0.133 0.098
{NMF} 0.932 0.050 1.017
}{\differanser}
\begin{tikzpicture}[scale=1] 
\begin{axis} [
title    = {MovieLens: MAE},
width  = 0.5*\textwidth,
height = 8cm,
symbolic x coords={{SVD},{SVD++},{Random},{CAMF-CI},{CAMF-C},{DeepWalk + kNN},{IS-UserBased},{NMF}},
minor ytick={0.7,0.8,0.9,1,1.1,1.2,1.3,1.4},
yminorgrids,
xtick=data,
ticklabel style = {font=\tiny},
x tick label style={rotate=60,anchor=east},
legend style={at={(0.05,0.95)},anchor=north west,cells={anchor=west},column
sep=1ex}
]
\addplot+[blue, very thick, forget plot,only marks] 
plot[very thick, error bars/.cd, y dir=plus, y explicit]
table[x=x,y=y,y error expr=\thisrow{y-max}] {\differanser};
\addplot+[red, very thick, only marks,xticklabels=\empty] 
plot[very thick, error bars/.cd, y dir=minus, y explicit]
table[x=x,y=y,y error expr=\thisrow{y-min}] {\differanser};
\end{axis} 
\end{tikzpicture}

\pgfplotstableread{
x y y-min y-max
{SVD} 0.390 0 0.001
{SVD++} 0.388 0.001 0.002
{Random} 0.227 0.002 0.002
{CAMF-CI}  0.390 0.001 0.002
{CAMF-C}  0.393 0.002 0.003
{DeepWalk + kNN} 0.290 0.002 0.002
{IS-UserBased} 0.217 0.009 0.021
{NMF} 0.377 0.001 0.001
{LightGCN} 0.148 0.002 0.001
{NGCF} 0.141 0.002 0.008
{KGAT} 0.139 0.001 0.003
}{\differanser}
\begin{tikzpicture}[scale=1] 
\begin{axis} [
title    = {MovieLens: Precision},
width  = 0.5*\textwidth,
height = 8cm,
symbolic x coords={{SVD},{SVD++},{Random},{CAMF-CI},{CAMF-C},{DeepWalk + kNN},{IS-UserBased},{NMF},{LightGCN},{NGCF},{KGAT}},
minor ytick={0.0,0.1,0.2,0.3,0.4,0.5,0.6,0.7,0.8},
yminorgrids,
xtick=data,
ticklabel style = {font=\tiny},
x tick label style={rotate=60,anchor=east},
legend style={at={(0.05,0.95)},anchor=north west,cells={anchor=west},column
sep=1ex}
]
\addplot+[blue, very thick, forget plot,only marks] 
plot[very thick, error bars/.cd, y dir=plus, y explicit]
table[x=x,y=y,y error expr=\thisrow{y-max}] {\differanser};
\addplot+[red, very thick, only marks,xticklabels=\empty] 
plot[very thick, error bars/.cd, y dir=minus, y explicit]
table[x=x,y=y,y error expr=\thisrow{y-min}] {\differanser};
\end{axis} 
\end{tikzpicture}

\pgfplotstableread{
x y y-min y-max
{SVD} 0.894 0.002 0.003
{SVD++} 0.886 0.003 0.003
{Random} 0.481 0.006 0.005
{CAMF-CI}  0.892 0.003 0.002
{CAMF-C}  0.903 0.006 0.004
{DeepWalk + kNN} 0.877 0.004 0.002
{IS-UserBased} 0.910 0.038 0.023
{NMF} 0.853 0.004 0.002
{LightGCN} 0.342 0.003 0.003
{NGCF} 0.328 0.008 0.020 
{KGAT} 0.322 0.008 0.009
}{\differanser}
\begin{tikzpicture}[scale=1] 
\begin{axis} [
title    = {MovieLens: Recall},
width  = 0.5*\textwidth,
height = 8cm,
symbolic x coords={{SVD},{SVD++},{Random},{CAMF-CI},{CAMF-C},{DeepWalk + kNN},{IS-UserBased},{NMF},{LightGCN},{NGCF},{KGAT}},
minor ytick={0.0,0.1,0.2,0.3,0.4,0.5,0.6,0.7,0.8},
yminorgrids,
xtick=data,
ticklabel style = {font=\tiny},
x tick label style={rotate=60,anchor=east},
legend style={at={(0.05,0.95)},anchor=north west,cells={anchor=west},column
sep=1ex}
]
\addplot+[blue, very thick, forget plot,only marks] 
plot[very thick, error bars/.cd, y dir=plus, y explicit]
table[x=x,y=y,y error expr=\thisrow{y-max}] {\differanser};
\addplot+[red, very thick, only marks,xticklabels=\empty] 
plot[very thick, error bars/.cd, y dir=minus, y explicit]
table[x=x,y=y,y error expr=\thisrow{y-min}] {\differanser};
\end{axis} 
\end{tikzpicture}

\pgfplotstableread{
x y y-min y-max
{SVD} 0.502 0.013 0.014
{SVD++} 0.507 0.008 0.010
{Random} 0.451 0.017 0.019
{CAMF-CI}  0.476 0.104 0.083
{CAMF-C}  0.492 0.055 0.070
{DeepWalk + kNN} 0.511 0.008 0.007
{IS-UserBased} 0.558 0.057 0.076
{NMF} 0.473 0.023 0.052
{LightGCN} 0.297 0.003 0.005
{NGCF}  0.316 0.006 0.018
{KGAT} 0.438 0.007 0.011
}{\differanser}
\begin{tikzpicture}[scale=1] 
\begin{axis} [
title  = {MovieLens: NDCG},
width  = 0.5*\textwidth,
height = 8cm,
symbolic x coords={{SVD},{SVD++},{Random},{CAMF-CI},{CAMF-C},{DeepWalk + kNN},{IS-UserBased},{NMF},{LightGCN},{NGCF},{KGAT}},
minor ytick={0.0,0.1,0.2,0.3,0.4,0.5,0.6},
yminorgrids,
xtick=data,
ticklabel style = {font=\tiny},
x tick label style={rotate=60,anchor=east},
legend style={at={(0.05,0.95)},anchor=north west,cells={anchor=west},column
sep=1ex}
]
\addplot+[blue, very thick, forget plot,only marks] 
plot[very thick, error bars/.cd, y dir=plus, y explicit]
table[x=x,y=y,y error expr=\thisrow{y-max}] {\differanser};
\addplot+[red, very thick, only marks,xticklabels=\empty] 
plot[very thick, error bars/.cd, y dir=minus, y explicit]
table[x=x,y=y,y error expr=\thisrow{y-min}] {\differanser};
\end{axis}
\end{tikzpicture}

\pgfplotstableread{
x y y-min y-max
{SVD} 0.0046 0.0005 0.0003
{SVD++} 0.0062 0.0006 0.0010
{Random} 0.00055 0.00012 0.00027
{CAMF-CI}  0.00007 0.00004 0.00005
{CAMF-C}  0.00013 0.00005 0.00007
{DeepWalk + kNN} 0.044 0.001 0.003
{IS-UserBased} 0.046 0.010 0.011
{NMF} 0.00075 0.00011 0.00019
{LightGCN} 0.000007 0.000001 0
{NGCF}  0.0000112 0.0000002 0.0000006
{KGAT} 0.0000018 0.0000005 0.00000004
}{\differanser}
\begin{tikzpicture}[scale=1] 
\begin{axis} [
title  = {MovieLens: MAP},
width  = 0.5*\textwidth,
height = 8cm,
symbolic x coords={{SVD},{SVD++},{Random},{CAMF-CI},{CAMF-C},{DeepWalk + kNN},{IS-UserBased},{NMF},{LightGCN},{NGCF},{KGAT}},
minor ytick={0.0,0.1},
yminorgrids,
xtick=data,
ticklabel style = {font=\tiny},
x tick label style={rotate=60,anchor=east},
legend style={at={(0.05,0.95)},anchor=north west,cells={anchor=west},column
sep=1ex}
]
\addplot+[blue, very thick, forget plot,only marks] 
plot[very thick, error bars/.cd, y dir=plus, y explicit]
table[x=x,y=y,y error expr=\thisrow{y-max}] {\differanser};
\addplot+[red, very thick, only marks,xticklabels=\empty] 
plot[very thick, error bars/.cd, y dir=minus, y explicit]
table[x=x,y=y,y error expr=\thisrow{y-min}] {\differanser};
\end{axis}
\end{tikzpicture}


\subsubsection{MovieLens 100k}
The MovieLens 100k dataset does not contain any contextual variables outside of a timestamp that can be discretized.
\Cref{subsec:experimentprotocol} defined how the timestamp was discretized into the day of the week and a time interval.
Compared to the LDOS-CoMoDa we expected context-aware methods to not perform substantially better since only context based on the timestamp is available compared to the different contextual variables included in the LDOS-CoMoDa dataset. 
In terms of the methods that handle context, we see that \textit{CAMF-C} achieves the highest Precision@10 and F1@10, while \textit{IS-UserBased} underperforms most methods on Precision@10 and outperforms all methods on Recall@10.
\\
In terms of NDCG@10 \textit{LightGCN} underperforms on MovieLens, showing the worst results, while both \textit{CAMF} methods also provide poorer NDCG@10.
\textit{KGAT}, a graph-based method utilizing side-information performs the best in terms of NDCG@10 on the MovieLens data but underperforms on MAP@10.
Other graph-based methods, such as \textit{NGCF} and \textit{LightGCN} show similar poor performances as well on this dataset, being outperformed on Precision@10, Recall@10, F1@10, and MAP@10.
\textit{DeepWalk} outperforms other graph-based methods on Precision@10, Recall@10, and F1@10, showing results close to SVD-based approaches, as well as the best results on MAP@10.
\\
In terms of RMSE and MAE, we see that both \textit{CAMF} and both \textit{SVD} approaches perform the best, with \textit{Random} underperforming as expected.
Overall, we see a tendency for the graph-based methods to perform poorly on this dataset except for \textit{DeepWalk}, while the SVD-based methods perform the best in terms of F1@10.
% MovieLens

\pgfplotstableread{
x y y-min y-max
{SVD} 0.935 0.003 0.007
{SVD++} 0.962 0.003 0.003
{Random} 1.702 0.007 0.013
{CAMF-CI}  0.978 0.007 0.002
{CAMF-C}  0.931 0.002 0.004
{DeepWalk + kNN} 1.124 0.010 0.004
{IS-UserBased} 1.366 0.018 0.021
{NMF} 0.962 0.003 0.004
}{\differanser}
\begin{figure}
\begin{tikzpicture}[scale=1] 
\begin{axis} [
title    = {MovieLens: RMSE},
width  = 0.5*\textwidth,
height = 8cm,
symbolic x coords={{CAMF-C},{SVD},{NMF},{SVD++},{CAMF-CI},{DeepWalk + kNN},{IS-UserBased},{Random}},
minor ytick={0.9,1,1.1,1.2,1.3,1.4,1.5,1.6,1.7,1.8},
yminorgrids,
xtick=data,
ticklabel style = {font=\tiny},
x tick label style={rotate=60,anchor=east},
legend style={at={(0.05,0.95)},anchor=north west,cells={anchor=west},column
sep=1ex}
]
\addplot+[blue, very thick, forget plot,only marks] 
plot[very thick, error bars/.cd, y dir=plus, y explicit]
table[x=x,y=y,y error expr=\thisrow{y-max}] {\differanser};
\addplot+[red, very thick, only marks,xticklabels=\empty] 
plot[very thick, error bars/.cd, y dir=minus, y explicit]
table[x=x,y=y,y error expr=\thisrow{y-min}] {\differanser};
\end{axis} 
\end{tikzpicture}
\caption{Graph that shows RMSE for MovieLens.}
\label{fig:movielens-rmse}
\end{figure}

\pgfplotstableread{
x y y-min y-max
{SVD} 0.737 0.003 0.005
{SVD++} 0.757 0.004 0.002
{Random} 1.391 0.008 0.014
{CAMF-CI}  0.773 0.004 0.004
{CAMF-C}  0.738 0.005 0.004
{DeepWalk + kNN} 0.871 0.006 0.005
{IS-UserBased} 1.041 0.013 0.010
{NMF} 0.757 0.004 0.002
}{\differanser}
\begin{figure}
\begin{tikzpicture}[scale=1] 
\begin{axis} [
title    = {MovieLens: MAE},
width  = 0.5*\textwidth,
height = 8cm,
symbolic x coords={{SVD},{CAMF-C},{SVD++},{NMF},{CAMF-CI},{DeepWalk + kNN},{IS-UserBased},{Random}},
minor ytick={0.7,0.75,0.8,0.85,0.9,0.95,1,1.05,1.1,1.15,1.2,1.25,1.3,1.35,1.4},
yminorgrids,
xtick=data,
ticklabel style = {font=\tiny},
x tick label style={rotate=60,anchor=east},
legend style={at={(0.05,0.95)},anchor=north west,cells={anchor=west},column
sep=1ex}
]
\addplot+[blue, very thick, forget plot,only marks] 
plot[very thick, error bars/.cd, y dir=plus, y explicit]
table[x=x,y=y,y error expr=\thisrow{y-max}] {\differanser};
\addplot+[red, very thick, only marks,xticklabels=\empty] 
plot[very thick, error bars/.cd, y dir=minus, y explicit]
table[x=x,y=y,y error expr=\thisrow{y-min}] {\differanser};
\end{axis} 
\end{tikzpicture}
\caption{Graph that shows MAE for MovieLens.}
\label{fig:movielens-mae}
\end{figure}

\pgfplotstableread{
x y y-min y-max
{SVD} 0.719 0.008 0.009
{SVD++} 0.702 0.006 0.009
{Random} 0.538 0.009 0.005
{CAMF-CI}  0.713 0.007 0.007
{CAMF-C}  0.726 0.012 0.007
{DeepWalk + kNN} 0.679 0.005 0.006
{IS-UserBased} 0.344 0.013 0.022
{NMF} 0.702 0.006 0.009
{LightGCN} 0.249 0.003 0.004
{NGCF} 0.339 0.008 0.006
{KGAT} 0.353 0.007 0.009
}{\differanser}
\begin{figure}
\begin{tikzpicture}[scale=1] 
\begin{axis} [
title    = {MovieLens: Precision@10},
width  = 0.5*\textwidth,
height = 8cm,
symbolic x coords={{CAMF-C},{SVD},{CAMF-CI},{SVD++},{NMF},{DeepWalk + kNN},{Random},{KGAT},{IS-UserBased},{NGCF},{LightGCN}},
minor ytick={0.2,0.25,0.3,0.35,0.4,0.45,0.5,0.55,0.6,0.65,0.7,0.75},
yminorgrids,
xtick=data,
ticklabel style = {font=\tiny},
x tick label style={rotate=60,anchor=east},
legend style={at={(0.05,0.95)},anchor=north west,cells={anchor=west},column
sep=1ex}
]
\addplot+[blue, very thick, forget plot,only marks] 
plot[very thick, error bars/.cd, y dir=plus, y explicit]
table[x=x,y=y,y error expr=\thisrow{y-max}] {\differanser};
\addplot+[red, very thick, only marks,xticklabels=\empty] 
plot[very thick, error bars/.cd, y dir=minus, y explicit]
table[x=x,y=y,y error expr=\thisrow{y-min}] {\differanser};
\end{axis} 
\end{tikzpicture}
\caption{Graph that shows Precision@10 for MovieLens.}
\label{fig:movielens-precision}
\end{figure}

\pgfplotstableread{
x y y-min y-max
{SVD} 0.633 0.007 0.004
{SVD++} 0.606 0.010 0.010
{Random} 0.415 0.002 0.001
{CAMF-CI}  0.629 0.007 0.002
{CAMF-C}  0.645 0.011 0.005
{DeepWalk + kNN} 0.659 0.006 0.004
{IS-UserBased} 0.785 0.013 0.011
{NMF} 0.606 0.010 0.010
{LightGCN} 0.166 0.004 0.004
{NGCF} 0.223 0.010 0.005
{KGAT} 0.232 0.004 0.005
}{\differanser}
\begin{figure}
\begin{tikzpicture}[scale=1] 
\begin{axis} [
title    = {MovieLens: Recall@10},
width  = 0.5*\textwidth,
height = 8cm,
symbolic x coords={{IS-UserBased},{DeepWalk + kNN},{CAMF-C},{SVD},{CAMF-CI},{SVD++},{NMF},{Random},{KGAT},{NGCF},{LightGCN}},
minor ytick={0.15,0.2,0.25,0.3,0.35,0.4,0.45,0.5,0.55,0.6,0.65,0.7,0.75,0.8},
yminorgrids,
xtick=data,
ticklabel style = {font=\tiny},
x tick label style={rotate=60,anchor=east},
legend style={at={(0.05,0.95)},anchor=north west,cells={anchor=west},column
sep=1ex}
]
\addplot+[blue, very thick, forget plot,only marks] 
plot[very thick, error bars/.cd, y dir=plus, y explicit]
table[x=x,y=y,y error expr=\thisrow{y-max}] {\differanser};
\addplot+[red, very thick, only marks,xticklabels=\empty] 
plot[very thick, error bars/.cd, y dir=minus, y explicit]
table[x=x,y=y,y error expr=\thisrow{y-min}] {\differanser};
\end{axis} 
\end{tikzpicture}
\caption{Graph that shows Recall@10 for MovieLens.}
\label{fig:movielens-recall}
\end{figure}

\pgfplotstableread{
x y y-min y-max
{SVD} 0.508 0.014 0.010
{SVD++} 0.441 0.050 0.042
{Random} 0.476 0.016 0.014
{CAMF-CI}  0.402 0.046 0.095
{CAMF-C}  0.375 0.032 0.068
{DeepWalk + kNN} 0.512 0.003 0.017
{IS-UserBased} 0.527 0.036 0.016
{NMF} 0.443 0.042 0.045
{LightGCN} 0.295 0.007 0.006
{NGCF}  0.420 0.010 0.007
{KGAT} 0.691 0.007 0.015
}{\differanser}
\begin{figure}
\begin{tikzpicture}[scale=1] 
\begin{axis} [
title  = {MovieLens: NDCG@10},
width  = 0.5*\textwidth,
height = 8cm,
symbolic x coords={{KGAT},{IS-UserBased},{DeepWalk + kNN},{SVD},{Random},{NMF},{SVD++},{NGCF},{CAMF-CI},{CAMF-C},{LightGCN}},
minor ytick={0.3,0.35,0.4,0.45,0.5,0.55,0.6,0.65,0.7},
yminorgrids,
xtick=data,
ticklabel style = {font=\tiny},
x tick label style={rotate=60,anchor=east},
legend style={at={(0.05,0.95)},anchor=north west,cells={anchor=west},column
sep=1ex}
]
\addplot+[blue, very thick, forget plot,only marks] 
plot[very thick, error bars/.cd, y dir=plus, y explicit]
table[x=x,y=y,y error expr=\thisrow{y-max}] {\differanser};
\addplot+[red, very thick, only marks,xticklabels=\empty] 
plot[very thick, error bars/.cd, y dir=minus, y explicit]
table[x=x,y=y,y error expr=\thisrow{y-min}] {\differanser};
\end{axis}
\end{tikzpicture}
\caption{Graph that shows NDCG@10 for MovieLens.}
\label{fig:movielens-ndcg}
\end{figure}

\pgfplotstableread{
x y y-min y-max
{SVD} 0.0163 0.0016 0.0011
{SVD++} 0.0047 0.0005 0.0005
{Random} 0.0044 0.0003 0.0004
{CAMF-CI}  0.00071 0.00034 0.00029
{CAMF-C}  0.00248 0.00181 0.0033
{DeepWalk + kNN} 0.029 0.0018 0.0023
{IS-UserBased} 0.012 0.0013 0.0008
{NMF} 0.0049 0.0004 0.0004
{LightGCN} 0.000079 0.000001 0
{NGCF}  0.000254 0.000006 0.000004
{KGAT} 0.00004 0 0
}{\differanser}
\begin{figure}
\begin{tikzpicture}[scale=1] 
\begin{axis} [
title  = {MovieLens: MAP@10},
width  = 0.5*\textwidth,
height = 8cm,
symbolic x coords={{DeepWalk + kNN},{SVD},{IS-UserBased},{NMF},{SVD++},{Random},{CAMF-C},{CAMF-CI},{NGCF},{LightGCN},{KGAT}},
minor ytick={0,0.005,0.01,0.015,0.02,0.025,0.03},
yminorgrids,
xtick=data,
ticklabel style = {font=\tiny},
x tick label style={rotate=60,anchor=east},
legend style={at={(0.05,0.95)},anchor=north west,cells={anchor=west},column
sep=1ex}
]
\addplot+[blue, very thick, forget plot,only marks] 
plot[very thick, error bars/.cd, y dir=plus, y explicit]
table[x=x,y=y,y error expr=\thisrow{y-max}] {\differanser};
\addplot+[red, very thick, only marks,xticklabels=\empty] 
plot[very thick, error bars/.cd, y dir=minus, y explicit]
table[x=x,y=y,y error expr=\thisrow{y-min}] {\differanser};
\end{axis}
\end{tikzpicture}
\caption{Graph that shows MAP@10 for MovieLens.}
\label{fig:movielens-map}
\end{figure}


\pgfplotstableread{
x y y-min y-max
{SVD} 0.673 0.006 0.003
{SVD++} 0.651 0.009 0.004
{Random} 0.469 0.005 0.002
{CAMF-CI} 0.668 0.001 0.002
{CAMF-C} 0.683 0.009 0.007
{DeepWalk + kNN} 0.669 0.006 0.004
{IS-UserBased} 0.478 0.013 0.018
{NMF} 0.651 0.009 0.007
{LightGCN} 0.199 0.003 0.002
{NGCF}  0.269 0.010 0.005
{KGAT} 0.280 0.004 0.006
}{\differanser}
\begin{figure}
\begin{tikzpicture}[scale=1] 
\begin{axis} [
title  = {MovieLens: F1@10},
width  = 0.5*\textwidth,
height = 8cm,
symbolic x coords={{CAMF-C},{SVD},{DeepWalk + kNN},{CAMF-CI},{SVD++},{NMF},{IS-UserBased},{Random},{KGAT},{NGCF},{LightGCN}},
minor ytick={0.2,0.25,0.3,0.35,0.4,0.45,0.5,0.55,0.6,0.65,0.7},
yminorgrids,
xtick=data,
ticklabel style = {font=\tiny},
x tick label style={rotate=60,anchor=east},
legend style={at={(0.05,0.95)},anchor=north west,cells={anchor=west},column
sep=1ex}
]
\addplot+[blue, very thick, forget plot,only marks] 
plot[very thick, error bars/.cd, y dir=plus, y explicit]
table[x=x,y=y,y error expr=\thisrow{y-max}] {\differanser};
\addplot+[red, very thick, only marks,xticklabels=\empty] 
plot[very thick, error bars/.cd, y dir=minus, y explicit]
table[x=x,y=y,y error expr=\thisrow{y-min}] {\differanser};
\end{axis}
\end{tikzpicture}
\caption{Graph that shows F1@10 for MovieLens.}
\label{fig:movielens-f1}
\end{figure}


\subsubsection{Yahoo! Movies}
For the Yahoo! Movies dataset, we opted to use side-information such as gender and age group as the contextual variables for the context-aware methods since the dataset does not contain any contextual information with the recorded interactions.
Because of this, we do not expect the context-aware methods to perform better than the other baselines.
We see that \textit{SVD++} performs best in terms of RMSE and MAE, closely followed by other methods such as \textit{SVD}, \textit{CAMF}, and \textit{NMF}.
\textit{CAMF-C} is able to achieve the highest Precision@10 which could suggest that it is able to utilize the side information, but it is only marginally better than \textit{SVD}, \textit{SVD++}, and \textit{CAMF-CI}.
\\
For Recall@10, MAP@10, and NDCG@10 \textit{IS-UserBased} is the best performing method.
As this method is context-aware, this could suggest that it is able to improve its predictions using side information.
The high values of MAP@10 and NDCG@10 also suggest that it is better at recommending relevant items in the front of the list compared to other methods.
However \textit{IS-UserBased} does not perform well on Precision@10 nor RMSE.
\textit{DeepWalk} only performs well on Recall@10, NDCG@10 and MAP@10 and struggles on the other metrics.
\\
\textit{LightGCN}, \textit{NGCF} and \textit{KGAT} all perform evenly in Precision@10, but still show worse results than the methods that are not graph-based.
\textit{LightGCN} shows the best Precision@10 of these methods, but its Recall@10 and NDCG@10 are worse than many of the baselines.
The methods achieve much worse MAP@10 values than the other methods.
\textit{KGAT} has a very low Precision@10 but achieves an NDCG@10 that is almost comparable to the other baselines.
\\
In general, it does not seem like these three methods are suitable for this kind of dataset.
\pgfplotstableread{
x y y-min y-max
{SVD} 1.013 0.003 0.004
{SVD++} 0.999 0.001 0.001
{Random} 1.996 0.005 0.003
{CAMF-CI}  1.030 0.002 0.002
{CAMF-C}  1.003 0.002 0.002
{DeepWalk + kNN} 1.197 0.007 0.006
{IS-UserBased} 1.235 0.124 0.218
{NMF} 1.089 0.003 0.003
}{\differanser}
\begin{figure}
\begin{tikzpicture}[scale=1] 
\begin{axis} [
title    = {Yahoo!: RMSE},
width  = 0.5*\textwidth,
height = 8cm,
symbolic x coords={{SVD++},{CAMF-C},{SVD},{CAMF-CI},{NMF},{DeepWalk + kNN},{IS-UserBased},{Random}},
minor ytick={0.9,1,1.1,1.2,1.3,1.4,1.5,1.6,1.7,1.8},
yminorgrids,
xtick=data,
ticklabel style = {font=\tiny},
x tick label style={rotate=60,anchor=east},
legend style={at={(0.05,0.95)},anchor=north west,cells={anchor=west},column
sep=1ex}
]
\addplot+[blue, very thick, forget plot,only marks] 
plot[very thick, error bars/.cd, y dir=plus, y explicit]
table[x=x,y=y,y error expr=\thisrow{y-max}] {\differanser};
\addplot+[red, very thick, only marks,xticklabels=\empty] 
plot[very thick, error bars/.cd, y dir=minus, y explicit]
table[x=x,y=y,y error expr=\thisrow{y-min}] {\differanser};
\end{axis} 
\end{tikzpicture}
\caption{Graph that shows RMSE for Yahoo!.}
\label{fig:yahoo-rmse}
\end{figure}

\pgfplotstableread{
x y y-min y-max
{SVD} 0.732 0.001 0.003
{SVD++} 0.709 0 0.003
{Random} 1.663 0.003 0.004
{CAMF-CI}  0.751 0.002 0.001
{CAMF-C}  0.736 0.010 0.007
{DeepWalk + kNN} 0.818 0.004 0.004
{IS-UserBased} 0.794 0.083 0.119
{NMF} 0.790 0.002 0.002
}{\differanser}
\begin{figure}
\begin{tikzpicture}[scale=1] 
\begin{axis} [
title    = {Yahoo!: MAE},
width  = 0.5*\textwidth,
height = 8cm,
symbolic x coords={{SVD++},{SVD},{CAMF-C},{CAMF-CI},{NMF},{IS-UserBased},{DeepWalk + kNN},{Random}},
minor ytick={0.7,0.8,0.9,1,1.1,1.2,1.3,1.4},
yminorgrids,
xtick=data,
ticklabel style = {font=\tiny},
x tick label style={rotate=60,anchor=east},
legend style={at={(0.05,0.95)},anchor=north west,cells={anchor=west},column
sep=1ex}
]
\addplot+[blue, very thick, forget plot,only marks] 
plot[very thick, error bars/.cd, y dir=plus, y explicit]
table[x=x,y=y,y error expr=\thisrow{y-max}] {\differanser};
\addplot+[red, very thick, only marks,xticklabels=\empty] 
plot[very thick, error bars/.cd, y dir=minus, y explicit]
table[x=x,y=y,y error expr=\thisrow{y-min}] {\differanser};
\end{axis} 
\end{tikzpicture}
\caption{Graph that shows MAE for Yahoo!.}
\label{fig:yahoo-mae}
\end{figure}

\pgfplotstableread{
x y y-min y-max
{SVD} 0.390 0 0.001
{SVD++} 0.388 0.001 0.002
{Random} 0.227 0.002 0.002
{CAMF-CI}  0.390 0.001 0.002
{CAMF-C}  0.393 0.002 0.003
{DeepWalk + kNN} 0.290 0.002 0.002
{IS-UserBased} 0.217 0.009 0.021
{NMF} 0.377 0.001 0.001
{LightGCN} 0.148 0.002 0.001
{NGCF} 0.141 0.002 0.008
{KGAT} 0.139 0.001 0.003
}{\differanser}
\begin{figure}
\begin{tikzpicture}[scale=1] 
\begin{axis} [
title    = {Yahoo!: Precision@10},
width  = 0.5*\textwidth,
height = 8cm,
symbolic x coords={{CAMF-C},{CAMF-CI},{SVD},{SVD++},{NMF},{DeepWalk + kNN},{Random},{IS-UserBased},{LightGCN},{NGCF},{KGAT}},
minor ytick={0.0,0.1,0.2,0.3,0.4,0.5,0.6,0.7,0.8},
yminorgrids,
xtick=data,
ticklabel style = {font=\tiny},
x tick label style={rotate=60,anchor=east},
legend style={at={(0.05,0.95)},anchor=north west,cells={anchor=west},column
sep=1ex}
]
\addplot+[blue, very thick, forget plot,only marks] 
plot[very thick, error bars/.cd, y dir=plus, y explicit]
table[x=x,y=y,y error expr=\thisrow{y-max}] {\differanser};
\addplot+[red, very thick, only marks,xticklabels=\empty] 
plot[very thick, error bars/.cd, y dir=minus, y explicit]
table[x=x,y=y,y error expr=\thisrow{y-min}] {\differanser};
\end{axis} 
\end{tikzpicture}
\caption{Graph that shows Precision@10 for Yahoo!.}
\label{fig:yahoo-precision}
\end{figure}

\pgfplotstableread{
x y y-min y-max
{SVD} 0.894 0.002 0.003
{SVD++} 0.886 0.003 0.003
{Random} 0.481 0.006 0.005
{CAMF-CI}  0.892 0.003 0.002
{CAMF-C}  0.903 0.006 0.004
{DeepWalk + kNN} 0.877 0.004 0.002
{IS-UserBased} 0.910 0.038 0.023
{NMF} 0.853 0.004 0.002
{LightGCN} 0.342 0.003 0.003
{NGCF} 0.328 0.008 0.020 
{KGAT} 0.322 0.008 0.009
}{\differanser}
\begin{figure}
\begin{tikzpicture}[scale=1] 
\begin{axis} [
title    = {Yahoo!: Recall@10},
width  = 0.5*\textwidth,
height = 8cm,
symbolic x coords={{IS-UserBased},{CAMF-C},{SVD},{CAMF-CI},{SVD++},{DeepWalk + kNN},{NMF},{Random},{LightGCN},{NGCF},{KGAT}},
minor ytick={0.0,0.1,0.2,0.3,0.4,0.5,0.6,0.7,0.8},
yminorgrids,
xtick=data,
ticklabel style = {font=\tiny},
x tick label style={rotate=60,anchor=east},
legend style={at={(0.05,0.95)},anchor=north west,cells={anchor=west},column
sep=1ex}
]
\addplot+[blue, very thick, forget plot,only marks] 
plot[very thick, error bars/.cd, y dir=plus, y explicit]
table[x=x,y=y,y error expr=\thisrow{y-max}] {\differanser};
\addplot+[red, very thick, only marks,xticklabels=\empty] 
plot[very thick, error bars/.cd, y dir=minus, y explicit]
table[x=x,y=y,y error expr=\thisrow{y-min}] {\differanser};
\end{axis} 
\end{tikzpicture}
\caption{Graph that shows Recall@10 for Yahoo!.}
\label{fig:yahoo-recall}
\end{figure}

\pgfplotstableread{
x y y-min y-max
{SVD} 0.502 0.013 0.014
{SVD++} 0.507 0.008 0.010
{Random} 0.451 0.017 0.019
{CAMF-CI}  0.476 0.104 0.083
{CAMF-C}  0.492 0.055 0.070
{DeepWalk + kNN} 0.511 0.008 0.007
{IS-UserBased} 0.558 0.057 0.076
{NMF} 0.473 0.023 0.052
{LightGCN} 0.297 0.003 0.005
{NGCF}  0.316 0.006 0.018
{KGAT} 0.438 0.007 0.011
}{\differanser}
\begin{figure}
\begin{tikzpicture}[scale=1] 
\begin{axis} [
title  = {Yahoo!: NDCG@10},
width  = 0.5*\textwidth,
height = 8cm,
symbolic x coords={{IS-UserBased},{DeepWalk + kNN},{SVD++},{SVD},{CAMF-C},{CAMF-CI},{NMF},{Random},{KGAT},{NGCF},{LightGCN}},
minor ytick={0.0,0.1,0.2,0.3,0.4,0.5,0.6},
yminorgrids,
xtick=data,
ticklabel style = {font=\tiny},
x tick label style={rotate=60,anchor=east},
legend style={at={(0.05,0.95)},anchor=north west,cells={anchor=west},column
sep=1ex}
]
\addplot+[blue, very thick, forget plot,only marks] 
plot[very thick, error bars/.cd, y dir=plus, y explicit]
table[x=x,y=y,y error expr=\thisrow{y-max}] {\differanser};
\addplot+[red, very thick, only marks,xticklabels=\empty] 
plot[very thick, error bars/.cd, y dir=minus, y explicit]
table[x=x,y=y,y error expr=\thisrow{y-min}] {\differanser};
\end{axis}
\end{tikzpicture}
\caption{Graph that shows NDCG@10 for Yahoo!.}
\label{fig:yahoo-ndcg}
\end{figure}

\pgfplotstableread{
x y y-min y-max
{SVD} 0.0046 0.0005 0.0003
{SVD++} 0.0062 0.0006 0.0010
{Random} 0.00055 0.00012 0.00027
{CAMF-CI}  0.00007 0.00004 0.00005
{CAMF-C}  0.00013 0.00005 0.00007
{DeepWalk + kNN} 0.044 0.001 0.003
{IS-UserBased} 0.046 0.010 0.011
{NMF} 0.00075 0.00011 0.00019
{LightGCN} 0.000007 0.000001 0
{NGCF}  0.0000112 0.0000002 0.0000006
{KGAT} 0.0000018 0.0000005 0.00000004
}{\differanser}
\begin{figure}
\begin{tikzpicture}[scale=1] 
\begin{axis} [
title  = {Yahoo!: MAP@10},
width  = 0.5*\textwidth,
height = 8cm,
symbolic x coords={{IS-UserBased},{DeepWalk + kNN},{SVD++},{SVD},{NMF},{Random},{CAMF-C},{CAMF-CI},{NGCF},{LightGCN},{KGAT}},
minor ytick={0.0,0.1},
yminorgrids,
xtick=data,
ticklabel style = {font=\tiny},
x tick label style={rotate=60,anchor=east},
legend style={at={(0.05,0.95)},anchor=north west,cells={anchor=west},column
sep=1ex}
]
\addplot+[blue, very thick, forget plot,only marks] 
plot[very thick, error bars/.cd, y dir=plus, y explicit]
table[x=x,y=y,y error expr=\thisrow{y-max}] {\differanser};
\addplot+[red, very thick, only marks,xticklabels=\empty] 
plot[very thick, error bars/.cd, y dir=minus, y explicit]
table[x=x,y=y,y error expr=\thisrow{y-min}] {\differanser};
\end{axis}
\end{tikzpicture}
\caption{Graph that shows MAP@10 for Yahoo!.}
\label{fig:yahoo-map}
\end{figure}


\pgfplotstableread{
x y y-min y-max
{SVD} 0.543 0.001 0.001
{SVD++} 0.540 0.002 0.001
{Random} 0.308 0.003 0.003
{CAMF-CI} 0.543 0.002 0.001
{CAMF-C}  0.548 0.003 0.004
{DeepWalk + kNN} 0.436 0.002 0.002
{IS-UserBased} 0.350 0.013 0.025
{NMF} 0.523 0.001 0.001
{LightGCN} 0.206 0.002 0.002
{NGCF}  0.198 0.004 0.011
{KGAT} 0.194 0.004 0.004
}{\differanser}
\begin{figure}
\begin{tikzpicture}[scale=1] 
\begin{axis} [
title  = {Yahoo!: F1@10},
width  = 0.5*\textwidth,
height = 8cm,
symbolic x coords={{CAMF-C},{CAMF-CI},{SVD},{SVD++},{NMF},{DeepWalk + kNN},{IS-UserBased},{Random},{LightGCN},{NGCF},{KGAT}},
minor ytick={0.0,0.1},
yminorgrids,
xtick=data,
ticklabel style = {font=\tiny},
x tick label style={rotate=60,anchor=east},
legend style={at={(0.05,0.95)},anchor=north west,cells={anchor=west},column
sep=1ex}
]
\addplot+[blue, very thick, forget plot,only marks] 
plot[very thick, error bars/.cd, y dir=plus, y explicit]
table[x=x,y=y,y error expr=\thisrow{y-max}] {\differanser};
\addplot+[red, very thick, only marks,xticklabels=\empty] 
plot[very thick, error bars/.cd, y dir=minus, y explicit]
table[x=x,y=y,y error expr=\thisrow{y-min}] {\differanser};
\end{axis}
\end{tikzpicture}
\caption{Graph that shows F1@10 for Yahoo!.}
\label{fig:yahoo-f1}
\end{figure}

\subsubsection{Frappé}
The Frappé dataset is a little different, in that it is the only dataset containing implicit ratings, focusing solely on interactions.
As such, only results from methods that are able to produce a top-$N$ list are reported.
For Precision@10 we see that \textit{LightGCN} outperforms the others, and \textit{NGCF} shows worse performance on the metric compared to both \textit{LightGCN} and \textit{KGAT}.
For Recall@10 however, \textit{KGAT} shows the best performance by a substantial amount, whereas \textit{LightGCN} and \textit{NGCF} perform mostly evenly.
\textit{KGAT}'s overall performance on Precision@10 and Recall@10 lead to it performing the best on F1@10, where \textit{LightGCN} also outperforms \textit{NGCF}.
For these metrics, we see that the random baseline performs substantially worse than the state-of-the-art methods.
\textit{KGAT} outperforms the other methods except for random on NDCG@10, while performing poorly on MAP@10, which \textit{NGCF} performs the best on with the exception of random that surprisingly performs decently on the MAP@10 metric.
\pgfplotstableread{
x y y-min y-max
{KGAT} 0.174 0.012 0.010
{LightGCN} 0.086 0.009 0.012
{NGCF} 0.126 0.022 0.027
{Random}  0.46898 0.06749 0.06089
}{\differanser}
\begin{tikzpicture}[scale=1] 
\begin{axis} [
title    = {Frappe: NDCG@10},
width  = 0.5*\textwidth,
height = 8cm,
symbolic x coords={{KGAT},{LightGCN},{NGCF},{Random}},
minor ytick={0.9,1,1.1,1.2,1.3,1.4,1.5,1.6,1.7,1.8},
yminorgrids,
xtick=data,
ticklabel style = {font=\tiny},
x tick label style={rotate=60,anchor=east},
legend style={at={(0.05,0.95)},anchor=north west,cells={anchor=west},column
sep=1ex}
]
\addplot+[blue, very thick, forget plot,only marks] 
plot[very thick, error bars/.cd, y dir=plus, y explicit]
table[x=x,y=y,y error expr=\thisrow{y-max}] {\differanser};
\addplot+[red, very thick, only marks,xticklabels=\empty] 
plot[very thick, error bars/.cd, y dir=minus, y explicit]
table[x=x,y=y,y error expr=\thisrow{y-min}] {\differanser};
\end{axis} 
\end{tikzpicture}

\pgfplotstableread{
x y y-min y-max
{KGAT} 0.050 0.004 0.005
{LightGCN} 0.056 0.003 0.002
{NGCF} 0.047 0.008 0.007
{Random}  0.00202 0.0003 0.00029
}{\differanser}
\begin{tikzpicture}[scale=1] 
\begin{axis} [
title    = {Frappe: Precision@10},
width  = 0.5*\textwidth,
height = 8cm,
symbolic x coords={{KGAT},{LightGCN},{NGCF},{Random}},
minor ytick={0.9,1,1.1,1.2,1.3,1.4,1.5,1.6,1.7,1.8},
yminorgrids,
xtick=data,
ticklabel style = {font=\tiny},
x tick label style={rotate=60,anchor=east},
legend style={at={(0.05,0.95)},anchor=north west,cells={anchor=west},column
sep=1ex}
]
\addplot+[blue, very thick, forget plot,only marks] 
plot[very thick, error bars/.cd, y dir=plus, y explicit]
table[x=x,y=y,y error expr=\thisrow{y-max}] {\differanser};
\addplot+[red, very thick, only marks,xticklabels=\empty] 
plot[very thick, error bars/.cd, y dir=minus, y explicit]
table[x=x,y=y,y error expr=\thisrow{y-min}] {\differanser};
\end{axis} 
\end{tikzpicture}

\pgfplotstableread{
x y y-min y-max
{KGAT} 0.090 0.007 0.003
{LightGCN} 0.071 0.009 0.011
{NGCF} 0.075 0.009 0.010
{Random}  0.00227 0.00073 0.00148
}{\differanser}
\begin{tikzpicture}[scale=1] 
\begin{axis} [
title    = {Frappe: Recall@10},
width  = 0.5*\textwidth,
height = 8cm,
symbolic x coords={{KGAT},{LightGCN},{NGCF},{Random}},
minor ytick={0.9,1,1.1,1.2,1.3,1.4,1.5,1.6,1.7,1.8},
yminorgrids,
xtick=data,
ticklabel style = {font=\tiny},
x tick label style={rotate=60,anchor=east},
legend style={at={(0.05,0.95)},anchor=north west,cells={anchor=west},column
sep=1ex}
]
\addplot+[blue, very thick, forget plot,only marks] 
plot[very thick, error bars/.cd, y dir=plus, y explicit]
table[x=x,y=y,y error expr=\thisrow{y-max}] {\differanser};
\addplot+[red, very thick, only marks,xticklabels=\empty] 
plot[very thick, error bars/.cd, y dir=minus, y explicit]
table[x=x,y=y,y error expr=\thisrow{y-min}] {\differanser};
\end{axis} 
\end{tikzpicture}

\pgfplotstableread{
x y y-min y-max
{KGAT} 0.000001 0.000001 0.000003
{LightGCN} 0.000012 0.000002 0.000002
{NGCF} 0.000029 0.000009 0.000006
{Random}  0.00099 0.00051 0.00063
}{\differanser}
\begin{tikzpicture}[scale=1] 
\begin{axis} [
title    = {Frappe: MAP@10},
width  = 0.5*\textwidth,
height = 8cm,
symbolic x coords={{KGAT},{LightGCN},{NGCF},{Random}},
minor ytick={0.9,1,1.1,1.2,1.3,1.4,1.5,1.6,1.7,1.8},
yminorgrids,
xtick=data,
ticklabel style = {font=\tiny},
x tick label style={rotate=60,anchor=east},
legend style={at={(0.05,0.95)},anchor=north west,cells={anchor=west},column
sep=1ex}
]
\addplot+[blue, very thick, forget plot,only marks] 
plot[very thick, error bars/.cd, y dir=plus, y explicit]
table[x=x,y=y,y error expr=\thisrow{y-max}] {\differanser};
\addplot+[red, very thick, only marks,xticklabels=\empty] 
plot[very thick, error bars/.cd, y dir=minus, y explicit]
table[x=x,y=y,y error expr=\thisrow{y-min}] {\differanser};
\end{axis} 
\end{tikzpicture}
\subsection{Graphs test}

\begin{figure}[h]
\begin{tikzpicture}
\begin{axis} [
    title    = {MovieLens: RMSE and MAE},
    xbar=5pt,
    /pgf/bar width=5pt,
    y axis line style = { opacity = 0 },
    axis x line       = none,
    tickwidth         = 0pt,
    enlarge x limits  = 0.02,
    ytick=data,
    y=0.8cm,
    nodes near coords={\pgfmathprintnumber[fixed zerofill, precision=3]{\pgfplotspointmeta}},
    legend style={at={(0.5,-0.03)}, anchor=north,legend columns=-1},
    symbolic y coords = {SVD,SVD++,CAMF-C,CAMF-CI,IS-UserBased,NMF,DeepWalk+kNN,Random},
  ]
\addplot coordinates{(0.935,SVD) (0.917,SVD++) (0.931,CAMF-C) (0.978,CAMF-CI) (1.366,IS-UserBased) (0.962,NMF) (1.121,DeepWalk+kNN) (1.702,Random)};
\addplot coordinates{(0.737,SVD) (0.719,SVD++) (0.738,CAMF-C) (0.773,CAMF-CI) (1.041,IS-UserBased) (0.757,NMF) (0.869,DeepWalk+kNN) (1.391,Random)};
\legend{RMSE,MAE}
\end{axis}
\end{tikzpicture}
\caption{RMSE and MAE results for MovieLens dataset across the investigated methods}
\label{graph:MLRMSEMAE}
\end{figure}

\begin{figure}[h]
\begin{tikzpicture}
\begin{axis} [
    title    = {Yahoo: RMSE and MAE},
    xbar=5pt,
    /pgf/bar width=5pt,
    y axis line style = { opacity = 0 },
    axis x line       = none,
    tickwidth         = 0pt,
    enlarge x limits  = 0.02,
    ytick=data,
    y=0.8cm,
    nodes near coords={\pgfmathprintnumber[fixed zerofill, precision=3]{\pgfplotspointmeta}},
    legend style={at={(0.5,-0.03)}, anchor=north,legend columns=-1},
    symbolic y coords = {SVD,SVD++,CAMF-C,CAMF-CI,IS-UserBased,NMF,DeepWalk+kNN,Random},
  ]
\addplot coordinates{(1.013,SVD) (0.999,SVD++) (1.003,CAMF-C) (1.030,CAMF-CI) (1.234,IS-UserBased) (1.089,NMF) (1.195,DeepWalk+kNN) (1.996,Random)};
\addplot coordinates{(0.732,SVD) (0.709,SVD++) (0.736,CAMF-C) (0.751,CAMF-CI) (0.794,IS-UserBased) (0.790,NMF) (0.815,DeepWalk+kNN) (1.663,Random)};
\legend{RMSE,MAE}
\end{axis}
\end{tikzpicture}
\caption{RMSE and MAE results for Yahoo dataset across the investigated methods}
\label{graph:YahooRMSEMAE}
\end{figure}

\begin{figure}[h]
\begin{tikzpicture}
\begin{axis} [
    title    = {LDOS-CoMoDa: RMSE and MAE},
    xbar=5pt,
    /pgf/bar width=5pt,
    y axis line style = { opacity = 0 },
    axis x line       = none,
    tickwidth         = 0pt,
    enlarge x limits  = 0.02,
    ytick=data,
    y=0.8cm,
    nodes near coords={\pgfmathprintnumber[fixed zerofill, precision=3]{\pgfplotspointmeta}},
    legend style={at={(0.5,-0.03)}, anchor=north,legend columns=-1},
    symbolic y coords = {SVD,SVD++,CAMF-C,CAMF-CI,IS-UserBased,NMF,DeepWalk+kNN,Random},
  ]
\addplot coordinates{(1.019,SVD) (1.020,SVD++) (0.669,CAMF-C) (0.824,CAMF-CI) (1.118,IS-UserBased) (1.173,NMF) (1.211,DeepWalk+kNN) (1.763,Random)};
\addplot coordinates{(0.826,SVD) (0.828,SVD++) (0.476,CAMF-C) (0.625,CAMF-CI) (0.791,IS-UserBased) (0.932,NMF) (1.207,DeepWalk+kNN) (1.454,Random)};
\legend{RMSE,MAE}
\end{axis}
\end{tikzpicture}
\caption{RMSE and MAE results for LDOS-CoMoDa dataset across the investigated methods}
\label{graph:CoMoDaRMSEMAE}
\end{figure}

\begin{figure}[h]
\begin{tikzpicture}
\begin{axis} [
    title    = {MovieLens: Precision@10, Recall@10, F1},
    xbar=5pt,
    /pgf/bar width=5pt,
    y axis line style = { opacity = 0 },
    axis x line       = none,
    tickwidth         = 0pt,
    enlarge x limits  = 0.02,
    ytick=data,
    y=1.2cm,
    nodes near coords={\pgfmathprintnumber[fixed zerofill, precision=3]{\pgfplotspointmeta}},
    legend style={at={(0.5,-0.03)}, anchor=north,legend columns=-1},
    symbolic y coords = {SVD,SVD++,CAMF-C,CAMF-CI,IS-UserBased,NMF,DeepWalk+kNN,Random,LightGCN,KGAT,NGCF},
  ]
\addplot coordinates{(0.719,SVD) (0.721,SVD++) (0.726,CAMF-C) (0.713,CAMF-CI) (0.344,IS-UserBased) (0.702,NMF) (0.679,DeepWalk+kNN) (0.538,Random) (0.3299,LightGCN) (0.1320,KGAT) (0.3386,NGCF)};
\addplot coordinates{(0.633,SVD) (0.631,SVD++) (0.645,CAMF-C) (0.629,CAMF-CI) (0.785,IS-UserBased) (0.606,NMF) (0.659,DeepWalk+kNN) (0.415,Random) (0.2222,LightGCN) (0.7072,KGAT) (0.2228,NGCF)};
\addplot coordinates{(0.673,SVD) (0.673,SVD++) (0.683,CAMF-C) (0.668,CAMF-CI) (0.478,IS-UserBased) (0.650,NMF) (0.669,DeepWalk+kNN) (0.469,Random) (0.265514,LightGCN) (0.222441,KGAT) (0.268757,NGCF)};
\legend{Precision,Recall,F-1}
\end{axis}
\end{tikzpicture}
\caption{Precision, recall and F1 results for MovieLens dataset across the investigated methods}
\label{graph:MLPrecRecF1}
\end{figure}

\begin{figure}[h]
\begin{tikzpicture}
\begin{axis} [
    title    = {Yahoo: Precision@10, Recall@10, F1},
    xbar=5pt,
    /pgf/bar width=5pt,
    y axis line style = { opacity = 0 },
    axis x line       = none,
    tickwidth         = 0pt,
    enlarge x limits  = 0.02,
    ytick=data,
    y=1.2cm,
    nodes near coords={\pgfmathprintnumber[fixed zerofill, precision=3]{\pgfplotspointmeta}},
    legend style={at={(0.5,-0.03)}, anchor=north,legend columns=-1},
    symbolic y coords = {SVD,SVD++,CAMF-C,CAMF-CI,IS-UserBased,NMF,DeepWalk+kNN,Random,LightGCN,KGAT,NGCF},
  ]
\addplot coordinates{(0.390,SVD) (0.388,SVD++) (0.394,CAMF-C) (0.390,CAMF-CI) (0.217,IS-UserBased) (0.377,NMF) (0.291,DeepWalk+kNN) (0.227,Random) (0.3544,LightGCN) (0.0342,KGAT) (0.1414,NGCF)};
\addplot coordinates{(0.894,SVD) (0.886,SVD++) (0.903,CAMF-C) (0.892,CAMF-CI) (0.910,IS-UserBased) (0.853,NMF) (0.878,DeepWalk+kNN) (0.481,Random) (0.1548,LightGCN) (0.6840,KGAT) (0.3280,NGCF)};
\addplot coordinates{(0.543,SVD) (0.540,SVD++) (0.549,CAMF-C) (0.543,CAMF-CI) (0.350,IS-UserBased) (0.523,NMF) (0.437,DeepWalk+kNN) (0.308,Random) (0.2155,LightGCN) (0.0652,KGAT) (0.1976,NGCF)};
\legend{Precision,Recall,F-1}
\end{axis}
\end{tikzpicture}
\caption{Precision, recall and F1 results for Yahoo dataset across the investigated methods}
\label{graph:YahooPrecRecF1}
\end{figure}

\begin{figure}[h]
\begin{tikzpicture}
\begin{axis} [
    title    = {LDOS-CoMoDa: Precision@10, Recall@10, F1},
    xbar=5pt,
    /pgf/bar width=5pt,
    y axis line style = { opacity = 0 },
    axis x line       = none,
    tickwidth         = 0pt,
    enlarge x limits  = 0.02,
    ytick=data,
    y=1.2cm,
    nodes near coords={\pgfmathprintnumber[fixed zerofill, precision=3]{\pgfplotspointmeta}},
    legend style={at={(0.5,-0.03)}, anchor=north,legend columns=-1},
    symbolic y coords = {SVD,SVD++,CAMF-C,CAMF-CI,IS-UserBased,NMF,DeepWalk+kNN,Random,LightGCN,KGAT,NGCF},
  ]
\addplot coordinates{(0.482,SVD) (0.480,SVD++) (0.487,CAMF-C) (0.483,CAMF-CI) (0.161,IS-UserBased) (0.438,NMF) (0.361,DeepWalk+kNN) (0.305,Random) (0.08665008,LightGCN) (0.20284,KGAT) (0.5416,NGCF)};
\addplot coordinates{(0.856,SVD) (0.853,SVD++) (0.834,CAMF-C) (0.823,CAMF-CI) (0.919,IS-UserBased) (0.739,NMF) (0.884,DeepWalk+kNN) (0.450,Random) (0.067787792,LightGCN) (0.220396,KGAT) (0.41,NGCF)};
\addplot coordinates{(0.617,SVD) (0.614,SVD++) (0.615,CAMF-C) (0.609,CAMF-CI) (0.274,IS-UserBased) (0.550,NMF) (0.512,DeepWalk+kNN) (0.364,Random) (0.0760671,LightGCN) (0.211254,KGAT) (0.466700,NGCF)};
\legend{Precision,Recall,F-1}
\end{axis}
\end{tikzpicture}
\caption{Precision, recall and F1 results for LDOS-CoMoDa dataset across the investigated methods}
\label{graph:CoMoDaPrecRecF1}
\end{figure}

\begin{figure}[h]
\begin{tikzpicture}
\begin{axis} [
    title    = {MAP@10 for the datasets},
    xbar=5pt,
    /pgf/bar width=5pt,
    y axis line style = { opacity = 0 },
    axis x line       = none,
    tickwidth         = 0pt,
    enlarge x limits  = 0.02,
    ytick=data,
    y=1.2cm,
    nodes near coords={\pgfmathprintnumber[fixed zerofill, precision=4]{\pgfplotspointmeta}},
    legend style={at={(0.5,-0.03)}, anchor=north,legend columns=-1},
    symbolic y coords = {SVD,SVD++,CAMF-C,CAMF-CI,IS-UserBased,NMF,DeepWalk+kNN,Random,LightGCN,KGAT,NGCF},
  ]
\addplot coordinates{(0.0162465951248284,SVD) (0.0230514217391177,SVD++) (0.00249,CAMF-C) (0.000707,CAMF-CI) (0.0120,IS-UserBased) (0.00470921880091866,NMF) (0.0302,DeepWalk+kNN) (0.00443467504522534,Random) (0.0000892961875081354,LightGCN) (0.0000332,KGAT) (0.0002538,NGCF)}; % MovieLens
\addplot coordinates{(0.00459364446058646,SVD) (0.00622607894470324,SVD++) (0.00013,CAMF-C) (0.000069,CAMF-CI) (0.046,IS-UserBased) (0.000747900814877045,NMF) (0.0439,DeepWalk+kNN) (0.000546288347257189,Random) (0.00000693,LightGCN) (0.0000017781423399345,KGAT) (0.0000112,NGCF)}; % Yahoo
\addplot coordinates{(0.022,SVD) (0.02,SVD++) (0.0083,CAMF-C) (0.0103,CAMF-CI) (0.007,IS-UserBased) (0.0136,NMF) (0.01127,DeepWalk+kNN) (0.0126990696917064,Random) (0.00337181254,LightGCN) (0.00035,KGAT) (0.020748,NGCF)}; % CoMoDa
\legend{MovieLens,Yahoo,LDOS-CoMoDa}
\end{axis}
\end{tikzpicture}
\caption{MAP@10 results for the dataset across the investigated methods}
\label{graph:MAP10}
\end{figure}

\begin{figure}[h]
\begin{tikzpicture}
\begin{axis} [
    title    = {NDCG for the datasets},
    xbar=5pt,
    /pgf/bar width=5pt,
    y axis line style = { opacity = 0 },
    axis x line       = none,
    tickwidth         = 0pt,
    enlarge x limits  = 0.02,
    ytick=data,
    y=1.2cm,
    nodes near coords={\pgfmathprintnumber[fixed zerofill, precision=3]{\pgfplotspointmeta}},
    legend style={at={(0.5,-0.03)}, anchor=north,legend columns=-1},
    symbolic y coords = {SVD,SVD++,CAMF-C,CAMF-CI,IS-UserBased,NMF,DeepWalk+kNN,Random,LightGCN,KGAT,NGCF},
  ]
\addplot coordinates{(0.507788476243958,SVD) (0.540462250066491,SVD++) (0.3752,CAMF-C) (0.4025,CAMF-CI) (0.270,IS-UserBased) (0.441025641977599,NMF) (0.5201,DeepWalk+kNN) (0.475627993916583,Random) (0.3939,LightGCN) (0.6497,KGAT) (0.4204,NGCF)}; % MovieLens
\addplot coordinates{(0.501928001717915,SVD) (0.507223962441533,SVD++) (0.492,CAMF-C) (0.476,CAMF-CI) (0.558,IS-UserBased) (0.472880077638768,NMF) (0.5089,DeepWalk+kNN) (0.451040991988228,Random) (0.3090,LightGCN) (0.4618,KGAT) (0.3164,NGCF)}; % Yahoo
\addplot coordinates{(0.502,SVD) (0.478,SVD++) (0.477,CAMF-C) (0.469,CAMF-CI) (0.565,IS-UserBased) (0.5385,NMF) (0.5828,DeepWalk+kNN) (0.484100128099788,Random) (0.103,LightGCN) (0.21176,KGAT) (0.6522,NGCF) }; % CoMoDa
\legend{MovieLens,Yahoo,LDOS-CoMoDa}
\end{axis}
\end{tikzpicture}
\caption{NDCG@10 results for the dataset across the investigated methods}
\label{graph:NDCG10}
\end{figure}


\section{Discussion}\label{sec:discussion}
The following section will investigate the results acquired in \autoref{sec:experiments}.
The reported values for the different metrics will be discussed in order to gain an understanding of why the different methods perform as reported, and which results are interesting.
The comparability between different methods and datasets will also be discussed, as well as what information can be inferred in relation to the datasets.
Finally, the usefulness of context for recommendation will be examined, as well as the performance of graph-based methods.

\subsection{Experimental results discussion}
In the following subsection, we discuss some of the metrics used in the experiments to evaluate the methods.
The main goal is to answer the following question:
\begin{itemize}
    \item Which metrics are the most interesting for comparison?
    \item Are the experimental results comparable?
\end{itemize}

\subsubsection{The value of RMSE and MAE}
Two of the metrics that were used for evaluating the performance of the methods that compute rating predictions were RMSE and MAE.
One of the things to be aware of with these metrics is to be careful when using them for determining the value of the different context variables.
For example, when a user is happy, it can be expected that every movie will be rated higher across the board which may result in better or worse RMSE and MAE.
The issue is that it will probably not change the top-$N$ list of recommended items, so instead one should use metrics such as Precision@N, Recall@N, or F1@N to determine the value of the context variables.
The interesting context variables to examine would be the ones that affect individual items or categories of items differently.
That would result in a change in the order of preference when doing recommendations.
RMSE and MAE may not be able to express this but a better or worse Precision@N, Recall@N, F1@N would be able to capture this.
However the RMSE and MAE metrics can still be used when optimizing an algorithm as they give an indication of how far the predicted ratings are from the actual ratings.
In terms of usage when evaluating if a method is useful in real-world applications, other metrics that look at the quality of the top-$N$ predictions are more useful as those recommendations are what the user will be interacting with.

\subsubsection{The trade-off between precision and recall and the F1-measure}\label{subsub:precrectradeoff}
A trade-off between precision and recall is expected as mentioned in \autoref{sec:evaluationmetrics}.
This is due to their definitions: precision uses the length of the proposed top-$N$ list as the denominator while recall uses the number of total relevant items.
This means that if the amount of relevant items increases, the algorithm is more likely to hit a relevant item in the recommendation, thus improving overall precision.
However, an increasing number of relevant items penalizes recall, thus there is a trade-off.
As such, simply looking at one of these metrics can be misleading, as the results can be influenced in this manner.
We thus use F1 as a metric to combine these metrics and generate more comparable results between the methods.
\autoref{graph:MLPrecRecF1}, \autoref{graph:YahooPrecRecF1} and \autoref{graph:CoMoDaPrecRecF1} report the results for all of these methods.
Because of the trade-off, F1 is the most interesting metric for comparison.

\subsubsection{Comparison of MAP@N and NDCG}
MAP@N and NDCG serve a similar purpose, as defined in \autoref{sec:evaluationmetrics}.
For the predicted top-$N$ list we are not just interested in hitting relevant items, but also frontloading the list with these relevant items.
These metrics serve to measure how well a method accomplishes this.
As is evident from \autoref{graph:MAP10}, the MAP@N metric is very punishing, culminating in results of very small values approaching 0.
NDCG on the other hand, \autoref{graph:NDCG10} tends to report more easily relatable results between 0 and 1 as an indicator.
The reason for this discrepancy is in the way the calculation takes place.
NDCG looks at whether or not items are frontloaded in a top-$N$ list whenever the top-$N$ list contains relevant items, unlike MAP@N, for which any list of top-$N$ contributes to the final results, rather than just those containing relevant items.
A shortcoming of the NDCG approach is that it is easier for baselines such as random to perform well on this metric.
Since it counts frontloaded items only in lists that contain relevant items, it can get lucky and produce a list for a user containing frontloaded items and thus achieve a good result.
This is why it achieves comparable results to other methods that would be expected to outperform a random choice on this metric.

\subsubsection{Are the results comparable?}
The algorithms investigated were split into two categories: Top-$N$ recommendations and rating prediction.
While some of the algorithms (SVD, SVD++) focus on rating prediction, most of the algorithms focus on generating a Top-$N$ list of recommendations for a given user.
This means that some of the metrics used for the various methods may not be related to their original goals, such as using the SVD algorithms for top-$N$ recommendations.
However, as we wanted the focus of this paper to be a broad comparison of state-of-the-art methods, we concluded that it would be beneficial to include all algorithms in all metrics.
For the rating prediction methods, this was done by generating a predicted rating for all items for a given user, and sorting it by estimated rating and presenting the highest rated values in this list to the user.
This approach does, however, have a minor flaw: When doing top-$N$ recommendation for a user, more than $N$ items may have the same estimated rating, in which case only some of them will be given to the user.
This can influence the results, since we do not know whether these $N$ items are the best, or if the best results are actually in the remainder of the list.
None the less, the methods that are taken out of their initial purpose have actually shown good results in other metrics, such as SVD performing the best in precision on the MovieLens dataset, even though it is meant for rating prediction.

\subsection{Graph based methods, context-aware methods and utilizing side information}
In the following subsection, we will try to answer the following questions: 
\begin{itemize}
    \item How do the context-aware methods utilize context?
    \item Did context and side information matter?
    \item Are timestamps good contextual variables?
    \item How do graph-based methods compare to non-graph based methods?
\end{itemize}

\subsubsection{How do the context-aware methods utilize context?} 
\textit{IS-UserBased} utilizes context by performing item splitting as described in \autoref{method:IS-UserBased-Graph}.
While this item splitting approach is a simple method to model the context for items, it generates more sparsity in the data.
Before performing the item splitting, two users could be connected through a common item, but this connection will be broken in case they did not rate the item in the same context.
For example, if user $u_1$ rated the item $i_1$ in the context \textit{(morning, monday)} and user $u_2$ rated the same item in context \textit{(evening, monday)}, the item splitting will result in them having interacted with two different fictional items.
To combat this sparsity, we decided to look into adding side-information to the items to re-establish this connection.
One way to do this is by adding side-information about the movie, such as director, actors, or genre as we have seen in the KGAT method.
We theorize that adding this side-information could help improve the recommendations generated, since it allows re-connecting these fictional items through multiple relations.
Additionally, utilizing this along a multi-hop approach as seen in \textit{KGAT}, we may be able to infer that a user has a preference for a given genre or actor in a given context, rather than just a preference for a given movie.\\
\textit{CAMF} on the other hand does not experience the sparsity problems seen in the \textit{IS-UserBased} approach.
In \textit{CAMF-C} a parameter is instead introduced for each contextual value, and for \textit{CAMF-CI} a parameter is introduced for each contextual value and item pair as mentioned in \autoref{subsec:camf}.
A potential problem with the amount of model parameters in \textit{CAMF-C}, however, is the training data needed to contain a sufficient amount of recorded interactions within the different contexts. 
Otherwise the contextual parameters will not be learned properly.
This is especially prominent in the \textit{CAMF-CI} approach as the amount of model parameters scales with the amount of items in the dataset.

\subsubsection{Did context and side information matter?}
As reported in \autoref{subsec:resultsofexperiment}, \textit{IS-UserBased} showed the highest recall score for the LDOS-CoMoDa and MovieLens datasets, which are the ones with contextual information.
However, at the same time it had the second worst precision, which is most likely due to the precision/recall trade-off discussed in \autoref{subsub:precrectradeoff}.
Due to the very low precision score, it ends up with the third worst F1 score of the 11 methods, only surpassing LightGCN and KGAT.
To do a more complete comparison, the method was also run using side-information for item splitting for the Yahoo dataset, in which case it once again had the best recall score, and the third lowest precision.
In this case, it outperformed NGCF and KGAT for the precision score.
However, using side-information instead of context it ends up outperforming both NGCF, KGAT, LightGCN and Random in F1 scores.\\
Looking at the other context-based methods, \textit{CAMF-C} and \textit{CAMF-CI}, they end up with respectively a second and forth place in the F1 metric on the LDOS-CoMoDa dataset, only being outperformed by SVD and SVD++.\\
For MovieLens, \textit{CAMF-C} gets a first place on F1, and \textit{CAMF-CI} on a fifth place, being outperformed by \textit{SVD, SVD++} and \textit{DeepWalk}.\\
Looking at the results in general, the \textit{IS-Userbased} method does well on recall, and \textit{CAMF-C(I)} does well on F1 scores, which bodes well for using context and side-information.\\
The \textit{MAP@10} results are a slightly more inconclusive on the other hand, we see that the \textit{IS-UserBased} method achieves the best results on the Yahoo dataset, while the \textit{CAMF-C(I)} methods perform very poorly on the same dataset.
For the two other datasets, both methods give very mediocre results.
Finally, for the NDCG metric, we see that \textit{IS-Userbased} performs very well on the Yahoo dataset, mediocre on LDOS-CoMoDa and poorly on MovieLens.\\\\
\\\\
The Yahoo dataset does not contain timestamps, nor any other contextual information.
It includes age of the user and their gender, which are considered to be user attributes rather than contextual information.
As such, this dataset only contains side information.
\autoref{subsec:desc-of-datasets} described how this side information was used as contextual variables for context-aware methods.
This could influence the results on this datasets for the methods utilizing context.
However, on the results we see generally poorer performance all methods on the Yahoo dataset with the exception of \textit{IS-UserBased}, but not substantially.
For the context-aware methods, we do not see a decrease in performance compared to the other methods, indicating that utilizing this side-information as context does not negatively impact the results.
An explanation for the worse performance is likely the increased sparsity of the Yahoo dataset, making better results more difficult to achieve.\\
To check whether using the side-information in contextual methods made a difference, we re-ran the the \textit{CAMF-C(I)} and \textit{IS-UserBased} methods on the MovieLens dataset using respectively contextual (timestamp) and side-information (age and gender) for the models.
\begin{table*}[!htp]\centering
    \caption{Results for using respectively context and side-information for the MovieLens dataset, highlighted numbers are the best results.}\label{tab:movielenscontextsideinfo}
    \scriptsize
    \begin{tabular}{lrrrrrrrr}\toprule
    &\textbf{RMSE} &\textbf{MAE} &\textbf{Precision@10} &\textbf{Recall@10} &\textbf{MAP@10} &\textbf{NDCG} &\textbf{F1@10} \\\cmidrule{2-8}
    \textbf{CAMF-C Context} &0.931 &0.738 &\textbf{0.726} &0.645 &0.0025 &0.375 &\textbf{0.683} \\\cmidrule{1-8}
    \textbf{CAMF-CI Context} &0.978 &0.773 &0.713 &0.628 &0.0007 &0.403 &0.668 \\\cmidrule{1-8}
    \textbf{CAMF-C Side-info} &\textbf{0.930} &\textbf{0.736} &0.726 &0.646 &0.0026 &0.376 &0.684 \\\cmidrule{1-8}
    \textbf{CAMF-CI Side-info} &0.952 &0.753 &0.714 &0.630 &0.0005 &0.403 &0.669 \\\cmidrule{1-8}
    \textbf{IS-UserBased Side-info} &1.329 &0.976 &0.536 &\textbf{0.789} &\textbf{0.0220} &\textbf{0.538} &0.638 \\\cmidrule{1-8}
    \textbf{IS-UserBased Context} &1.366 &1.041 &0.344 &0.785 &0.0120 &0.270 &0.478 \\\cmidrule{1-8}
    \bottomrule
    \end{tabular}
\end{table*}
As seen on \autoref{tab:movielenscontextsideinfo}, the models generally perform better when provided with side-information rather than contextual information.
\\\\
To conclude, it is inconclusive whether adding context or side-information improves the methods in general, but we have observed that it does well on some metrics in some datasets, while performing poorly on others.
These varying performances amongst datasets can be due to a variety of factors, including the size and density of the datasets, the quality of the contextual variables used and how the contextual variables have been discretized.\\

\subsubsection{Discretization of context}
For the datasets we chose to focus on certain contextual variables or side information categories.
In \autoref{subsec:experimentprotocol} we defined the side information categories as well as the contextual variables used.
We chose to discretize the timestamp for the rating in the MovieLens dataset, leading to two different variables, time of day and day of the week.
This discretization could have been done differently, such as extracting seasonal data to look at, for example, summer and winter as context, or simply including less discretized categories, such as only the weekend and week for day of the week, or larger intervals for the time of the day.
For future work, it may be interesting to look into other ways to discretize the timestamp in ways such as season, holidays, or whether it is weekend or not, to confirm whether this has a significant impact on the quality of recommendations.

\subsubsection{Timestamp as a context variable}
A timestamp is often the only available context variable in larger datasets that can be extracted and used.
However, an assumption made when doing so is that the rating was given by the user at the time of finishing consumption of the item, if we wish to use it for recommendations.
For example, if a user just signed up for a service in the summer and starts rating Christmas movies, the system might assume that the user has a preference for watching Christmas movies in the summer.
To combat this, the collection of data will have to keep track of both when the rating was made, and when the item was consumed.
This information is not available in any datasets that we are aware of, meaning that the assumption that a rating is made soon after consumption is required.
As described in \autoref{subsec:desc-of-datasets}, \cite{COMODA2013} defines that some of the most influential context variables are the dominant emotions during the watching of the movie as well as the final emotion felt.
As such, we can infer that ratings provided shortly after watching do a better job of reflecting the contextual influence on the rating.
However, this cannot be guaranteed as some users might rate a batch of items in quick succession, or even rate a single item after a substantial amount of time has passed since consumption.
This means the data could lose some of the contextual impact if timestamp is the only context variable that is available or inferable.
\autoref{tbl:contextprotocol} defines the context variables used for the experiments, in which it is shown that the MovieLens dataset only has a timestamp for context, meaning this is at least a factor for the results on this dataset. 

\subsubsection{The performance of graph based methods} 
The results were shortly examined in \autoref{subsec:desc-of-datasets}, where it was noted that the graph based methods generally did not perform well.
For precision and F1, only NGCF achieved comparable results on a single dataset, which was LDOS-CoMoDa.
IS-UserBased and KGAT generally showed good performance on recall.
In terms of frontloading the top-$N$ list, KGAT, NGCF and IS-UserBased show the best results on MovieLens, LDOS-CoMoDa and Yahoo respectively.
While this does not indicate that the graph-based methods overall outperform the other methods, they show decent performance in terms of frontloading items.
A reason for the poorer performance of graph based methods might be the sizes of the datasets.
We suspect this because the datasets that were used for the development of these methods are generally a lot larger.
For example, the smallest dataset used in the \textit{LightGCN}\cite{LightGCN} paper has about a million interactions.
Therefore it could be interesting to run the experiment on a larger dataset to see if the graph-based methods achieve better performance.

\section{Conclusion}\label{sec:conclusion}
Throughout this paper, we have looked into various metrics used for evaluating recommender systems. What we found was that there is no consistent set of metrics used for evaluation, but the most commonly used are NDCG and Recall.
We found that utilizing bipartite graphs is a popular and intuitive way to handle information in the system, but in the investigated metrics, they are often outperformed by traditional methods such as SVD.
The cause of this is speculative but may be because methods like SVD make use of explicit rating information, where the investigated graph-based methods use implicit ratings.
\\\\
From the knowledge acquired throughout this paper, we propose that instead of only looking into context, as information about the interaction, it could be beneficial to expand the model to include side-information about the items or users in addition to the context.
This could allow the system to identify otherwise unseen trends, such as a user liking a specific movie genre in a certain context, rather than just looking at which specific movies they like.
Additionally, adding information about the users may help connect like-minded users in scenarios where the current user has not yet rated anything in their given context.

\section{Future work}\label{sec:futurework}
In the following section, we will define potential future work based on the information acquired.
In \Cref{sec:discussion}, we highlighted multiple possible directions for future work.

\subsection{Side-information to improve density}
The \textit{IS-UserBased} method shows potential for getting good results, but due to its item splitting approach, it creates a greater level of sparsity than already exists in the datasets.
To combat this sparsity, we decided to look into adding side-information to the items to re-establish this connection.
One way to do this is by adding side-information about the movie, such as director, actors, or genre as we have seen in the \textit{KGAT} method.
We theorize that adding this side-information could help improve the recommendations generated, since it allows re-connecting these fictional items through multiple relations.
Additionally, utilizing this along a multi-hop approach as seen in \textit{KGAT}, we may be able to infer that a user has a preference for a given genre or actor in a given context, rather than just a preference for a given movie.\\
For future work on this method, adding side-information to items or users may help alleviate this sparsity and reconnect the fictional items it creates.
Adding this side-information to other methods such as \textit{NGCF} or \textit{LightGCN} may also be interesting to look into.

\subsection{Adding context to existing methods}
It could be interesting to look at ways to add context to \textit{LightGCN} or \textit{NGCF}.
As they generally performed worse, this might improve their performance for various metrics.
One way of doing this could be by employing methods to represent context in a format usable by the initial embeddings for these methods, and investigating results on the contextual datasets.

\subsection{Data discretization}
We defined certain categories for discretization in \Cref{subsub:datadiscretization}.
A potential way to extend the results would be to employ different categories and examine changes in results.

\subsection{Collect better context datasets}
One of the major problems when working with CARS is the lack of datasets that have both useful contextual variables and a reasonable amount of interactions.
Currently, the most appropriate datasets for CARS are CoMoDa with 2,296 ratings and 12 contextual variables, or MovieLens 100k which has 100,000 ratings but only a timestamp as contextual information.
In order to further investigate CARS, it is necessary to gather more information with useful contextual information.

\subsection{Explicit and implicit rating investigation}
In \Cref{subsub:graphperformance} we speculate that the worse performance of \textit{LightGCN}, \textit{NGCF} and \textit{KGAT} is due to utilizing implicit ratings.
A more in depth investigation and comparison of utilizing implicit and explicit ratings would be necessary to confirm this.

\subsection{Investigating more CARS methods}
Another extension is to simply involve more CARS methods in the experiment to gain a better understanding of their performance in relation to methods that do not utilize context.


%----------------------------------------------------------------------------------------
%	REFERENCE LIST
%----------------------------------------------------------------------------------------

\printbibliography[heading=bibintoc]
\label{bib:mybiblio}

%----------------------------------------------------------------------------------------

%----------------------------------------------------------------------------------------
%	APPENDIX
%----------------------------------------------------------------------------------------
\newpage
\onecolumn
\appendix
\section{\\Results for methods and datasets}\label{app:tables}
This appendix contains all the average results across 5 folds for the investigated methods and datasets.
The best values are indicated with a bold number, the worst are indicated by being underlined.


\begin{table}[!htp]\centering
\caption{Frappe results}\label{tab:frappetable}
\scriptsize
\begin{tabular}{lrrrrrr}\toprule
&\textbf{Precision@10} &\textbf{Recall@10} &\textbf{MAP@10} &\textbf{NDCG} &\textbf{F1} \\\cmidrule{2-6}
\textbf{Random} &\ul{0.0005} &\ul{0.0004} &\textbf{0.000095} &\textbf{0.354} &\ul{0.0004} \\\cmidrule{1-6}
\textbf{LightGCN} &\textbf{0.056} &0.071 &0.000012 &\ul{0.086} &0.062 \\\cmidrule{1-6}
\textbf{NGCF} &0.047 &0.075 &0.000029 &0.126 &0.058 \\\midrule
\textbf{KGAT} &0.050 &\textbf{0.090} &\ul{0.000001} &0.174 &\textbf{0.064} \\
\bottomrule
\end{tabular}
\end{table}

\begin{table}[!htp]\centering
\caption{LDOS-CoMoDa results}\label{tab:comodatable}
\scriptsize
\begin{tabular}{lrrrrrrrr}\toprule
&\textbf{RMSE} &\textbf{MAE} &\textbf{Precision@10} &\textbf{Recall@10} &\textbf{MAP@10} &\textbf{NDCG} &\textbf{F1} \\\cmidrule{2-8}
\textbf{SVD} &1.020 &0.826 &0.482 &0.856 &\textbf{0.021743} &0.502 &\textbf{0.616} \\\cmidrule{1-8}
\textbf{SVD++} &1.020 &0.828 &0.480 &0.853 &0.020942 &0.487 &0.614 \\\cmidrule{1-8}
\textbf{CAMF-C} &\textbf{0.669} &\textbf{0.475} &0.487 &0.834 &0.008216 &0.477 &0.615 \\\cmidrule{1-8}
\textbf{CAMF-CI} &0.824 &0.625 &0.483 &0.823 &0.010292 &0.469 &0.609 \\\cmidrule{1-8}
\textbf{IS-UserBased} &0.944 &0.629 &0.160 &\textbf{0.905} &0.011000 &0.503 &0.272 \\\cmidrule{1-8}
\textbf{NMF} &1.173 &0.932 &0.438 &0.739 &0.012189 &0.531 &0.550 \\\cmidrule{1-8}
\textbf{DeepWalk + kNN} &1.216 &0.887 &0.369 &0.896 &0.010796 &0.597 &0.523 \\\cmidrule{1-8}
\textbf{Random} &\ul{1.763} &\ul{1.454} &0.305 &0.450 &0.012699 &0.484 &0.364 \\\cmidrule{1-8}
\textbf{LightGCN} & & &\ul{0.082} &\ul{0.067} &0.003121 &\ul{0.096} &0.074 \\\cmidrule{1-1}\cmidrule{1-8}
\textbf{NGCF} & & &\textbf{0.536} &0.411 &0.020518 &\textbf{0.667} &0.465 \\\midrule
\textbf{KGAT} & & &0.123 &0.074 &\ul{0.000021} &0.160 &\ul{0.093} \\
\bottomrule
\end{tabular}
\end{table}

\begin{table}[!htp]\centering
\caption{Yahoo! results}\label{tab:yahootable}
\scriptsize
\begin{tabular}{lrrrrrrrr}\toprule
&\textbf{RMSE} &\textbf{MAE} &\textbf{Precision@10} &\textbf{Recall@10} &\textbf{MAP@10} &\textbf{NDCG} &\textbf{F1} \\\cmidrule{2-8}
\textbf{SVD} &1.013 &0.732 &0.390 &0.894 &0.004594 &0.502 &0.543 \\\cmidrule{1-8}
\textbf{SVD++} &\textbf{0.999} &\textbf{0.709} &0.388 &0.886 &0.006226 &0.507 &0.540 \\\cmidrule{1-8}
\textbf{CAMF-C} &1.003 &0.736 &\textbf{0.393} &0.903 &0.000133 &0.492 &\textbf{0.548} \\\cmidrule{1-8}
\textbf{CAMF-CI} &1.030 &0.751 &0.390 &0.892 &0.000068 &0.476 &0.543 \\\cmidrule{1-8}
\textbf{IS-UserBased} &1.235 &0.794 &0.217 &\textbf{0.910} &\textbf{0.046000} &\textbf{0.558} &0.350 \\\cmidrule{1-8}
\textbf{NMF} &1.089 &0.790 &0.377 &0.853 &0.000748 &0.473 &0.523 \\\cmidrule{1-8}
\textbf{DeepWalk + kNN} &1.197 &0.818 &0.290 &0.877 &0.043900 &0.511 &0.436 \\\cmidrule{1-8}
\textbf{Random} &\ul{1.996} &\ul{1.663} &0.227 &0.481 &0.000546 &0.451 &0.308 \\\cmidrule{1-8}
\textbf{LightGCN} & & &0.148 &0.341 &\ul{0.000007} &\ul{0.297} &0.206 \\\cmidrule{1-1}\cmidrule{1-8}
\textbf{NGCF} & & &0.141 &0.328 &0.000011 &0.316 &0.198 \\\midrule
\textbf{KGAT} & & &\ul{0.139} &\ul{0.322} &0.000002 &0.438 &\ul{0.194} \\
\bottomrule
\end{tabular}
\end{table}

\begin{table}[!htp]\centering
\caption{MovieLens 100k results}\label{tab:ml100ktable}
\scriptsize
\begin{tabular}{lrrrrrrrr}\toprule
&\textbf{RMSE} &\textbf{MAE} &\textbf{Precision@10} &\textbf{Recall@10} &\textbf{MAP@10} &\textbf{NDCG} &\textbf{F1} \\\cmidrule{2-8}
\textbf{SVD} &0.935 &\textbf{0.737} &0.719 &0.633 &\textbf{0.016247} &0.508 &0.673 \\\cmidrule{1-8}
\textbf{SVD++} &0.962 &0.757 &0.702 &0.606 &0.004709 &0.441 &0.651 \\\cmidrule{1-8}
\textbf{CAMF-C} &\textbf{0.931} &0.738 &\textbf{0.726} &0.645 &0.002478 &0.375 &\textbf{0.683} \\\cmidrule{1-8}
\textbf{CAMF-CI} &0.978 &0.773 &0.713 &0.628 &0.000706 &0.402 &0.668 \\\cmidrule{1-8}
\textbf{IS-UserBased} &1.366 &1.041 &0.344 &\textbf{0.785} &0.012000 &0.527 &0.478 \\\cmidrule{1-8}
\textbf{NMF} &0.962 &0.757 &0.702 &0.606 &0.004910 &0.443 &0.650 \\\cmidrule{1-8}
\textbf{DeepWalk + kNN} &1.124 &0.872 &0.679 &0.659 &0.000706 &0.518 &0.669 \\\cmidrule{1-8}
\textbf{Random} &\ul{1.702} &\ul{1.391} &0.538 &0.415 &0.004435 &0.476 &0.469 \\\cmidrule{1-8}
\textbf{LightGCN} & & &\ul{0.249} &\ul{0.166} &0.000079 &\ul{0.295} &\ul{0.199} \\\cmidrule{1-1}\cmidrule{1-8}
\textbf{KGAT} & & &0.353 &0.232 &\ul{0.000044} &\textbf{0.650} &0.280 \\\midrule
\textbf{NGCF} & & &0.339 &0.223 &0.000254 &0.420 &0.269 \\
\bottomrule
\end{tabular}
\end{table}

\begin{table}[!htp]\centering
\caption{MovieLens 1m results}\label{tab:ml1mtable}
\scriptsize
\begin{tabular}{lrrrrrrrr}\toprule
&\textbf{RMSE} &\textbf{MAE} &\textbf{Precision@10} &\textbf{Recall@10} &\textbf{MAP@10} &\textbf{NDCG} &\textbf{F1} \\\cmidrule{2-8}
\textbf{SVD} &0.874 &0.686 &0.810 &0.567 &0.0179379 &0.520 &0.667 \\\cmidrule{1-8}
\textbf{SVD++} &\textbf{0.862} &\textbf{0.672} &\textbf{0.812} &0.567 &\textbf{0.0240000} &0.536 &0.668 \\\cmidrule{1-8}
\textbf{CAMF-C} &0.910 &0.719 &0.808 &0.563 &0.0000691 &0.338 &0.664 \\\cmidrule{1-8}
\textbf{CAMF-CI} &0.928 &0.734 &0.800 &0.563 &0.0000691 &0.411 &0.661 \\\cmidrule{1-8}
\textbf{IS-UserBased} &1.303 &0.977 &0.390 &\textbf{0.795} &0.0080911 &0.540 &0.524 \\\cmidrule{1-8}
\textbf{NMF} &0.917 &0.724 &0.796 &0.556 &0.0025421 &0.393 &0.655 \\\cmidrule{1-8}
\textbf{DeepWalk + kNN} &1.102 &0.852 &0.748 &0.605 &0.0218720 &0.516 &\textbf{0.669} \\\cmidrule{1-8}
\textbf{Random} &\ul{1.709} &\ul{1.397} &0.621 &0.373 &0.0024968 &0.460 &0.466 \\\cmidrule{1-8}
\textbf{LightGCN} & & &\ul{0.302} &\ul{0.136} &0.0000127 &\ul{0.336} &\ul{0.188} \\\cmidrule{1-1}\cmidrule{1-8}
\textbf{KGAT} & & &0.340 &0.155 &\ul{0.0000002} &\textbf{0.657} &0.212 \\\midrule
\textbf{NGCF} & & &0.336 &0.154 &0.0000392 &0.404 &0.211 \\
\bottomrule
\end{tabular}
\end{table}

\end{document}
